%% Time-stamp: <2018-10-18 20:24:12 (marc)>
\documentclass[xcolor=x11names,compress, mathserif]{beamer}

\newcommand{\hackspace}{\hspace{4.2mm}}
\newcommand{\showstudent}[1]{}
\newcommand\hmmax{0}
\newcommand\bmmax{0}





% talk/author information
\newcommand{\authorname}{Yingzhen Li}
\newcommand{\authoremail}{yingzhen.li@imperial.ac.uk}
\newcommand{\authortwitter}{liyzhen2}
\newcommand{\authoraffiliation}{
  Department of Computing\\Imperial
  College London}
\newcommand{\slidesettitle}{\imperialBlue{Bias-Variance Tradeoff}}
\newcommand{\footertitle}{Bias-Variance Tradeoff}
\newcommand{\location}{Imperial College London}
\newcommand{\talkDate}{Nov 18, 2022}



\date{\imperialGray{\talkDate}}




% load defaults
%\usepackage{../MarkMathCmds}
\selectcolormodel{rgb}
\usepackage{ifxetex,ifluatex}
\newif\ifxetexorluatex
\ifxetex
  \xetexorluatextrue
\else
  \ifluatex
    \xetexorluatextrue
  \else
    \xetexorluatexfalse
  \fi
\fi

\usepackage{textpos}
%\usepackage{arabtex}
\usepackage{tikz}
\usetikzlibrary{decorations.markings}
\usetikzlibrary{arrows}
\usetikzlibrary{shapes}
\usetikzlibrary{plotmarks}
\usetikzlibrary{mindmap,trees,backgrounds}

\tikzstyle{every picture}+=[remember picture]

%\usepackage{movie15}
% \usepackage{pdfpages}
%\usepackage{xmpmulti}

\usepackage{anyfontsize}
\usepackage{wrapfig}
\usepackage{animate}
\usepackage{multirow}
\usepackage{multimedia}
\usepackage{xmpmulti}
%\usepackage[latin9]{inputenc}
\usepackage[english]{babel}
\usepackage{scalefnt}
\usepackage{verbatim}
\usepackage{url}
% \usepackage{pgf,pgfarrows,pgfnodes}
\usepackage{textpos}
\usepackage[tight,ugly]{units}
\usepackage{url}
\usepackage{bbm}
\usepackage[english]{babel}
\usepackage{fancyhdr}
\usepackage{bm} % correct bold symbols, like \bm
\usepackage{amsmath}
\usepackage{amsfonts}
\usepackage{amssymb}
\usepackage{mathrsfs}
\usepackage{mathtools}
\usepackage{color}
\usepackage{cancel}
\usepackage{algorithm}
\usepackage{algpseudocode}
\usepackage{mathrsfs}
\usepackage{listings}
\usepackage{graphicx} % for pdf, bitmapped graphics files
\usepackage{mathtools}
\usepackage{units}
\usepackage{subfig}
\usepackage{enumerate}
\usepackage{natbib}
\usepackage{dsfont}


\ifxetexorluatex
\usepackage{fontspec}
\setmainfont[Scale=0.8]{OpenDyslexic-Regular}
\else
\usefonttheme{professionalfonts}
\fi

\renewcommand{\vec}[1]{{\boldsymbol{{#1}}}} % vector
\newcommand{\mat}[1]{{\boldsymbol{{#1}}}} % matrix
% \newcommand{\KL}[2]{\mathrm{KL}(#1\|#2)} % KL divergence
\newcommand{\R}[0]{\mathds{R}} % real numbers
\newcommand{\Z}[0]{\mathds{Z}} % integers
\newcommand{\tr}[0]{\text{tr}} % trace
% \newcommand{\inv}{^{-1}}
% \DeclareMathOperator*{\diag}{diag}
\newcommand{\E}{\mathds{E}} % expectation
\newcommand{\var}{\mathds{V}}
\newcommand{\gauss}[2]{\mathcal{N}\big(#1,\,#2\big)}
\newcommand{\gaussx}[3]{\mathcal{N}\big(#1\,|\,#2,\,#3\big)}
\newcommand{\gaussBig}[2]{\mathcal{N}\left(#1,\,#2\right)}
\newcommand{\gaussxBig}[3]{\mathcal{N}\left(#1\,\left|\,#2,\,#3\right.\right)}
\newcommand{\Ber}[0]{\mathrm{Ber}} % Bernoulli distribution
\DeclareMathOperator{\cov}{Cov}
\ifxetexorluatex
\renewcommand{\T}[0]{^\top}
\renewcommand{\d}[0]{\text{d}} % derivative
\else
\newcommand{\T}[0]{^\top}
\renewcommand{\d}[0]{\text{d}} % derivative
\fi
% calculus
\newcommand{\pdiff}[1]{\frac{\partial}{\partial #1}}
\newcommand{\pdiffF}[2]{\frac{\partial #1}{\partial #2}}
\newcommand{\diffF}[2]{\frac{{\d}#1}{{\d}#2}}
\newcommand{\diffFII}[2]{\frac{{\d}^2 #1}{{\d}#2^2}}
\newcommand{\diff}[1]{\frac{{\d}}{{\d}#1}}
\newcommand{\diffII}[1]{\frac{{\d}^2}{{\d}#1^2}}
\newcommand{\class}[0]{\mathcal{C}}

\newcommand{\idx}[1]{{(#1)}}
% \newcommand{\norm}[1]{\left\|#1\right\|}
\newcommand{\proj}[1]{\tilde{#1}}
\newcommand{\pcacoord}{z}
\newcommand{\pcacoordnew}{\zeta}
\newcommand{\latent}{z}
% \newcommand{\given}{\,|\,}
\newcommand{\genset}[1]{\mathrm{span}[#1]} % generating set
\newcommand{\set}[1]{\mathcal{#1}} % set
\newcommand{\fixgmfont}[1]{\scalebox{0.8}{#1}}



\usepackage{pifont}% http://ctan.org/pkg/pifont
\newcommand{\cmark}{{\color{green!40!black}\ding{51}}}%
\newcommand{\xmark}{{\color{red}\ding{55}}}%
\newcommand{\green}[1]{{\bf{\textcolor{green}{#1}}}}
\newcommand{\red}[1]{{\bf{\textcolor{red}{#1}}}}

\newcommand<>\red[1]{{\color#2[rgb]{1,0,0}#1}}
\newcommand<>\blue[1]{{\color#2[rgb]{0,0,1}#1}}
\newcommand<>\yellow[1]{{\color#2{camyellow}#1}}
\newcommand<>\green[1]{{\color#2[rgb]{0,0.6,0.0}#1}}
\newcommand<>\violet[1]{{\color#2[rgb]{0.6,0,0.6}#1}}
\newcommand<>\orange[1]{{\color#2[rgb]{1,0.5,0}#1}}
\newcommand<>\black[1]{{\color#2[rgb]{0,0,0}#1}}
\newcommand<>\steel[1]{{\color#2[rgb]{0,0,0.8}#1}}
\newcommand<>\darkblue[1]{{\color#2[rgb]{0,0,0.6}#1}}
\newcommand<>\lightblue[1]{{\color#2[rgb]{0.4,0.4,0.7}#1}}
\newcommand<>\gray[1]{{\color#2[rgb]{0.4,0.4,0.4}#1}}
\newcommand<>\greenish[1]{{\color#2[rgb]{0.45, 0.66, 0.45}#1}}
\newcommand<>\redish[1]{{\color#2[rgb]{0.7843    0.3706    0.3706}#1}}
\definecolor{redishTIKZ}{rgb}{0.7843, 0.3706, 0.3706}
\definecolor{imperialBlue}{rgb}{0.058, 0.219, 0.418}
\definecolor{aimsbrown}{rgb}{0.539, 0.117, 0.015}
% \definecolor{imperialGray}{rgb}{0.414, 0.488, 0.671 }
\definecolor{imperialGray}{RGB}{109,153, 204}
\definecolor{aimslightbrown}{RGB}{138,88,84}
\newcommand<>\imperialBlue[1]{{\color#2[rgb]{0.058, 0.219, 0.418}#1}}
\newcommand<>\aimsbrown[1]{{\color#2[rgb]{0.539, 0.117, 0.015}#1}}
%\newcommand<>\imperialGray[1]{{\color#2[rgb]{0.414, 0.488, 0.671}#1}}
\newcommand<>\imperialGray[1]{{\color#2[RGB]{109,153, 204}#1}}
\newcommand<>\aimslightbrown[1]{{\color#2[RGB]{138,88,84}#1}}
\newcommand<>\lightgray[1]{{\color#2[rgb]{0.8,0.8,0.8}#1}}
%\newcommand<>\highlightcolor[1]{{\color#2[rgb]{0,0,1}#1}}
\newcommand{\highlight}[1]{{\bf\steel{#1}}}
%\newcommand{\newblock}[0]{}

%\newcommand{\arrow}[0]{\includegraphics[height=5pt]{./figures/arrow}\hspace{3pt}}

\renewcommand{\emph}[1]{\textbf{\steel{{#1}}}}

\renewcommand{\alert}[1]{{\bf\red{{#1}}}}

\newcommand{\arrow}{
\begin{tikzpicture}
\draw [black!40!green, fill=black!40!green] (0,-0.12) -- (0,0.12) --
(0.15,0);
\draw [black!40!green, fill=black!40!green] (0.15,-0.12) -- (0.15,0.12) --
(0.3,0); 
\end{tikzpicture}
}

\geometry{left=0.45cm,top=0cm,right=0.45cm}


\newcommand{\logoimagepath}{./figures/imperial}
\newcommand{\highlightcolor}{blue!80!black}
%\newcommand{\headbarcolor}{imperialBlue}
\newcommand{\headbarcolor}{imperialBlue}
\institute{}

\newcommand{\coursetitle}{}

\newcommand{\slidesetsubtitle}{}
\newcommand{\slidesetnumber}{01}
\usefonttheme{professionalfonts}


\usetikzlibrary{decorations.fractals}
\input{../includes/tikzlibrarybayesnet.code.tex}
\input{../includes/tikzlibraryipe.code.tex}
\usetikzlibrary{matrix,positioning,decorations.pathreplacing}
\usetikzlibrary{calc,quotes,angles}
\usetikzlibrary{arrows, arrows.meta, patterns}

\usetikzlibrary{decorations.pathreplacing}
\tikzset{
    position label/.style={
       above = 3pt,
       text height = 2ex,
       text depth = 1ex
    }
}

% \usetikzlibrary{decorations.markings}
\tikzset{
  font={\fontsize{14pt}{12}\selectfont}
}



\useoutertheme[subsection=false,shadow]{miniframes}
\useinnertheme{default}
\usefonttheme{serif}
%\usepackage{palatino}
\usepackage{mathpazo}
%\usepackage{utopia}
\usepackage{stmaryrd} % for varodot, bigodot 
\usepackage{mathabx} % for \coAsterisk
%\usepackage{mnsymbol}
%\setbeamertemplate{itemize item}{\scriptsize\raise1.7pt\hbox{\donotcoloroutermaths$\Asterisk$}}
%\setbeamertemplate{itemize item}{\scriptsize\raise1.7pt\hbox{\donotcoloroutermaths$\varodot$}}
%\setbeamertemplate{itemize subitem}{\scriptsize\raise1.25pt\hbox{\donotcoloroutermaths$\rhd$}}

\usepackage{xifthen}% provides \isempty tesst

\setbeamerfont{title like}{shape=\scshape}
\setbeamerfont{frametitle}{}



\setbeamercolor*{lower separation line head}{bg=blue} 
\setbeamercolor*{normal text}{fg=black,bg=white} 
\setbeamercolor*{alerted text}{fg=red} 
\setbeamercolor*{example text}{fg=black} 
%\setbeamercolor*{frametitle}{fg=aimsbrown} 
\setbeamercolor*{frametitle}{fg=imperialBlue} 
\setbeamercolor*{structure}{fg=black} 
 
\setbeamercolor*{palette tertiary}{fg=black,bg=black!10} 
\setbeamercolor*{palette quaternary}{fg=black,bg=black!10} 

%\renewcommand{\(}{\begin{columns}}
%\renewcommand{\)}{\end{columns}}
%\newcommand{\<}[1]{\begin{column}{#1}}
%\renewcommand{\>}{\end{column}}

% ======================================
% custom commands 
\newcommand{\cemph}[1]{\textcolor{\highlightcolor}{#1}}
\newcommand{\calert}[1]{\textcolor{red}{#1}}

\setbeamertemplate{navigation symbols}{}
%\renewcommand\frametitle[1]{{\textsc{\Large \textcolor{\highlightcolor}{#1}}}\vspace{0.6cm}\par}

\setbeamertemplate{frametitle}
{
{\textsc\bf \insertframetitle}\vspace{0.2cm}\par
}


%%%%%%%%%%%%%%%%%%%%%%%%%%%%%%%%%%%%%%%%%%%%%%%%%%
\setbeamertemplate{headline}{% 
	\setbeamercolor{head1}{bg=\headbarcolor}
	 \hbox{%
  \begin{beamercolorbox}[wd=.01\paperwidth,ht=2.25ex,dp=50ex,center]{head1}%
  \fontsize{5}{5}\selectfont  
  \end{beamercolorbox}%
  }
  \vspace{-50ex}
}
\setbeamertemplate{footline}{
\begin{tiny}
\setbeamercolor{foot1}{fg=black,bg=gray!10}
\setbeamercolor{foot2}{fg=gray,bg=gray!15}
\setbeamercolor{foot3}{fg=gray,bg=gray!10}
\setbeamercolor{foot4}{fg=black,bg=gray!20}
\setbeamercolor{foot5}{fg=gray,bg=gray!15}
\setbeamercolor{foot6}{fg=black,bg=gray!20}

% taken from theme infolines and adapted
  \leavevmode%
  \hbox{%
  \begin{beamercolorbox}[wd=.45\paperwidth,ht=2.25ex,dp=1ex,center]{foot1}%
  \fontsize{5}{5}\selectfont
  \flushleft \hspace*{2ex}{\footertitle}
  \end{beamercolorbox}%
  % \begin{beamercolorbox}[wd=.08\paperwidth,ht=2.25ex,dp=1ex,center]{foot2}
  % \end{beamercolorbox}%
  %   \begin{beamercolorbox}[wd=.05\paperwidth,ht=2.25ex,dp=1ex,center]{foot3}
  % \end{beamercolorbox}%
    \begin{beamercolorbox}[wd=.45\paperwidth,ht=2.25ex,dp=1ex,center]{foot4}%
  \fontsize{5}{5}\selectfont
  \authorname\hspace{5mm}@\location, \talkDate%\ (\authorweb) 
  \end{beamercolorbox}%
  % \begin{beamercolorbox}[wd=.05\paperwidth,ht=2.25ex,dp=1ex,center]{foot5}
  % \end{beamercolorbox}%
  \begin{beamercolorbox}[wd=.1\paperwidth,ht=2.25ex,dp=1ex,right]{foot6}%
	\insertframenumber{}  \hspace*{2ex} 
  \end{beamercolorbox}}%
  \vskip0pt%
\end{tiny}
\vskip0pt
}


\setbeamertemplate{blocks}[rounded][shadow=false]


\newenvironment<>{myblock}[1]{%
  \begin{actionenv}#2%
      \def\insertblocktitle{#1}%
      \par%
      \mode<presentation>{%
%       \setbeamercolor{block title}{fg=black,bg=aimslightbrown!50!white}
      \setbeamercolor{block title}{fg=black,bg=imperialBlue!45!white}
       \setbeamercolor{block body}{fg=black,bg=gray!20}
       \setbeamercolor{itemize item}{fg=blue!40!white}
       \setbeamertemplate{itemize item}[triangle]
     }%
      \usebeamertemplate{block begin}}
    {\par\usebeamertemplate{block end}\end{actionenv}}

\newenvironment<>{myblock2}[1]{%
  \begin{actionenv}#2%
      \def\insertblocktitle{#1}%
      \par%
      \mode<presentation>{%
       \setbeamercolor{block title}{fg=white,bg=blue!80!black}
       \setbeamercolor{block body}{fg=black,bg=gray!20}
       \setbeamercolor{itemize item}{fg=green!60!black}
       \setbeamertemplate{itemize item}[triangle]
     }%
      \usebeamertemplate{block begin}}
    {\par\usebeamertemplate{block end}\end{actionenv}}

\gdef\colchar#1#2{%
  \tikz[baseline]{%
%  \node[anchor=base,inner sep=2pt,outer sep=0pt,fill = #2!20]
%  {\large{#1}};
  \node[anchor=base,inner sep=1pt,outer sep=0pt,fill = #2!20]
  {{\fontsize{11}{13}\selectfont #1}};
    }%
}%
\gdef\drawfontframe#1#2{%
  \tikz[baseline]{%
  \node[anchor=base,inner sep=2pt,outer sep=0pt,fill = #2!20] {#1};
    }%
  }%


\makeatletter
\let\@@magyar@captionfix\relax
\makeatother

%%% Local Variables:
%%% mode: latex
%%% TeX-master: "2018-09-arusha-linear-regression"
%%% End:

\usepackage{amssymb, amsmath, amsthm}
\usepackage{bm}
\DeclareMathOperator*{\argmax}{arg\,max}
\DeclareMathOperator*{\argmin}{arg\,min}

\newcommand{\bo}{\omega}
\newcommand{\KL}{\text{KL}}
\newcommand{\train}{\text{train}}
\newcommand{\D}{\mathcal{D}}
\newcommand{\softmax}{\text{Softmax}}

\newcommand{\logsumexp}{\text{log-sum-exp}}

%\newcommand{\R}{\mathbb{R}}
\newcommand{\N}{\mathcal{N}}
\newcommand{\cL}{\mathcal{L}}
\newcommand{\cO}{\mathcal{O}}
\newcommand{\svert}{~|~}
\newcommand{\td}{\text{d}}
\newcommand{\f}{\mathbf{f}}
\newcommand{\x}{\bm{x}}
\newcommand{\Bb}{\mathbf{b}}
\newcommand{\BB}{\mathbf{B}}
\newcommand{\BS}{\mathbf{S}}
\newcommand{\BA}{\mathbf{A}}
\newcommand{\BQ}{\mathbf{Q}}
\newcommand{\BP}{\mathbf{P}}
\newcommand{\BU}{\mathbf{U}}
\newcommand{\BV}{\mathbf{V}}
\newcommand{\Bg}{\mathbf{g}}
%\newcommand{\sBb}{\mathtt{b}}
\newcommand{\sBb}{\mathtt{z}}
\newcommand{\bx}{\overline{\x}}
\newcommand{\bb}{\overline{b}}
\newcommand{\y}{\mathbf{y}}
\newcommand{\z}{\bm{z}}
\newcommand{\bv}{\bm{v}}
\newcommand{\bV}{\mathbf{V}}
\newcommand{\bk}{\mathbf{k}}
\newcommand{\w}{\mathbf{w}}
\newcommand{\W}{\mathbf{W}}
\newcommand{\ba}{\mathbf{a}}
\newcommand{\m}{\mathbf{m}}
\newcommand{\ls}{\mathbf{l}}
\newcommand{\bL}{\mathbf{L}}
\newcommand{\A}{\mathbf{A}}
\newcommand{\X}{\mathbf{X}}
\newcommand{\Y}{\mathbf{Y}}
\newcommand{\F}{\mathbf{F}}
%\newcommand{\I}{\mathbf{I}}
\newcommand{\M}{\mathbf{M}}
\newcommand{\p}{\mathbf{p}}
\newcommand{\bp}{\overline{\p}}
\newcommand{\bz}{\mathbf{0}}
\newcommand{\bepsilon}{\text{\boldmath$\epsilon$}}
\newcommand{\bgamma}{\text{\boldmath$\gamma$}}
\newcommand{\s}{\mathbf{s}}
\newcommand{\Unif}{\text{Unif}}
\newcommand{\boh}{\widehat{\text{\boldmath$\omega$}}}
\newcommand{\bsigma}{\text{\boldmath$\sigma$}}
\newcommand{\bSigma}{\text{\boldmath$\Sigma$}}
\newcommand{\bmu}{\text{\boldmath$\mu$}}
\newcommand{\bphi}{\text{\boldmath$\phi$}}
\newcommand{\K}{\mathbf{K}}
\newcommand{\Kh}{\widehat{\mathbf{K}}}
\newcommand{\Cov}{\text{Cov}}
\newcommand{\Var}{\text{Var}}
%\newcommand{\tr}{\text{tr}}
\newcommand{\tdet}{\text{det}}
\newcommand{\diag}{\text{diag}}
% \newcommand{\KL}{\text{KL}}
\newcommand{\ind}{\mathds{1}}
\newcommand{\bc}{\mathbf{c}}
\newcommand{\reg}{\eta}
\newcommand{\weightdecay}{\lambda}
\newcommand{\h}{\mathbf{h}}

\newcommand{\ci}[0]{\perp\!\!\!\perp} % conditional independence

% variables
\newcommand{\mparam}{\bm{\theta}}	% model param
\newcommand{\vparam}{\bm{\phi}}	% variational param

% gradient approximation part
\newcommand{\hparam}{\bm{\varphi}}
\newcommand{\Xb}{\mathbb{X}}
\newcommand{\hgrad}{\overline{\nabla_{\x} \h}}
\newcommand{\Hmatrix}{\mathbf{H}}
\newcommand{\Grad}{\mathbf{G}}
\newcommand{\g}{\bm{g}}
\newcommand{\noise}{\bm{\epsilon}}
\newcommand{\data}{\mathcal{D}}





\newif\iflattersubsect

\AtBeginSection[] {
    \begin{frame}<beamer>
    \frametitle{Overview} %
    \tableofcontents[currentsection]  
    \end{frame}
    \lattersubsectfalse
}

\AtBeginSubsection[] {
    \iflattersubsect
    \begin{frame}<Coming Next>
    \frametitle{Overview} %
    \tableofcontents[currentsubsection]  
    \end{frame}
    \fi
    \lattersubsecttrue
}

\begin{document}


%%%%%%%%%%%%%%%%%%%%%%%%%%%%%%%%%%%%%%%%%%%%%%%%%%%%%%

{\setbeamertemplate{footline}{}
\begin{frame}
\title{\slidesettitle}
%\subtitle{SUBTITLE}
\author{\footnotesize
  \textbf{\authorname}
 }

 %%% LOGO

% \begin{flushright}
%   % \begin{columns}
%   %   \column{0.5\hsize}
%   %   \column{0.45\hsize}
%\includegraphics[height = 8mm]{./figures/qla}\hspace{2mm}
%     \includegraphics[height = 8mm]{./figures/aims-rwanda}\\[2mm]
%\includegraphics[height = 8mm]{./figures/imperial}
%%\end{columns}
%\end{flushright}

\vspace{-0cm}
%\begin{flushleft}
%\vspace{-1.5cm}{\small \textcolor{blue}{\coursetitle}}\\\vspace{2cm}
{\huge \slidesettitle \ifthenelse{\equal{\slidesetsubtitle}{}}%
    {}% if #1 is empty
    {: \\ {\large \slidesetsubtitle}}% if #1 is not empty
    } \\    
    %\vspace{20pt}
%\end{flushleft}
  
 
% this is all stuff below the talk title. make two columns, just in
% case you want to have a picture or a second affiliation here 
\begin{columns}[t]
\column{0.8\hsize}
%\begin{flushleft}
\begin{columns}[t]
\column{0.6\hsize}
\insertauthor \\[2mm]
\authoraffiliation\\[2mm]
\column{0.25\hsize}
\\[2mm]
\includegraphics[height = 0.3cm]{./figures-general/twitter}{\small @\authortwitter}\\[-1mm]
\mbox{\small \url{\authoremail}}
\end{columns}
\column{0.14\hsize}
\end{columns}
% \authorweb\\
\vspace{7mm}
% \aimslightbrown{The Nelson Mandela African Institute of Science and
%   Technology\\Arusha, Tanzania}\\[2mm]
\insertdate
%\end{flushleft}
\end{frame}
}

%%% Local Variables:
%%% mode: latex
%%% TeX-master: t
%%% End:

\linespread{1.2}

%\section{Overfitting}


%%%%%%%%%%%%%%%%%%%%%%%%%%%%%%%%%%%%%%%%

\begin{frame}
\frametitle{Regression with non-linear features}
\begin{minipage}{0.65\linewidth}
For \alert{non-linear regression}:
\begin{itemize}
	\item Key idea: using a non-linear feature mapping: \alert{$\phi(\cdot): \mathbb{R}^D \rightarrow \mathbb{R}^p$}
	\item The non-linear regression model:
	$$f(\x, \mparam) = \textcolor{red}{\phi(\x)}^\top \mparam$$
	$$y = f(\x, \mparam) + \epsilon, \ \epsilon \sim \mathcal{N}(0, \sigma^2)$$
	\item Recover linear regression when $\phi(\x) = \x$
\end{itemize}
\end{minipage}
\hfill
\begin{minipage}{0.3\linewidth}
\begin{figure}
\centering
\includegraphics[width=0.9\linewidth]{figures-bias-var-tradeoff/non_linear_regression.png}
\end{figure}
$$\phi(x) = [1, x, x^2]$$
\end{minipage}

\end{frame}


\begin{frame}{Overfitting}
  \begin{figure}
    \centering
    \includegraphics[width = 0.6\hsize]{./figures-bias-var-tradeoff/demo_regression_mle_10.pdf}
  \end{figure}
\begin{equation}
\phi(x) = [1\,\, x\,\, x^2\,\, x^3, \dots]^{\top}
\end{equation}
When the model is too flexible, risk of overfitting!
\end{frame}


\begin{frame}{Overfitting}

To help avoid overfitting:
\begin{itemize}
	\item Choose model with the right complexity (using validation data)
	\item \alert{Regularise the model} (this lecture)
		\begin{itemize}
		\item There's a bias-variance tradeoff here!
		\end{itemize}
\end{itemize}

\end{frame}

\begin{frame}
\frametitle{Regression with non-linear features}
Fitting regression model with a \emph{regulariser}:
$$L(\mparam) = \frac{1}{2 \sigma^2} \sum_{n} (f(\x_n, \mparam ) - y_n)^2 + \frac{\lambda}{2} || \mparam ||_2^2$$
\begin{itemize}
	\item \alert{Write $\Phi = [\phi(\x_1), ..., \phi(\x_N)]^\top \in \mathbb{R}^{N \times p}$}:
	$$\mparam^*_R = \argmin_{\mparam \in \Theta} \frac{1}{2 \sigma^2} || \y - \textcolor{red}{\Phi} \mparam ||_2^2  + \frac{\lambda}{2} ||\mparam||_2^2$$
	\item Optimal solution for $\mparam$:
	$$\mparam^*_R = (\sigma^2 \lambda \mathbf{I} + \textcolor{red}{ \Phi^\top \Phi })^{-1} \textcolor{red}{\Phi^\top} \y$$
\end{itemize}

\end{frame}


%%%%%%%%%

\begin{frame}
\frametitle{Intuition behind the regulariser}

Regression with polynomial functions as an example:
$$f(\x, \mparam) = \sum_{i=1}^p \theta_i \x^{i-1}$$

\begin{minipage}{0.45\linewidth}
\begin{figure}
\includegraphics[width=0.9\linewidth]{figures-bias-var-tradeoff/demo_regression_mle_5.pdf}
\end{figure}
\end{minipage}
\hfill
\begin{minipage}{0.45\linewidth}
\begin{figure}
\includegraphics[width=0.9\linewidth]{figures-bias-var-tradeoff/demo_regression_mle_10.pdf}
\end{figure}
\end{minipage}
%


Several solutions fit the training data almost equally well.

$\Rightarrow$ How to choose a model?

\end{frame}


\begin{frame}
\frametitle{Intuition behind the regulariser}

Regression with polynomial functions as an example:
$$f(\x, \mparam) = \sum_{i=1}^p \theta_i \x^{i-1}$$

The $\ell_2$ regulariser used in ridge regression:
$$R(\mparam) = || \mparam ||_2^2 = \sum_{i=1}^p \mparam_i^2$$ 
\begin{itemize}
\item shrinks elements of $\mparam$ to zero \pause
\item if $\theta_i = 0$, then feature $\x^{i-1}$ is not in use \\ $\Rightarrow$ simpler model!
\item Ridge regression balances between data fit and model simplicity
\end{itemize}

\end{frame}

\begin{frame}
\frametitle{Intuition behind the regulariser}

Potential questions on using regularisers:
\begin{itemize}
	\item Do we obtain the ground truth parameters?
	\item Why regularised models can sometimes better fit the data (in terms of test error)?
\end{itemize}

To answer these: study Bias-variance tradeoff

\end{frame}


%%%%%%%%%% bias variance trade off %%%%%%%

\begin{frame}{Bias-variance tradeoff}

The general concept of Bias-variance tradeoff:
\begin{itemize}
	\item Suppose there is an unknown quantity $x_0$ that we like to estimate; 
	\item Assume we have a \emph{stochastic estimator} $X$ for $x_0$;
	\item Calculating the expected $\ell_2$ error:
	$$\mathbb{E}[ || X - x_0 ||_2^2] = \underbrace{|| \mathbb{E}[X] - x_0 ||_2^2}_{bias^2} + \underbrace{\tr[\mathbb{V}[X]]}_{variance}$$
	\begin{itemize}
		\item \emph{Unbiased} estimator: $bias = 0 \quad \Rightarrow \quad \mathbb{E}[X] = x_0$ 
		\item \emph{Low variance} estimator: variance is small
	\end{itemize}
\end{itemize}

\end{frame}


\begin{frame}{Bias-variance tradeoff}
Visualising Bias-variance trade-off:
\vspace{1em}

\begin{minipage}{0.4\linewidth}
\includegraphics[width=1\linewidth]{figures-bias-var-tradeoff/bias_variance_vis.png}
\end{minipage}
\hfill
\begin{minipage}{0.55\linewidth}
\includegraphics[width=1\linewidth]{figures-bias-var-tradeoff/biasvariance.png}
\end{minipage}

\hfill \tiny{Figures from \url{http://scott.fortmann-roe.com/docs/BiasVariance.html}}
\end{frame}

%%%%%%% regression example %%%%%

\begin{frame}
\frametitle{Bias-variance tradeoff in regression}

Fact for Ridge regression (linear regression + $\ell_2$ regulariser):

Ridge regression returns estimator of $\mparam$ which 
\begin{itemize}
\item is \emph{biased} (when $\lambda > 0$, unbiased only when $\lambda = 0$)
\item has \emph{smaller variance} than the MLE solution
\end{itemize}

With good choices of $\lambda > 0$, \alert{the (expected) test error can be reduced}.

\end{frame}


\begin{frame}{Bias-variance tradeoff in regression}
How bias-variance tradeoff is relevant to overfitting:

\alert{Expected} prediction error for $\mparam^* = \mparam^*(\mathcal{D})$ over $\mathcal{D} \sim \pi^N$:
\begin{equation*}
\begin{aligned}
error_{pred}(\mparam^*) &= \mathbb{E}_{\data \sim \pi^N} [ \mathbb{E}_{(\x_{test}, y_{test}) \sim \pi}[|| y_{test} - f(\x_{test}, \mparam^*(\data)) ||_2^2] ] \\
&= \mathbb{E}_{\x_{test}} [\phi(\x_{test})^{\top} \textcolor{red}{ Error(\mparam^*)} \phi(\x_{test})] + \sigma^2 \\
\end{aligned}
\end{equation*} 

\begin{equation*}
\begin{aligned}
Error(\mparam^*) &= \mathbb{E}_{\data \sim \pi^N}[ ( \mparam^*(\data) - \mparam_0 ) (\mparam^*(\data) - \mparam_0)^\top ] \\
&:=  \Bb(\mparam^*) \Bb(\mparam^*)^\top + \BV(\mparam^*)
\end{aligned}
\end{equation*}

\begin{equation*}
\begin{aligned}
\text{bias:} \quad & \Bb(\mparam^*) = \mathbb{E}_{\data \sim \pi^N}[\mparam^*(\data)] - \mparam_0 \\
\text{variance:} \quad & \BV(\mparam^*) = \mathbb{V}_{\data \sim \pi^N}[\mparam^*(\data)]
\end{aligned}
\end{equation*}

\end{frame}


\begin{frame}{Bias-variance tradeoff in regression}
How bias-variance tradeoff is relevant to overfitting:

\alert{Expected} prediction error for $\mparam^* = \mparam^*(\mathcal{D})$ over $\mathcal{D} \sim \pi^N$:
\begin{equation*}
\begin{aligned}
error_{pred}(\mparam^*) &= \mathbb{E}_{\data \sim \pi^N} [ \mathbb{E}_{(\x_{test}, y_{test}) \sim \pi}[|| y_{test} - f(\x_{test}, \mparam^*(\data)) ||_2^2] ] \\
&= \mathbb{E}_{\x_{test}} [\phi(\x_{test})^{\top} \textcolor{red}{ Error(\mparam^*)} \phi(\x_{test})] + \sigma^2 \\
\end{aligned}
\end{equation*} 
%
\begin{equation*}
Error(\mparam^*) = \Bb(\mparam^*) \Bb(\mparam^*)^\top + \BV(\mparam^*)
\end{equation*}

\pause


\vspace{1em}
If we have two estimators $\mparam_1$, $\mparam_2$ based on $\mathcal{D} \sim \pi^N$:
\begin{equation*}
Error(\mparam_1) \preceq Error(\mparam_2) \quad \Rightarrow \quad error_{pred}(\mparam_1) \leq error_{pred}(\mparam_2)
\end{equation*} 
\vspace{-2em}
\begin{itemize}
\item Smaller estimation error $\Rightarrow$ smaller prediction error
\item Depends on bias-variance trade-off
\end{itemize}

\end{frame}


\begin{frame}{Linear regression returns an unbiased estimator}
Reminder for solving linear/ridge regression:
\begin{itemize}
	\item Write $\Phi = [\phi(\x_1), ..., \phi(\x_N)]^\top \in \mathbb{R}^{N \times p}$:
	$$\mparam^* = \argmin_{\mparam \in \Theta} \frac{1}{2 \sigma^2} || \y - \Phi \mparam ||_2^2  + \frac{\lambda}{2} ||\mparam||_2^2$$
	\item Optimal solution for $\mparam$ in ridge regression:
	$$\mparam^*_R = (\sigma^2 \lambda \mathbf{I} + \Phi^\top \Phi )^{-1} \Phi^\top \y$$
	\item Optimal solution for $\mparam$ in linear regression ($\lambda = 0$):
	$$\mparam^*_L = (\Phi^\top \Phi )^{-1} \Phi^\top \y$$
\end{itemize}

\end{frame}

\begin{frame}{Linear regression returns an unbiased estimator}
Optimal solution for linear regression: $\mparam^*_L = (\Phi^\top \Phi )^{-1} \Phi^\top \y$

\begin{itemize}
\item Assuming no model error:
$$\y = \Phi \mparam_0 + \bm{\epsilon}, \quad \bm{\epsilon} = [\epsilon_1, ..., \epsilon_N]^\top, \quad \epsilon_n \sim \mathcal{N}(0, \sigma^2)$$
\item Leading to optimal solution as:
$\mparam^*_L = (\Phi^\top \Phi )^{-1} \Phi^\top (\Phi \mparam_0 + \bm{\epsilon})$
\item \alert{Unbiased estimator}:
\begin{equation*}
\begin{aligned}
\mathbb{E}_{\data \sim \pi^N}[\mparam^*_L(\data)] &= \mathbb{E}_{\data \sim \pi^N}[(\Phi^\top \Phi )^{-1} \Phi^\top (\Phi \mparam_0 + \bm{\epsilon})] 
= \mparam_0
\end{aligned}
\end{equation*}
\end{itemize}

\end{frame}


\begin{frame}{Ridge regression returns a biased estimator}

The ridge regression estimator: $\mparam^*_R = (\sigma^2 \lambda \mathbf{I} + \Phi^\top \Phi )^{-1} \Phi^\top (\Phi \mparam_0 + \bm{\epsilon})$

\begin{itemize}
\item Compute the mean of $\mparam^*_R$ for $\data \sim \pi^N$: 
\begin{equation*}
\mathbb{E}_{\data \sim \pi^N}[\mparam^*_R(\data)] = ( \sigma^2 \lambda \mathbf{I} + \Phi^\top \Phi )^{-1} \Phi^\top \Phi \mparam_0
\end{equation*}
$\Rightarrow$ Ridge regression returns a \alert{biased estimator} \pause

\item Compute the covariance matrix of $\mparam^*_R$ for $\data \sim \pi^N$:
\begin{equation*}
\begin{aligned}
\mathbb{V}_{\data \sim \pi^N}[\mparam^*_R(\data)] &= \mathbb{V}_{\data \sim \pi^N}[(\sigma^2 \lambda \mathbf{I} + \Phi^\top \Phi )^{-1} \Phi^\top (\Phi \mparam_0 + \bm{\epsilon})] \\
&=\mathbb{V}_{\data \sim \pi^N}[(\sigma^2 \lambda \mathbf{I} + \Phi^\top \Phi )^{-1} \Phi^\top \bm{\epsilon}] \\
&= \sigma^{2} (\sigma^2 \lambda \mathbf{I} + \Phi^\top \Phi )^{-1} \Phi^\top \Phi (\sigma^2 \lambda \mathbf{I} + \Phi^\top \Phi )^{-1} \\
\end{aligned}
\end{equation*}

\end{itemize}

\end{frame}


\begin{frame}{Ridge regression returns a biased estimator}

Bias of ridge regression estimator ($\lambda > 0$):
\begin{equation*}
\begin{aligned}
\Bb(\lambda) := \mathbb{E}_{\data \sim \pi^N}[\mparam^*_R(\data)] - \mparam_0 &= ( \sigma^2 \lambda \mathbf{I} + \Phi^\top \Phi )^{-1} \Phi^\top \Phi \mparam_0 - \mparam_0 \\
&= -\sigma^2 \lambda ( \sigma^2 \lambda \mathbf{I} + \Phi^\top \Phi )^{-1} \mparam_0
\end{aligned}
\end{equation*}

Bias of linear regression estimator ($\lambda = 0$): 
$$\Bb(0) = \bm{0}$$ \pause

Variance of ridge regression estimator ($\lambda > 0$): 
\begin{equation*}
\begin{aligned}
\BV(\lambda) &:= \sigma^2 (\sigma^2 \lambda \mathbf{I} + \Phi^\top \Phi )^{-1} \Phi^\top \Phi (\sigma^2 \lambda \mathbf{I} + \Phi^\top \Phi )^{-1}
\end{aligned}
\end{equation*}


Variance of linear regression estimator ($\lambda = 0$): 
$$\BV(0) = \sigma^2 (\Phi^\top \Phi )^{-1}$$

\end{frame}


\begin{frame}{Ridge regression can perform better in prediction}

\alert{Expected} prediction error of ridge regression ($\lambda > 0$):
\begin{equation*}
\begin{aligned}
error_{pred}(\mparam^*_R) &= \mathbb{E}_{\x_{test}} [\phi(\x_{test})^{\top} \textcolor{red}{ Error(\mparam_R^*) } \phi(\x_{test})] + \sigma^2 \\
Error(\mparam_R^*) &= \Bb(\lambda) \Bb(\lambda)^{\top} + \BV(\lambda)
\end{aligned}
\end{equation*}

\alert{Expected} prediction error of linear regression ($\lambda = 0$):
\begin{equation*}
\begin{aligned}
error_{pred}(\mparam^*_L) &= \mathbb{E}_{\x_{test}} [\phi(\x_{test})^{\top} \textcolor{red}{ Error(\mparam_L^*) } \phi(\x_{test})] + \sigma^2 \\
Error(\mparam_L^*) &= \Bb(0) \Bb(0)^{\top} + \BV(0) \textcolor{red}{= \BV(0)}
\end{aligned}
\end{equation*} \pause

This means if there exists some $\lambda > 0$ such that: 
\begin{equation*}
\Bb(\lambda) \Bb(\lambda)^\top + \BV(\lambda) \preceq \BV(0) \quad \Rightarrow \quad error_{pred}(\mparam_R^*) \leq error_{pred}(\mparam_L^*)
\end{equation*}

\end{frame}

\begin{frame}{Ridge regression can perform better in prediction}

Derivations exercises in the exercise sheet: \\

\begin{itemize}
\item For $\lambda >0 $, we can show reduced variance:
	$$\BV(\lambda) - \BV(0) \preceq 0$$

\item We can choose e.g.~$0 \leq \lambda \leq \frac{2}{|| \mparam_0 ||_2^2}$ which leads to:
$$ \Bb(\lambda) \Bb(\lambda)^\top + \BV(\lambda) \preceq \BV(0) \quad \Rightarrow \quad error_{pred}(\mparam_R^*) \leq error_{pred}(\mparam_L^*) $$

\item[$\Rightarrow$] The smaller prediction error of $\mparam_R^*$ comes from \\ having \alert{smaller variance} in parameter estimate!

\item[$\Rightarrow$] $\lambda$ needs to be chosen carefully so that \emph{the bias is not too large}

\end{itemize}


\end{frame}

\begin{frame}
\frametitle{Bias-variance tradeoff in regression: Summary}
\begin{center}
Ridge regression can return estimator of $\mparam$ with \alert{smaller variance}.

\vspace{1em}
In such case the (expected) test error can be reduced.
\end{center}

\begin{itemize}
\item $\mparam^*_R$ is a biased estimator of $\mparam_0$ when $\lambda > 0$
\item There exists $\lambda$ such that 
\begin{itemize}
	\item Variance is smaller: $\BV(\lambda) \preceq \BV(0)$ 
	\item Bias is not too large
\end{itemize}
\item ... and it leads to $error_{pred}(\mparam^*_R) \leq error_{pred}(\mparam^*_L)$
\end{itemize}
\end{frame}

\begin{frame}{Bias-variance tradeoff}
Visualising Bias-variance trade-off:
\vspace{1em}

\begin{minipage}{0.4\linewidth}
\includegraphics[width=1\linewidth]{figures-bias-var-tradeoff/bias_variance_vis.png}
\end{minipage}
\hfill
\begin{minipage}{0.55\linewidth}
\includegraphics[width=1\linewidth]{figures-bias-var-tradeoff/biasvariance.png}
\end{minipage}

\hfill \tiny{Figures from \url{http://scott.fortmann-roe.com/docs/BiasVariance.html}}
\end{frame}


%%%%%%%%%%%%%%%%%%%%%%%%%%%%%%%%%%%%%%%%



\end{document}
%%% Local Variables: 
%%% mode: latex
%%% TeX-master: t
%%% End: 
