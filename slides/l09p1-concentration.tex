%% Time-stamp: <2018-10-18 20:24:12 (marc)>
\documentclass[xcolor=x11names,compress,mathserif,handout]{beamer}

\newcommand{\hackspace}{\hspace{4.2mm}}
\newcommand{\showstudent}[1]{}
\newcommand\hmmax{0}
\newcommand\bmmax{0}


\usepackage{../includes/MarkMathCmds}





% talk/author information
\newcommand{\authorname}{Mark van der Wilk}
\newcommand{\authoremail}{m.vdwilk@imperial.ac.uk}
\newcommand{\authoraffiliation}{
  Department of Computing\\Imperial
  College London}
\newcommand{\authortwitter}{markvanderwilk}
\newcommand{\slidesettitle}{\imperialBlue{Concentration Inequalities}}
\newcommand{\footertitle}{Overfitting \& Generalisation}
\newcommand{\location}{Imperial College London}
\newcommand{\talkDate}{November 7, 2022}



\date{\imperialGray{\talkDate}}




% load defaults
\selectcolormodel{rgb}
\usepackage{ifxetex,ifluatex}
\newif\ifxetexorluatex
\ifxetex
  \xetexorluatextrue
\else
  \ifluatex
    \xetexorluatextrue
  \else
    \xetexorluatexfalse
  \fi
\fi

\usepackage{textpos}
%\usepackage{arabtex}
\usepackage{tikz}
\usetikzlibrary{decorations.markings}
\usetikzlibrary{arrows}
\usetikzlibrary{shapes}
\usetikzlibrary{plotmarks}
\usetikzlibrary{mindmap,trees,backgrounds}

\tikzstyle{every picture}+=[remember picture]

%\usepackage{movie15}
% \usepackage{pdfpages}
%\usepackage{xmpmulti}

\usepackage{anyfontsize}
\usepackage{wrapfig}
\usepackage{animate}
\usepackage{multirow}
\usepackage{multimedia}
\usepackage{xmpmulti}
%\usepackage[latin9]{inputenc}
\usepackage[english]{babel}
\usepackage{scalefnt}
\usepackage{verbatim}
\usepackage{url}
% \usepackage{pgf,pgfarrows,pgfnodes}
\usepackage{textpos}
\usepackage[tight,ugly]{units}
\usepackage{url}
\usepackage{bbm}
\usepackage[english]{babel}
\usepackage{fancyhdr}
\usepackage{bm} % correct bold symbols, like \bm
\usepackage{amsmath}
\usepackage{amsfonts}
\usepackage{amssymb}
\usepackage{mathrsfs}
\usepackage{mathtools}
\usepackage{color}
\usepackage{cancel}
\usepackage{algorithm}
\usepackage{algpseudocode}
\usepackage{mathrsfs}
\usepackage{listings}
\usepackage{graphicx} % for pdf, bitmapped graphics files
\usepackage{mathtools}
\usepackage{units}
\usepackage{subfig}
\usepackage{enumerate}
\usepackage{natbib}
\usepackage{dsfont}


\ifxetexorluatex
\usepackage{fontspec}
\setmainfont[Scale=0.8]{OpenDyslexic-Regular}
\else
\usefonttheme{professionalfonts}
\fi

\renewcommand{\vec}[1]{{\boldsymbol{{#1}}}} % vector
\newcommand{\mat}[1]{{\boldsymbol{{#1}}}} % matrix
% \newcommand{\KL}[2]{\mathrm{KL}(#1\|#2)} % KL divergence
\newcommand{\R}[0]{\mathds{R}} % real numbers
\newcommand{\Z}[0]{\mathds{Z}} % integers
\newcommand{\tr}[0]{\text{tr}} % trace
% \newcommand{\inv}{^{-1}}
% \DeclareMathOperator*{\diag}{diag}
\newcommand{\E}{\mathds{E}} % expectation
\newcommand{\var}{\mathds{V}}
\newcommand{\gauss}[2]{\mathcal{N}\big(#1,\,#2\big)}
\newcommand{\gaussx}[3]{\mathcal{N}\big(#1\,|\,#2,\,#3\big)}
\newcommand{\gaussBig}[2]{\mathcal{N}\left(#1,\,#2\right)}
\newcommand{\gaussxBig}[3]{\mathcal{N}\left(#1\,\left|\,#2,\,#3\right.\right)}
\newcommand{\Ber}[0]{\mathrm{Ber}} % Bernoulli distribution
\DeclareMathOperator{\cov}{Cov}
\ifxetexorluatex
\renewcommand{\T}[0]{^\top}
\renewcommand{\d}[0]{\text{d}} % derivative
\else
\newcommand{\T}[0]{^\top}
\renewcommand{\d}[0]{\text{d}} % derivative
\fi
% calculus
\newcommand{\pdiff}[1]{\frac{\partial}{\partial #1}}
\newcommand{\pdiffF}[2]{\frac{\partial #1}{\partial #2}}
\newcommand{\diffF}[2]{\frac{{\d}#1}{{\d}#2}}
\newcommand{\diffFII}[2]{\frac{{\d}^2 #1}{{\d}#2^2}}
\newcommand{\diff}[1]{\frac{{\d}}{{\d}#1}}
\newcommand{\diffII}[1]{\frac{{\d}^2}{{\d}#1^2}}
\newcommand{\class}[0]{\mathcal{C}}

\newcommand{\idx}[1]{{(#1)}}
% \newcommand{\norm}[1]{\left\|#1\right\|}
\newcommand{\proj}[1]{\tilde{#1}}
\newcommand{\pcacoord}{z}
\newcommand{\pcacoordnew}{\zeta}
\newcommand{\latent}{z}
% \newcommand{\given}{\,|\,}
\newcommand{\genset}[1]{\mathrm{span}[#1]} % generating set
\newcommand{\set}[1]{\mathcal{#1}} % set
\newcommand{\fixgmfont}[1]{\scalebox{0.8}{#1}}



\usepackage{pifont}% http://ctan.org/pkg/pifont
\newcommand{\cmark}{{\color{green!40!black}\ding{51}}}%
\newcommand{\xmark}{{\color{red}\ding{55}}}%
\newcommand{\green}[1]{{\bf{\textcolor{green}{#1}}}}
\newcommand{\red}[1]{{\bf{\textcolor{red}{#1}}}}

\newcommand<>\red[1]{{\color#2[rgb]{1,0,0}#1}}
\newcommand<>\blue[1]{{\color#2[rgb]{0,0,1}#1}}
\newcommand<>\yellow[1]{{\color#2{camyellow}#1}}
\newcommand<>\green[1]{{\color#2[rgb]{0,0.6,0.0}#1}}
\newcommand<>\violet[1]{{\color#2[rgb]{0.6,0,0.6}#1}}
\newcommand<>\orange[1]{{\color#2[rgb]{1,0.5,0}#1}}
\newcommand<>\black[1]{{\color#2[rgb]{0,0,0}#1}}
\newcommand<>\steel[1]{{\color#2[rgb]{0,0,0.8}#1}}
\newcommand<>\darkblue[1]{{\color#2[rgb]{0,0,0.6}#1}}
\newcommand<>\lightblue[1]{{\color#2[rgb]{0.4,0.4,0.7}#1}}
\newcommand<>\gray[1]{{\color#2[rgb]{0.4,0.4,0.4}#1}}
\newcommand<>\greenish[1]{{\color#2[rgb]{0.45, 0.66, 0.45}#1}}
\newcommand<>\redish[1]{{\color#2[rgb]{0.7843    0.3706    0.3706}#1}}
\definecolor{redishTIKZ}{rgb}{0.7843, 0.3706, 0.3706}
\definecolor{imperialBlue}{rgb}{0.058, 0.219, 0.418}
\definecolor{aimsbrown}{rgb}{0.539, 0.117, 0.015}
% \definecolor{imperialGray}{rgb}{0.414, 0.488, 0.671 }
\definecolor{imperialGray}{RGB}{109,153, 204}
\definecolor{aimslightbrown}{RGB}{138,88,84}
\newcommand<>\imperialBlue[1]{{\color#2[rgb]{0.058, 0.219, 0.418}#1}}
\newcommand<>\aimsbrown[1]{{\color#2[rgb]{0.539, 0.117, 0.015}#1}}
%\newcommand<>\imperialGray[1]{{\color#2[rgb]{0.414, 0.488, 0.671}#1}}
\newcommand<>\imperialGray[1]{{\color#2[RGB]{109,153, 204}#1}}
\newcommand<>\aimslightbrown[1]{{\color#2[RGB]{138,88,84}#1}}
\newcommand<>\lightgray[1]{{\color#2[rgb]{0.8,0.8,0.8}#1}}
%\newcommand<>\highlightcolor[1]{{\color#2[rgb]{0,0,1}#1}}
\newcommand{\highlight}[1]{{\bf\steel{#1}}}
%\newcommand{\newblock}[0]{}

%\newcommand{\arrow}[0]{\includegraphics[height=5pt]{./figures/arrow}\hspace{3pt}}

\renewcommand{\emph}[1]{\textbf{\steel{{#1}}}}

\renewcommand{\alert}[1]{{\bf\red{{#1}}}}

\newcommand{\arrow}{
\begin{tikzpicture}
\draw [black!40!green, fill=black!40!green] (0,-0.12) -- (0,0.12) --
(0.15,0);
\draw [black!40!green, fill=black!40!green] (0.15,-0.12) -- (0.15,0.12) --
(0.3,0); 
\end{tikzpicture}
}

\geometry{left=0.45cm,top=0cm,right=0.45cm}


\newcommand{\logoimagepath}{./figures/imperial}
\newcommand{\highlightcolor}{blue!80!black}
%\newcommand{\headbarcolor}{imperialBlue}
\newcommand{\headbarcolor}{imperialBlue}
\institute{}

\newcommand{\coursetitle}{}

\newcommand{\slidesetsubtitle}{}
\newcommand{\slidesetnumber}{01}
\usefonttheme{professionalfonts}


\usetikzlibrary{decorations.fractals}
\input{../includes/tikzlibrarybayesnet.code.tex}
\input{../includes/tikzlibraryipe.code.tex}
\usetikzlibrary{matrix,positioning,decorations.pathreplacing}
\usetikzlibrary{calc,quotes,angles}
\usetikzlibrary{arrows, arrows.meta, patterns}

\usetikzlibrary{decorations.pathreplacing}
\tikzset{
    position label/.style={
       above = 3pt,
       text height = 2ex,
       text depth = 1ex
    }
}

% \usetikzlibrary{decorations.markings}
\tikzset{
  font={\fontsize{14pt}{12}\selectfont}
}



\useoutertheme[subsection=false,shadow]{miniframes}
\useinnertheme{default}
\usefonttheme{serif}
%\usepackage{palatino}
\usepackage{mathpazo}
%\usepackage{utopia}
\usepackage{stmaryrd} % for varodot, bigodot 
\usepackage{mathabx} % for \coAsterisk
%\usepackage{mnsymbol}
%\setbeamertemplate{itemize item}{\scriptsize\raise1.7pt\hbox{\donotcoloroutermaths$\Asterisk$}}
%\setbeamertemplate{itemize item}{\scriptsize\raise1.7pt\hbox{\donotcoloroutermaths$\varodot$}}
%\setbeamertemplate{itemize subitem}{\scriptsize\raise1.25pt\hbox{\donotcoloroutermaths$\rhd$}}

\usepackage{xifthen}% provides \isempty tesst

\setbeamerfont{title like}{shape=\scshape}
\setbeamerfont{frametitle}{}



\setbeamercolor*{lower separation line head}{bg=blue} 
\setbeamercolor*{normal text}{fg=black,bg=white} 
\setbeamercolor*{alerted text}{fg=red} 
\setbeamercolor*{example text}{fg=black} 
%\setbeamercolor*{frametitle}{fg=aimsbrown} 
\setbeamercolor*{frametitle}{fg=imperialBlue} 
\setbeamercolor*{structure}{fg=black} 
 
\setbeamercolor*{palette tertiary}{fg=black,bg=black!10} 
\setbeamercolor*{palette quaternary}{fg=black,bg=black!10} 

%\renewcommand{\(}{\begin{columns}}
%\renewcommand{\)}{\end{columns}}
%\newcommand{\<}[1]{\begin{column}{#1}}
%\renewcommand{\>}{\end{column}}

% ======================================
% custom commands 
\newcommand{\cemph}[1]{\textcolor{\highlightcolor}{#1}}
\newcommand{\calert}[1]{\textcolor{red}{#1}}

\setbeamertemplate{navigation symbols}{}
%\renewcommand\frametitle[1]{{\textsc{\Large \textcolor{\highlightcolor}{#1}}}\vspace{0.6cm}\par}

\setbeamertemplate{frametitle}
{
{\textsc\bf \insertframetitle}\vspace{0.2cm}\par
}


%%%%%%%%%%%%%%%%%%%%%%%%%%%%%%%%%%%%%%%%%%%%%%%%%%
\setbeamertemplate{headline}{% 
	\setbeamercolor{head1}{bg=\headbarcolor}
	 \hbox{%
  \begin{beamercolorbox}[wd=.01\paperwidth,ht=2.25ex,dp=50ex,center]{head1}%
  \fontsize{5}{5}\selectfont  
  \end{beamercolorbox}%
  }
  \vspace{-50ex}
}
\setbeamertemplate{footline}{
\begin{tiny}
\setbeamercolor{foot1}{fg=black,bg=gray!10}
\setbeamercolor{foot2}{fg=gray,bg=gray!15}
\setbeamercolor{foot3}{fg=gray,bg=gray!10}
\setbeamercolor{foot4}{fg=black,bg=gray!20}
\setbeamercolor{foot5}{fg=gray,bg=gray!15}
\setbeamercolor{foot6}{fg=black,bg=gray!20}

% taken from theme infolines and adapted
  \leavevmode%
  \hbox{%
  \begin{beamercolorbox}[wd=.45\paperwidth,ht=2.25ex,dp=1ex,center]{foot1}%
  \fontsize{5}{5}\selectfont
  \flushleft \hspace*{2ex}{\footertitle}
  \end{beamercolorbox}%
  % \begin{beamercolorbox}[wd=.08\paperwidth,ht=2.25ex,dp=1ex,center]{foot2}
  % \end{beamercolorbox}%
  %   \begin{beamercolorbox}[wd=.05\paperwidth,ht=2.25ex,dp=1ex,center]{foot3}
  % \end{beamercolorbox}%
    \begin{beamercolorbox}[wd=.45\paperwidth,ht=2.25ex,dp=1ex,center]{foot4}%
  \fontsize{5}{5}\selectfont
  \authorname\hspace{5mm}@\location, \talkDate%\ (\authorweb) 
  \end{beamercolorbox}%
  % \begin{beamercolorbox}[wd=.05\paperwidth,ht=2.25ex,dp=1ex,center]{foot5}
  % \end{beamercolorbox}%
  \begin{beamercolorbox}[wd=.1\paperwidth,ht=2.25ex,dp=1ex,right]{foot6}%
	\insertframenumber{}  \hspace*{2ex} 
  \end{beamercolorbox}}%
  \vskip0pt%
\end{tiny}
\vskip0pt
}


\setbeamertemplate{blocks}[rounded][shadow=false]


\newenvironment<>{myblock}[1]{%
  \begin{actionenv}#2%
      \def\insertblocktitle{#1}%
      \par%
      \mode<presentation>{%
%       \setbeamercolor{block title}{fg=black,bg=aimslightbrown!50!white}
      \setbeamercolor{block title}{fg=black,bg=imperialBlue!45!white}
       \setbeamercolor{block body}{fg=black,bg=gray!20}
       \setbeamercolor{itemize item}{fg=blue!40!white}
       \setbeamertemplate{itemize item}[triangle]
     }%
      \usebeamertemplate{block begin}}
    {\par\usebeamertemplate{block end}\end{actionenv}}

\newenvironment<>{myblock2}[1]{%
  \begin{actionenv}#2%
      \def\insertblocktitle{#1}%
      \par%
      \mode<presentation>{%
       \setbeamercolor{block title}{fg=white,bg=blue!80!black}
       \setbeamercolor{block body}{fg=black,bg=gray!20}
       \setbeamercolor{itemize item}{fg=green!60!black}
       \setbeamertemplate{itemize item}[triangle]
     }%
      \usebeamertemplate{block begin}}
    {\par\usebeamertemplate{block end}\end{actionenv}}

\gdef\colchar#1#2{%
  \tikz[baseline]{%
%  \node[anchor=base,inner sep=2pt,outer sep=0pt,fill = #2!20]
%  {\large{#1}};
  \node[anchor=base,inner sep=1pt,outer sep=0pt,fill = #2!20]
  {{\fontsize{11}{13}\selectfont #1}};
    }%
}%
\gdef\drawfontframe#1#2{%
  \tikz[baseline]{%
  \node[anchor=base,inner sep=2pt,outer sep=0pt,fill = #2!20] {#1};
    }%
  }%


\makeatletter
\let\@@magyar@captionfix\relax
\makeatother

%%% Local Variables:
%%% mode: latex
%%% TeX-master: "2018-09-arusha-linear-regression"
%%% End:





\newif\iflattersubsect

\AtBeginSection[] {
    \begin{frame}<beamer>
    \frametitle{Overview} %
    \tableofcontents[currentsection]  
    \end{frame}
    \lattersubsectfalse
}

\AtBeginSubsection[] {
    \iflattersubsect
    \begin{frame}<Coming Next>
    \frametitle{Overview} %
    \tableofcontents[currentsubsection]  
    \end{frame}
    \fi
    \lattersubsecttrue
}

\begin{document}


%%%%%%%%%%%%%%%%%%%%%%%%%%%%%%%%%%%%%%%%%%%%%%%%%%%%%%

{\setbeamertemplate{footline}{}
\begin{frame}
\title{\slidesettitle}
%\subtitle{SUBTITLE}
\author{\footnotesize
  \textbf{\authorname}
 }

 %%% LOGO

% \begin{flushright}
%   % \begin{columns}
%   %   \column{0.5\hsize}
%   %   \column{0.45\hsize}
%\includegraphics[height = 8mm]{./figures/qla}\hspace{2mm}
%     \includegraphics[height = 8mm]{./figures/aims-rwanda}\\[2mm]
%\includegraphics[height = 8mm]{./figures/imperial}
%%\end{columns}
%\end{flushright}

\vspace{-0cm}
%\begin{flushleft}
%\vspace{-1.5cm}{\small \textcolor{blue}{\coursetitle}}\\\vspace{2cm}
{\huge \slidesettitle \ifthenelse{\equal{\slidesetsubtitle}{}}%
    {}% if #1 is empty
    {: \\ {\large \slidesetsubtitle}}% if #1 is not empty
    } \\    
    %\vspace{20pt}
%\end{flushleft}
  
 
% this is all stuff below the talk title. make two columns, just in
% case you want to have a picture or a second affiliation here 
\begin{columns}[t]
\column{0.8\hsize}
%\begin{flushleft}
\begin{columns}[t]
\column{0.6\hsize}
\insertauthor \\[2mm]
\authoraffiliation\\[2mm]
\column{0.25\hsize}
\\[2mm]
\includegraphics[height = 0.3cm]{./figures-general/twitter}{\small @\authortwitter}\\[-1mm]
\mbox{\small \url{\authoremail}}
\end{columns}
\column{0.14\hsize}
\end{columns}
% \authorweb\\
\vspace{7mm}
% \aimslightbrown{The Nelson Mandela African Institute of Science and
%   Technology\\Arusha, Tanzania}\\[2mm]
\insertdate
%\end{flushleft}
\end{frame}
}

%%% Local Variables:
%%% mode: latex
%%% TeX-master: t
%%% End:

\linespread{1.2}




\begin{frame}[t]{Recap}
Last lecture: Careful mathematical reasoning to \emph{prove} that \pause
\begin{itemize}
\item Loss at deployment converged to \emph{expected loss} as $N\to\infty$ \pause
\item Test set loss converged to \emph{expected loss} as $N\to\infty$ \pause
\item Variance of test set loss scaled as $\frac{c}{N}$\pause
\end{itemize}

\vspace{0.3cm}
Cornerstone of the argument was a \emph{theorem}: Weak LLN:
\begin{align}
\mathbb{P}(|X_n - \mu| < \epsilon) = 1 && \text{for } X_n = \frac{1}{n}\sum_{i=1}^n X_i \,, \{X_n\} \mathrm{iid}\,, \mu = \Exp{}{X_n}
\end{align}
\pause
\vspace{-0.3cm}
\begin{itemize}
\item Doesn't say anything about the \emph{accuracy} for finite $N$! \pause
\item Intuitively, low variance $\implies$ unlikely to be far from mean. \pause
\item Can we use this? Can we make this \emph{precise}?
\end{itemize}
\end{frame}


\begin{frame}{Concentration Inequalities}
\begin{itemize}
\item Theorems are useful because they are \emph{black boxes} \pause
\item Abstract away details of a complex argument, \\ to give you simple answers \pause
\item Today: We break open the black box of the LLN (i.e.~the proof) \pause
\item We find tools that will help us answer questions about finite $N$! \pause
\end{itemize}

\vspace{0.3cm}

Questions:
\begin{enumerate}
\item How accurate is our estimate of the expected loss?
\item How big should our test set be, to get a certain accuracy?
\end{enumerate}
\end{frame}


\section{Intro: Proof of Weak Law of Large Numbers}



\begin{frame}{Weak Law of Large Numbers} 
\begin{itemize}
\item For a sequence of iid RVs $X_1, X_2, X_3, \dots, X_N$ 
\item with mean $\mu = \Exp{}{X}$ 
\item we can define a new RV $\overline{X}_N\ = \frac{1}{N}\sum_{n=1}^N X_n$ 
\item for which will hold:
\begin{align}
\lim_{N\to\infty}\mathbb P\left(|\overline X_n - \mu| < \epsilon\right) = 1
\end{align}
\end{itemize} \pause

\vspace{0.3cm}

\begin{itemize}
\item How to prove this? \pause
\item Let's understand how far samples lie from the mean. \pause
\item For positive RVs, since $|\overline X_n - \mu| \geq 0$!
\end{itemize}
\end{frame}



\begin{frame}[t]{Markov's inequality}
\vspace{-0.6cm}
\begin{myblock}{}
For a RV $X > 0$, and $a > 0$, then
\begin{align}
  P(X \geq a) \leq \frac{\Exp{}{X}}{a}
\end{align}
\end{myblock}\pause

Proof: \pause
\vspace{-0.5cm}
\begin{align}
\Exp{}{X} &= \int_0^\infty x p_X(x) \calcd x \\
\uncover<4->{&= \int_0^a x p_X(x) \calcd x + \int_a^\infty x p_X(x) \calcd x} \\
\uncover<5->{&\geq \int_a^\infty x p_X(x) \calcd x} \\
\uncover<6->{&\geq \int_a^\infty a p_X(x) \calcd x} \\
\uncover<7->{&= a P(X \geq a)} \\
\uncover<8->{\implies& P(X \geq a) \leq \frac{\Exp{}{X}}{a} && \text{Done.}}
\end{align}
\end{frame}


\begin{frame}{Markov's inequality}
For positive RVs (like deviations) with finite means: \pause
\begin{itemize}
\item Large values are increasingly unlikely! ($\propto \frac{1}{a}$) \pause
\item The expectation determines how large values can be \pause
\end{itemize}

\vspace{0.3cm}

Such bounds are powerful because they abstract away details of the distribution, which we may not know! \pause
\end{frame}


\begin{frame}[t]{Chebyshev's Inequality}
\vspace{-0.6cm}
\begin{myblock}{}
For a RV $X$, with finite $\exp{}{X} = \mu$, and finite $\Var{}{X} = \sigma^2$, then for $k>0$
\begin{align}
  P(|X-\mu| \geq k\sigma) \leq \frac{1}{k^2}
\end{align}
\end{myblock}\pause

Proof: Apply Markov's inequality to the RV of the squared deviation: \pause
%\vspace{-0.5cm}
\begin{align}
P\left((X-\mu)^2\geq a\right) &\leq \frac{\Exp{}{(X-\mu)^2}}{a} \\ &= \frac{\sigma^2}{a} \\
\uncover<4->{\implies P\left((X-\mu)^2\geq k^2\sigma^2 \right) &\leq \frac{1}{k^2} && \text{sub }a = k^2\sigma^2} \\
\uncover<5->{\implies P\left(|X-\mu| \geq k\sigma\right) &\leq \frac{1}{k^2} && \text{Done.}}
\end{align}
\end{frame}

\begin{frame}{Chebyshev's Inequality}
For \emph{any} RV with finite mean and variance, we \emph{limit} the probability of being $k$ standard deviations from the mean.
\end{frame}


\begin{frame}[t]{Weak Law of Large Numbers}
Proof of WLLN:
\begin{itemize}
\item Remember: $\overline{X}_N\ = \frac{1}{N}\sum_{n=1}^N X_n$ \pause
\item Note that: $\Var{}{\overline X_n} = \frac{\Var{}{X}}{N} = \frac{c}{N}$ (we assume finite variance) \pause
\item By Chebyshev:
\begin{align}
P(|\overline X_n - \Exp{}{X}| > \epsilon) &\leq \frac{\sigma^2}{\epsilon^2} \\
&= \frac{c}{N\epsilon^2}
\end{align} \pause
\item For \emph{any} fixed $\epsilon$, $\lim_{N\to\infty} \frac{c}{N\epsilon^2} = 0$ \pause
\item $\implies \lim_{N\to\infty}\mathbb P\left(|\overline X_n - \mu| < \epsilon\right) = 1\qquad\qquad$ Done.
\end{itemize}
\end{frame}



\section{Generalisation Error Bounds}
\begin{frame}{LLN is a Detour}
\begin{itemize}
\item LLN ignores the size of the variance \pause
\item To prove LLN, we used a bound that \emph{did} depend on the size of the variance! \pause
\end{itemize}

\vspace{0.3cm}

\begin{center}
Can we use knowledge of the size of the variance \\ to say something more about generalisation error?
\end{center}
\end{frame}




\begin{frame}{Generalisation Error Bound}
A \emph{Generalisation Error/Loss Bound} is a procedure for computing a number $\epsilon$ from data that you sample form the world, such that \pause
\begin{itemize}
\item with high probability, \pause
\item the expected loss is below $\epsilon$. \pause
\end{itemize}
\begin{align}
\mathbb P\left(|L_{\text{test}} - \mathrm{ER}| > \epsilon\right) < \delta \\
\mathrm{ER} = \Exp{\pi(x, y)}{\ell(f(x; \vtheta^*), y)}
\end{align}
\end{frame}


\begin{frame}{Classification GEB}
\begin{itemize}
\item Consider Classification where $f : \mathcal X \to [0, 1]$. \pause
\item For \emph{testing}, we use 0-1 loss function (classification accuracy)
\begin{align}
\ell(f(x; \vtheta^*), y) = \begin{cases}
0 \qquad \text{if } \text{\texttt{int}}(f(x; \vtheta^*)) = y \\
1 \qquad \text{otherwise}
\end{cases}
\end{align} \pause
\item Remember $L_{\text{test}} = \frac{1}{N} \sum_{n=1}^N \ell(f(x; \vtheta^*), y)$
\item Remember $\Exp{\pi(x, y)}{L_{test}} = \mathrm{ER}$ \\ ($x = [x_1, x_2, \dots]$, and $y = [y_1, y_2, \dots]$).
\end{itemize}
\end{frame}

\begin{frame}{Chebyshev GEB}
Apply Chebyshev: \pause
\begin{align}
\mathbb P(|L_{\text{test}} - \mathrm{ER}| > \epsilon) &< \frac{\sigma^2}{\epsilon^2} \\
\uncover<3->{\sigma^2 &= \Var{\pi(x, y)}{L_{\text{test}}} \\
 &= \frac{1}{N}\Var{\pi(x, y)}{\ell(f(x; \vtheta^*), y)}}
\end{align} \pause \pause
Notice: $\Var{\pi(x, y)}{\ell(f(x; \vtheta^*), y)} < 0.25$! \pause
\begin{align}
\mathbb P(|L_{\text{test}} - \mathrm{ER}| > \epsilon) < \frac{0.25}{N\epsilon^2}
\end{align} \pause
\vspace{-0.8cm}
\begin{align}
\implies \mathbb P(\mathrm{ER} > L_{\text{test}} + \epsilon) < \frac{0.25}{N\epsilon^2}
\end{align}
{\tiny(Draw double-sided plot on board. $L_{\text{test}}$ is RV, and we only care about under-estimation of ER.)}
\end{frame}


\begin{frame}{Example Chebyshev GEB}
Q1: How accurate is our estimate of the expected loss?
\begin{itemize}
\item You train a NN on MNIST \pause 
\item Test error with $N=10000$ gives $L_{\text{test}} = 0.01$ \pause
\item Then Chebyshev gives us the guarantee that \pause
\begin{align}
\mathbb P(\mathrm{ER} > L_{\text{test}} + 0.03) < \frac{0.25}{N\cdot 0.03^2} = 0.0278 && \text{Pretty confident} \\
\mathbb P(\mathrm{ER} > L_{\text{test}} + 0.01) < \frac{0.25}{N\cdot 0.01^2} = 0.25 && \text{Not confident} \\
\mathbb P(\mathrm{ER} > L_{\text{test}} + 0.001) < \frac{0.25}{N\cdot 0.001^2} = 25 && \text{\emph{Vacuous}}
\end{align}
\end{itemize}
\end{frame}


\begin{frame}{How good is this?}
\begin{itemize}
\item We can guarantee with high probability that the classifier isn't an order of magnitude worse than $L_{\text{test}}$ indicates \pause
\item However bound is not tight enough to distinguish different methods, which often differ in accuracy by ±0.001 \pause
\item Probably \emph{very} pessimistic \pause
\item Bound holds for \emph{any} distribution with a maximum variance!
\end{itemize}
\end{frame}


\begin{frame}{Flipping bound round}
Q2: How big should our test set be, to get a certain accuracy?
\begin{gather}
\mathbb P(\mathrm{ER} > L_{\text{test}} + \epsilon) < \delta \\
\implies N > \frac{0.25}{\delta\epsilon^2}
\end{gather} \pause

\begin{itemize}
\item For $\epsilon = 0.001$, and $\delta = \frac{0.25}{N\epsilon^2} < 0.05$, we need $N > 5\cdot 10^6$
\item For $\epsilon = 0.001$, and $\delta = \frac{0.25}{N\epsilon^2} < 0.01$, we need $N > 25\cdot 10^6$
\item For $\epsilon = 0.01$, and $\delta = \frac{0.25}{N\epsilon^2} < 0.05$, we need $N > 50\cdot 10^3$
\item For $\epsilon = 0.01$, and $\delta = \frac{0.25}{N\epsilon^2} < 0.01$, we need $N > 250\cdot 10^3$
\end{itemize}
\end{frame}


\section{Assumptions and Quality of Bounds}

\begin{frame}{Hoeffding's inequality}
\begin{myblock}{}
For iid RVs $X_1, X_2, \dots$, such that $a < X_n < b$, $S_N = \frac{1}{N}\sum_n X_n$, and $t > 0$, we have
\begin{align}
\mathbb P\left(|S_N - \Exp{\pi}{S_N}| \geq t\right) \leq 2\exp\left(-\frac{2t^2N}{(b-a)^2}\right)
\end{align}
\end{myblock}

Proof not covered in course :)
\end{frame}



\begin{frame}{Hoeffding GEB}
Again, for classification
\begin{gather}
\mathbb P\left(\mathrm{ER} > L_{\text{test}} + \epsilon\right) \leq \delta \\
\implies N \geq \frac{\log(2\delta^{-1})}{2\epsilon^2}
%\delta = 2\exp\left(-\frac{2t^2N}{(b-a)^2}\right)
\end{gather} \pause
\begin{itemize}
\item For $\epsilon = 0.001$, and $\delta = \frac{0.25}{N\epsilon^2} < 0.05$, we need $N > 1.85\cdot 10^6$
\item For $\epsilon = 0.001$, and $\delta = \frac{0.25}{N\epsilon^2} < 0.01$, we need $N > 2.65\cdot 10^6$
\item For $\epsilon = 0.01$, and $\delta = \frac{0.25}{N\epsilon^2} < 0.05$, we need $N > 18.5\cdot 10^3$
\item For $\epsilon = 0.01$, and $\delta = \frac{0.25}{N\epsilon^2} < 0.01$, we need $N > 26.5\cdot 10^3$
\end{itemize} \pause
\begin{center}
Significant reduction compared to Chebyshev!
\end{center}
\end{frame}


\section{Conclusion}
\begin{frame}{Conclusion}
\begin{itemize}
\item Applying concentration inequalities (skill)
\item Can tell us accuracy of test set estimates
\item Concentration inequalities all relied on unbiased estimates
\item Variance determined accuracy
\end{itemize}
\end{frame}










\end{document}
%%% Local Variables: 
%%% mode: latex
%%% TeX-master: t
%%% End: 
