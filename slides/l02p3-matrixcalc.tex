%% Time-stamp: <2018-10-18 20:24:12 (marc)>
\documentclass[xcolor=x11names,compress,mathserif]{beamer}

\newcommand{\hackspace}{\hspace{4.2mm}}
\newcommand{\showstudent}[1]{}
\newcommand\hmmax{0}
\newcommand\bmmax{0}


\usepackage{../includes/MarkMathCmds}





% talk/author information
\newcommand{\authorname}{Mark van der Wilk}
\newcommand{\authoremail}{m.vdwilk@imperial.ac.uk}
\newcommand{\authoraffiliation}{
  Department of Computing\\Imperial
  College London}
\newcommand{\authortwitter}{markvanderwilk}
\newcommand{\slidesettitle}{\imperialBlue{Matrix \& Array Derivatives}}
\newcommand{\footertitle}{Differentiation}
\newcommand{\location}{Imperial College London}
\newcommand{\talkDate}{October 12, 2021}



\date{\imperialGray{\talkDate}}




% load defaults
\selectcolormodel{rgb}
\usepackage{ifxetex,ifluatex}
\newif\ifxetexorluatex
\ifxetex
  \xetexorluatextrue
\else
  \ifluatex
    \xetexorluatextrue
  \else
    \xetexorluatexfalse
  \fi
\fi

\usepackage{textpos}
%\usepackage{arabtex}
\usepackage{tikz}
\usetikzlibrary{decorations.markings}
\usetikzlibrary{arrows}
\usetikzlibrary{shapes}
\usetikzlibrary{plotmarks}
\usetikzlibrary{mindmap,trees,backgrounds}

\tikzstyle{every picture}+=[remember picture]

%\usepackage{movie15}
% \usepackage{pdfpages}
%\usepackage{xmpmulti}

\usepackage{anyfontsize}
\usepackage{wrapfig}
\usepackage{animate}
\usepackage{multirow}
\usepackage{multimedia}
\usepackage{xmpmulti}
%\usepackage[latin9]{inputenc}
\usepackage[english]{babel}
\usepackage{scalefnt}
\usepackage{verbatim}
\usepackage{url}
% \usepackage{pgf,pgfarrows,pgfnodes}
\usepackage{textpos}
\usepackage[tight,ugly]{units}
\usepackage{url}
\usepackage{bbm}
\usepackage[english]{babel}
\usepackage{fancyhdr}
\usepackage{bm} % correct bold symbols, like \bm
\usepackage{amsmath}
\usepackage{amsfonts}
\usepackage{amssymb}
\usepackage{mathrsfs}
\usepackage{mathtools}
\usepackage{color}
\usepackage{cancel}
\usepackage{algorithm}
\usepackage{algpseudocode}
\usepackage{mathrsfs}
\usepackage{listings}
\usepackage{graphicx} % for pdf, bitmapped graphics files
\usepackage{mathtools}
\usepackage{units}
\usepackage{subfig}
\usepackage{enumerate}
\usepackage{natbib}
\usepackage{dsfont}


\ifxetexorluatex
\usepackage{fontspec}
\setmainfont[Scale=0.8]{OpenDyslexic-Regular}
\else
\usefonttheme{professionalfonts}
\fi

\renewcommand{\vec}[1]{{\boldsymbol{{#1}}}} % vector
\newcommand{\mat}[1]{{\boldsymbol{{#1}}}} % matrix
% \newcommand{\KL}[2]{\mathrm{KL}(#1\|#2)} % KL divergence
\newcommand{\R}[0]{\mathds{R}} % real numbers
\newcommand{\Z}[0]{\mathds{Z}} % integers
\newcommand{\tr}[0]{\text{tr}} % trace
% \newcommand{\inv}{^{-1}}
% \DeclareMathOperator*{\diag}{diag}
\newcommand{\E}{\mathds{E}} % expectation
\newcommand{\var}{\mathds{V}}
\newcommand{\gauss}[2]{\mathcal{N}\big(#1,\,#2\big)}
\newcommand{\gaussx}[3]{\mathcal{N}\big(#1\,|\,#2,\,#3\big)}
\newcommand{\gaussBig}[2]{\mathcal{N}\left(#1,\,#2\right)}
\newcommand{\gaussxBig}[3]{\mathcal{N}\left(#1\,\left|\,#2,\,#3\right.\right)}
\newcommand{\Ber}[0]{\mathrm{Ber}} % Bernoulli distribution
\DeclareMathOperator{\cov}{Cov}
\ifxetexorluatex
\renewcommand{\T}[0]{^\top}
\renewcommand{\d}[0]{\text{d}} % derivative
\else
\newcommand{\T}[0]{^\top}
\renewcommand{\d}[0]{\text{d}} % derivative
\fi
% calculus
\newcommand{\pdiff}[1]{\frac{\partial}{\partial #1}}
\newcommand{\pdiffF}[2]{\frac{\partial #1}{\partial #2}}
\newcommand{\diffF}[2]{\frac{{\d}#1}{{\d}#2}}
\newcommand{\diffFII}[2]{\frac{{\d}^2 #1}{{\d}#2^2}}
\newcommand{\diff}[1]{\frac{{\d}}{{\d}#1}}
\newcommand{\diffII}[1]{\frac{{\d}^2}{{\d}#1^2}}
\newcommand{\class}[0]{\mathcal{C}}

\newcommand{\idx}[1]{{(#1)}}
% \newcommand{\norm}[1]{\left\|#1\right\|}
\newcommand{\proj}[1]{\tilde{#1}}
\newcommand{\pcacoord}{z}
\newcommand{\pcacoordnew}{\zeta}
\newcommand{\latent}{z}
% \newcommand{\given}{\,|\,}
\newcommand{\genset}[1]{\mathrm{span}[#1]} % generating set
\newcommand{\set}[1]{\mathcal{#1}} % set
\newcommand{\fixgmfont}[1]{\scalebox{0.8}{#1}}



\usepackage{pifont}% http://ctan.org/pkg/pifont
\newcommand{\cmark}{{\color{green!40!black}\ding{51}}}%
\newcommand{\xmark}{{\color{red}\ding{55}}}%
\newcommand{\green}[1]{{\bf{\textcolor{green}{#1}}}}
\newcommand{\red}[1]{{\bf{\textcolor{red}{#1}}}}

\newcommand<>\red[1]{{\color#2[rgb]{1,0,0}#1}}
\newcommand<>\blue[1]{{\color#2[rgb]{0,0,1}#1}}
\newcommand<>\yellow[1]{{\color#2{camyellow}#1}}
\newcommand<>\green[1]{{\color#2[rgb]{0,0.6,0.0}#1}}
\newcommand<>\violet[1]{{\color#2[rgb]{0.6,0,0.6}#1}}
\newcommand<>\orange[1]{{\color#2[rgb]{1,0.5,0}#1}}
\newcommand<>\black[1]{{\color#2[rgb]{0,0,0}#1}}
\newcommand<>\steel[1]{{\color#2[rgb]{0,0,0.8}#1}}
\newcommand<>\darkblue[1]{{\color#2[rgb]{0,0,0.6}#1}}
\newcommand<>\lightblue[1]{{\color#2[rgb]{0.4,0.4,0.7}#1}}
\newcommand<>\gray[1]{{\color#2[rgb]{0.4,0.4,0.4}#1}}
\newcommand<>\greenish[1]{{\color#2[rgb]{0.45, 0.66, 0.45}#1}}
\newcommand<>\redish[1]{{\color#2[rgb]{0.7843    0.3706    0.3706}#1}}
\definecolor{redishTIKZ}{rgb}{0.7843, 0.3706, 0.3706}
\definecolor{imperialBlue}{rgb}{0.058, 0.219, 0.418}
\definecolor{aimsbrown}{rgb}{0.539, 0.117, 0.015}
% \definecolor{imperialGray}{rgb}{0.414, 0.488, 0.671 }
\definecolor{imperialGray}{RGB}{109,153, 204}
\definecolor{aimslightbrown}{RGB}{138,88,84}
\newcommand<>\imperialBlue[1]{{\color#2[rgb]{0.058, 0.219, 0.418}#1}}
\newcommand<>\aimsbrown[1]{{\color#2[rgb]{0.539, 0.117, 0.015}#1}}
%\newcommand<>\imperialGray[1]{{\color#2[rgb]{0.414, 0.488, 0.671}#1}}
\newcommand<>\imperialGray[1]{{\color#2[RGB]{109,153, 204}#1}}
\newcommand<>\aimslightbrown[1]{{\color#2[RGB]{138,88,84}#1}}
\newcommand<>\lightgray[1]{{\color#2[rgb]{0.8,0.8,0.8}#1}}
%\newcommand<>\highlightcolor[1]{{\color#2[rgb]{0,0,1}#1}}
\newcommand{\highlight}[1]{{\bf\steel{#1}}}
%\newcommand{\newblock}[0]{}

%\newcommand{\arrow}[0]{\includegraphics[height=5pt]{./figures/arrow}\hspace{3pt}}

\renewcommand{\emph}[1]{\textbf{\steel{{#1}}}}

\renewcommand{\alert}[1]{{\bf\red{{#1}}}}

\newcommand{\arrow}{
\begin{tikzpicture}
\draw [black!40!green, fill=black!40!green] (0,-0.12) -- (0,0.12) --
(0.15,0);
\draw [black!40!green, fill=black!40!green] (0.15,-0.12) -- (0.15,0.12) --
(0.3,0); 
\end{tikzpicture}
}

\geometry{left=0.45cm,top=0cm,right=0.45cm}


\newcommand{\logoimagepath}{./figures/imperial}
\newcommand{\highlightcolor}{blue!80!black}
%\newcommand{\headbarcolor}{imperialBlue}
\newcommand{\headbarcolor}{imperialBlue}
\institute{}

\newcommand{\coursetitle}{}

\newcommand{\slidesetsubtitle}{}
\newcommand{\slidesetnumber}{01}
\usefonttheme{professionalfonts}


\usetikzlibrary{decorations.fractals}
\input{../includes/tikzlibrarybayesnet.code.tex}
\input{../includes/tikzlibraryipe.code.tex}
\usetikzlibrary{matrix,positioning,decorations.pathreplacing}
\usetikzlibrary{calc,quotes,angles}
\usetikzlibrary{arrows, arrows.meta, patterns}

\usetikzlibrary{decorations.pathreplacing}
\tikzset{
    position label/.style={
       above = 3pt,
       text height = 2ex,
       text depth = 1ex
    }
}

% \usetikzlibrary{decorations.markings}
\tikzset{
  font={\fontsize{14pt}{12}\selectfont}
}



\useoutertheme[subsection=false,shadow]{miniframes}
\useinnertheme{default}
\usefonttheme{serif}
%\usepackage{palatino}
\usepackage{mathpazo}
%\usepackage{utopia}
\usepackage{stmaryrd} % for varodot, bigodot 
\usepackage{mathabx} % for \coAsterisk
%\usepackage{mnsymbol}
%\setbeamertemplate{itemize item}{\scriptsize\raise1.7pt\hbox{\donotcoloroutermaths$\Asterisk$}}
%\setbeamertemplate{itemize item}{\scriptsize\raise1.7pt\hbox{\donotcoloroutermaths$\varodot$}}
%\setbeamertemplate{itemize subitem}{\scriptsize\raise1.25pt\hbox{\donotcoloroutermaths$\rhd$}}

\usepackage{xifthen}% provides \isempty tesst

\setbeamerfont{title like}{shape=\scshape}
\setbeamerfont{frametitle}{}



\setbeamercolor*{lower separation line head}{bg=blue} 
\setbeamercolor*{normal text}{fg=black,bg=white} 
\setbeamercolor*{alerted text}{fg=red} 
\setbeamercolor*{example text}{fg=black} 
%\setbeamercolor*{frametitle}{fg=aimsbrown} 
\setbeamercolor*{frametitle}{fg=imperialBlue} 
\setbeamercolor*{structure}{fg=black} 
 
\setbeamercolor*{palette tertiary}{fg=black,bg=black!10} 
\setbeamercolor*{palette quaternary}{fg=black,bg=black!10} 

%\renewcommand{\(}{\begin{columns}}
%\renewcommand{\)}{\end{columns}}
%\newcommand{\<}[1]{\begin{column}{#1}}
%\renewcommand{\>}{\end{column}}

% ======================================
% custom commands 
\newcommand{\cemph}[1]{\textcolor{\highlightcolor}{#1}}
\newcommand{\calert}[1]{\textcolor{red}{#1}}

\setbeamertemplate{navigation symbols}{}
%\renewcommand\frametitle[1]{{\textsc{\Large \textcolor{\highlightcolor}{#1}}}\vspace{0.6cm}\par}

\setbeamertemplate{frametitle}
{
{\textsc\bf \insertframetitle}\vspace{0.2cm}\par
}


%%%%%%%%%%%%%%%%%%%%%%%%%%%%%%%%%%%%%%%%%%%%%%%%%%
\setbeamertemplate{headline}{% 
	\setbeamercolor{head1}{bg=\headbarcolor}
	 \hbox{%
  \begin{beamercolorbox}[wd=.01\paperwidth,ht=2.25ex,dp=50ex,center]{head1}%
  \fontsize{5}{5}\selectfont  
  \end{beamercolorbox}%
  }
  \vspace{-50ex}
}
\setbeamertemplate{footline}{
\begin{tiny}
\setbeamercolor{foot1}{fg=black,bg=gray!10}
\setbeamercolor{foot2}{fg=gray,bg=gray!15}
\setbeamercolor{foot3}{fg=gray,bg=gray!10}
\setbeamercolor{foot4}{fg=black,bg=gray!20}
\setbeamercolor{foot5}{fg=gray,bg=gray!15}
\setbeamercolor{foot6}{fg=black,bg=gray!20}

% taken from theme infolines and adapted
  \leavevmode%
  \hbox{%
  \begin{beamercolorbox}[wd=.45\paperwidth,ht=2.25ex,dp=1ex,center]{foot1}%
  \fontsize{5}{5}\selectfont
  \flushleft \hspace*{2ex}{\footertitle}
  \end{beamercolorbox}%
  % \begin{beamercolorbox}[wd=.08\paperwidth,ht=2.25ex,dp=1ex,center]{foot2}
  % \end{beamercolorbox}%
  %   \begin{beamercolorbox}[wd=.05\paperwidth,ht=2.25ex,dp=1ex,center]{foot3}
  % \end{beamercolorbox}%
    \begin{beamercolorbox}[wd=.45\paperwidth,ht=2.25ex,dp=1ex,center]{foot4}%
  \fontsize{5}{5}\selectfont
  \authorname\hspace{5mm}@\location, \talkDate%\ (\authorweb) 
  \end{beamercolorbox}%
  % \begin{beamercolorbox}[wd=.05\paperwidth,ht=2.25ex,dp=1ex,center]{foot5}
  % \end{beamercolorbox}%
  \begin{beamercolorbox}[wd=.1\paperwidth,ht=2.25ex,dp=1ex,right]{foot6}%
	\insertframenumber{}  \hspace*{2ex} 
  \end{beamercolorbox}}%
  \vskip0pt%
\end{tiny}
\vskip0pt
}


\setbeamertemplate{blocks}[rounded][shadow=false]


\newenvironment<>{myblock}[1]{%
  \begin{actionenv}#2%
      \def\insertblocktitle{#1}%
      \par%
      \mode<presentation>{%
%       \setbeamercolor{block title}{fg=black,bg=aimslightbrown!50!white}
      \setbeamercolor{block title}{fg=black,bg=imperialBlue!45!white}
       \setbeamercolor{block body}{fg=black,bg=gray!20}
       \setbeamercolor{itemize item}{fg=blue!40!white}
       \setbeamertemplate{itemize item}[triangle]
     }%
      \usebeamertemplate{block begin}}
    {\par\usebeamertemplate{block end}\end{actionenv}}

\newenvironment<>{myblock2}[1]{%
  \begin{actionenv}#2%
      \def\insertblocktitle{#1}%
      \par%
      \mode<presentation>{%
       \setbeamercolor{block title}{fg=white,bg=blue!80!black}
       \setbeamercolor{block body}{fg=black,bg=gray!20}
       \setbeamercolor{itemize item}{fg=green!60!black}
       \setbeamertemplate{itemize item}[triangle]
     }%
      \usebeamertemplate{block begin}}
    {\par\usebeamertemplate{block end}\end{actionenv}}

\gdef\colchar#1#2{%
  \tikz[baseline]{%
%  \node[anchor=base,inner sep=2pt,outer sep=0pt,fill = #2!20]
%  {\large{#1}};
  \node[anchor=base,inner sep=1pt,outer sep=0pt,fill = #2!20]
  {{\fontsize{11}{13}\selectfont #1}};
    }%
}%
\gdef\drawfontframe#1#2{%
  \tikz[baseline]{%
  \node[anchor=base,inner sep=2pt,outer sep=0pt,fill = #2!20] {#1};
    }%
  }%


\makeatletter
\let\@@magyar@captionfix\relax
\makeatother

%%% Local Variables:
%%% mode: latex
%%% TeX-master: "2018-09-arusha-linear-regression"
%%% End:





\newif\iflattersubsect

\AtBeginSection[] {
    \begin{frame}<beamer>
    \frametitle{Overview} %
    \tableofcontents[currentsection]  
    \end{frame}
    \lattersubsectfalse
}

\AtBeginSubsection[] {
    \iflattersubsect
    \begin{frame}<Coming Next>
    \frametitle{Overview} %
    \tableofcontents[currentsubsection]  
    \end{frame}
    \fi
    \lattersubsecttrue
}

\begin{document}


%%%%%%%%%%%%%%%%%%%%%%%%%%%%%%%%%%%%%%%%%%%%%%%%%%%%%%

{\setbeamertemplate{footline}{}
\begin{frame}
\title{\slidesettitle}
%\subtitle{SUBTITLE}
\author{\footnotesize
  \textbf{\authorname}
 }

 %%% LOGO

% \begin{flushright}
%   % \begin{columns}
%   %   \column{0.5\hsize}
%   %   \column{0.45\hsize}
%\includegraphics[height = 8mm]{./figures/qla}\hspace{2mm}
%     \includegraphics[height = 8mm]{./figures/aims-rwanda}\\[2mm]
%\includegraphics[height = 8mm]{./figures/imperial}
%%\end{columns}
%\end{flushright}

\vspace{-0cm}
%\begin{flushleft}
%\vspace{-1.5cm}{\small \textcolor{blue}{\coursetitle}}\\\vspace{2cm}
{\huge \slidesettitle \ifthenelse{\equal{\slidesetsubtitle}{}}%
    {}% if #1 is empty
    {: \\ {\large \slidesetsubtitle}}% if #1 is not empty
    } \\    
    %\vspace{20pt}
%\end{flushleft}
  
 
% this is all stuff below the talk title. make two columns, just in
% case you want to have a picture or a second affiliation here 
\begin{columns}[t]
\column{0.8\hsize}
%\begin{flushleft}
\begin{columns}[t]
\column{0.6\hsize}
\insertauthor \\[2mm]
\authoraffiliation\\[2mm]
\column{0.25\hsize}
\\[2mm]
\includegraphics[height = 0.3cm]{./figures-general/twitter}{\small @\authortwitter}\\[-1mm]
\mbox{\small \url{\authoremail}}
\end{columns}
\column{0.14\hsize}
\end{columns}
% \authorweb\\
\vspace{7mm}
% \aimslightbrown{The Nelson Mandela African Institute of Science and
%   Technology\\Arusha, Tanzania}\\[2mm]
\insertdate
%\end{flushleft}
\end{frame}
}

%%% Local Variables:
%%% mode: latex
%%% TeX-master: t
%%% End:

\linespread{1.2} 

\makeatletter
\newcommand{\Pause}[1][]{\unless\ifmeasuring@\relax
\pause[#1]%
\fi}
\makeatother


\begin{frame}{Motivation: Next level curve fitting}
You have now solved Linear Regression. A key design choice was which \emph{basis functions} to use, e.g.:
\begin{equation}
\vphi(x)\transpose = \begin{bmatrix}x^3 \,\, x^2 \,\, x \,\, 1\end{bmatrix}
\end{equation} \pause

\begin{center}
Instead, can we learn the basis functions? \pause
\end{center}

A \emph{neural network} parameterises functions:
\begin{align}
f_{\ell}(\vx) = \sigma(\mathbf A_\ell \vx + \mathbf b_\ell) \\
f_{NN}(\vx) = f_L(f_{L-1}(\dots f_1(\vx) \dots))
\end{align} \pause
\vspace{-0.4cm}
\begin{itemize}
\item Parameters are $\vtheta = \{\mathbf A_\ell, \mathbf b_\ell\}_{\ell=1}^L$. \pause
\item How do we differentiate w.r.t.~matrices?
\end{itemize}

\end{frame}



\begin{frame}{Derivatives of matrices/arrays}
How should we find derivatives like $\deriv[]{\theta}\vx\transpose\mathbf A(\theta)\vx$ or $\deriv[]{\mathbf A}||\mathbf A\vx - \vy||^2$? \pause \\
\begin{itemize}
\item By the same arguments as before (can look at differences of any array, and directional derivative arguments), we can just collect the gradients of all outputs w.r.t.~all inputs:
\begin{align}
\pderiv[f_{ijkl}]{x_{abc}}
\end{align} \pause
% \item In index notation, the chain rule works in the same way too, e.g.~for $f(\mat G(\vx))$, with $G: \Reals^D \to \Reals^{N\times M}$:
% \begin{align}
% \pderiv[f]{x_d} = \sum_{ij} \pderiv[f]{G_{ij}}\pderiv[G_{ij}]{x_d}
% \end{align}
% \item Exactly the same principle as vector derivatives, but with reshaped inputs/outputs.
\end{itemize}


\begin{center}
Wouldn't it be nice if there was a chain rule?
\end{center}
\begin{align*}
\deriv[f]{\theta} = \deriv[f]{\mathbf A}\deriv[\mathbf A]{\theta} \text{?} && \text{or} && \deriv[f]{\mathbf A} =  \deriv[f]{\vg}\deriv[\vg]{\mathbf A} \text{?}
\end{align*}

\end{frame}


\begin{frame}{Chain rule}
A function of a matrix $\mathbf A \in \Reals^{M\times N}$ is \emph{just a multivariate function}:
\begin{align*}
f(\mathbf A) &= ||\mathbf A\vx - \vy||^2 \\
f(A_{11}, A_{21}, \dots, A_{M1} \dots A_{MN}) &= \sum_{i} (\sum_{ij}A_{ij}x_j - y_i)^2
\end{align*} \pause

We just \emph{arrange} the numbers in a different way. \pause
So the chain rule is the same!
\begin{gather}
f(\vg) = ||\vg||^2\,, \qquad \vg(\mathbf A) = \mathbf A\vx - \vy \\
\pderiv[f]{A_{ij}} = \sum_{k} \pderiv[f]{g_k} \pderiv[g_k]{A_{ij}}
\end{gather} \pause
\begin{gather}
f(\mathbf A) = \vx\transpose\mathbf A\vx \\
\pderiv[f]{\theta} = \sum_{jk} \pderiv[f]{A_{jk}} \pderiv[A_{jk}]{\theta}
\end{gather}
\end{frame}



\begin{frame}{Chain rule is not straightforward}
\begin{align*}
f(\mathbf A) &= \vx\transpose\mathbf A\vx && f(\vg) = ||\vg||^2\,, \, \vg(\mathbf A) = \mathbf A\vx - \vy \\
\pderiv[f]{\theta} &= \sum_{jk} \pderiv[f]{A_{jk}} \pderiv[A_{jk}]{\theta} && \pderiv[f]{A_{ij}} = \sum_{k} \pderiv[f]{g_k} \pderiv[g_k]{A_{ij}}
\end{align*}
Can we find a convenient notation like earlier?
\begin{align}
\deriv[f]{\theta} = \deriv[f]{\mathbf A} \deriv[\mathbf A]{\theta} \text{?} && \deriv[f]{\mat A} = \deriv[f]{\vg} \deriv[\vg]{\mathbf A} \text{?}
\end{align} \pause
\vspace{-0.4cm}
\begin{itemize}
\item NOT matrix multiplication, even though both $\deriv[f]{\mathbf A}$ and $\deriv[\mathbf A]{\theta}$ look like matrices. \pause Check the shapes!
\item Shape of $\deriv[\vg]{\mathbf A}$ isn't even a matrix!
\end{itemize}
\end{frame}




%%%%%%%%%%%%%%%%%%%%%%%%%%%%%%%%%%%%%%%%%
\begin{frame}
  \frametitle{Derivatives with Respect to Matrices}

  \begin{itemize}
  \item Recall: A function $\vec
  f:\R^\text{\colchar{\scriptsize{$N$}}{blue}}\to\R^\text{\colchar{\scriptsize{$M$}}{orange}}$
  has a gradient that is an
  $M\times N$-matrix
  with
    $$
    \diffF{\vec f}{\vec x}\in \R^{M\times N}\,,\qquad \mathrm{d}\vec f[m,n] =
    \frac{\partial f_m}{\partial x_n}
    $$
    \begin{center}
     Gradient dimension: \colchar{ \# target dimensions }{orange} $\times$ \colchar{\# input dimensions}{blue}
    \end{center}
    \pause
  \item This generalizes to when the inputs ($N$) or targets ($M$) are \emph{matrices}
  \pause
\item Function $\vec
  f:\R^\text{\colchar{\scriptsize{$M\times N$}}{blue}}\to
  \R^\text{\colchar{\scriptsize{$P\times Q$}}{orange}}$, has a
  gradient that is a \mbox{\colchar{$(P\times Q)$}{orange} $\times$ \colchar{$(M\times N)$}{blue}} object (array)
  $$
    \diffF{\vec f}{\mat X}\in \R^{(P\times Q) \times (M \times N)}\,,\qquad 
    \mathrm{d}\vec f[p,q,m,n] =\frac{\partial f_{pq}}{\partial X_{mn}}
    $$
  \end{itemize} \pause
\begin{center}
Autodiff packages have similar consistency of shapes.
\end{center}
  
\end{frame}




%%%%%%%%%%%%%%%%%%%%%%%%%%%%%%%%%%%%%%%%%
\begin{frame}
  \frametitle{Example 1: Derivatives with Respect to Matrices}

  \begin{align*}
    \vec f = \mat A\vec x\,,\quad \vec f\in\R^M, \mat A\in\R^{M\times
    N}, \vec x \in\R^N 
  \end{align*}
 {\scriptsize 
		\begin{align*}
        \begin{bmatrix}
          \colchar{$y_1$}{red}\\
          \vdots\\
          \colchar{$y_M$}{blue}
        \end{bmatrix}
        =
        \begin{bmatrix}
          \colchar{$f_1(\vec x)$}{red}\\
          \vdots\\
          \colchar{$f_M(\vec x)$}{blue}
        \end{bmatrix}
        =
        \begin{bmatrix}
          \colchar{$A_{11}x_1$}{red} + \colchar{$A_{12}x_2$}{red} + &
          \cdots & + \colchar{$A_{1N}x_N$}{red}\\
          \vdots \qquad \qquad \vdots & \vdots & \quad  \vdots \\
          \colchar{$A_{M1}x_1$}{blue} + \colchar{$A_{M2}x_2$}{blue} + & 
          \cdots & 
          + \colchar{$A_{MN}x_N$}{blue}
        \end{bmatrix}
      \end{align*}}
      
\begin{align*}
    &\diffF{\vec f}{\mat A} \in\R^{\only<1>{\colchar{$?$}{green}} \visible<2>{\colchar{$ \text{\# target dim} \times \text{\# input dim}$}{green} = \colchar{$M\times (M\times N)$}{green}}}\\
    \visible<2>{&\frac{d\vec f}{d \mat A} =
    \begin{bmatrix}
      \frac{\partial f_1}{\partial \mat A}\\
      \vdots\\
      \frac{\partial f_M}{\partial \mat A}
    \end{bmatrix}\,,\quad \frac{\partial f_i}{\partial \mat A}\in\R^{1\times (M\times N)}}
\end{align*}
  

\end{frame}

%%%%%%%%%%%%%%%%%%%%%%%%%%%%%%%%%%%%%%%%%
\begin{frame}
  \frametitle{Example 2: Derivatives with Respect to Matrices}
  \vspace{-5mm}
  \begin{align*}
    f_i &= \sum_{j=1}^NA_{ij} x_j, \quad i = 1,\dotsc, M
  \end{align*}
  \vspace{-4mm}
  { 
		\begin{align*}
        \begin{bmatrix}
          y_1\\
          \vdots\\
          \colchar{$y_i$}{red}\\
		  \vdots\\
          y_M
        \end{bmatrix}
        =
        \begin{bmatrix}
          f_1(\vec x)\\
          \vdots\\
          \colchar{$f_i(\vec x)$}{red}\\
          \vdots\\
          f_M(\vec x)
        \end{bmatrix}
        =
        \begin{bmatrix}
          A_{11}x_1 + A_{12}x_2 + &
          \cdots & + A_{1N}x_N\\
          \vdots \qquad \qquad \vdots & \vdots & \quad  \vdots \\
          \colchar{$A_{i1}x_1$}{red} + \colchar{$A_{i2}x_2$}{red} & 
          \cdots &  +\colchar{$A_{iN}x_N$}{red}\\
          \vdots \qquad \qquad \vdots & \vdots & \quad \vdots  \\
          A_{M1}x_1 + A_{M2}x_2 + & 
          \cdots & + A_{MN}x_N
        \end{bmatrix}
    \end{align*}}
    \vspace{-3mm}
    \begin{align*}
    	\frac{\partial f_i}{\partial A_{iq}} = \only<1>{\colchar{$?$}{green}} \visible<2-5>{\underbrace{x_q}_{\in \: \R}}
    \quad
    \frac{\partial f_i}{\partial A_{i,:}} = \only<1-2>{\colchar{$?$}{green}} \visible<3-5>{\underbrace{\vec x\T}_{\in \: \R^{1\times 1\times N}}}
    \quad
    \frac{\partial f_i}{\partial A_{k\neq i,:}} = \only<1-3>{\colchar{$?$}{green}} \visible<4-5>{\underbrace{\vec 0\T}_{\in \R^{1\times 1\times  N}}}
    \quad
    \frac{\partial f_i}{\partial \mat A} =
    \only<1-4>{\colchar{$?$}{green}}
    \visible<5>{\footnotesize \underbrace{\begin{bmatrix}
      \vec 0\T\\
      \vdots\\
      \vec x\T\\
      \vdots\\
      \vec 0\T
    \end{bmatrix}}_{\in \: \R^{1\times (M\times N)}}}    
  \end{align*}
  {\color{gray} Index notation on the board}
\end{frame}



\begin{frame}{Chain rule}
\begin{itemize}
\item We now understand how gradients involving matrices are arranged in ``multidimensional arrays'' or ``tensors''. \pause
\item How do we perform the chain rule? Can we find a meaning for convenient notation like:
\begin{align}
\deriv[f]{\theta} = \deriv[f]{\mathbf A} \deriv[\mathbf A]{\theta} \text{?} && \deriv[f]{\theta} = \deriv[f]{\vg} \deriv[\vg]{\mathbf A} \text{?}
\end{align} \pause
\item Recall: Function $\vec
  f:\R^\text{\colchar{\scriptsize{$M\times N$}}{blue}}\to
  \R^\text{\colchar{\scriptsize{$P\times Q$}}{orange}}$, has a
  gradient that is a \mbox{\colchar{$(P\times Q)$}{orange} $\times$ \colchar{$(M\times N)$}{blue}} object (tensor)
  $$
    \diffF{\vec f}{\mat X}\in \R^{(P\times Q) \times (M \times N)}\,,\qquad 
    \mathrm{d}\vec f[p,q,m,n] =\frac{\partial f_{pq}}{\partial X_{mn}}
    $$
\end{itemize}
\end{frame}




\begin{frame}{Chain rule (2)}
\begin{itemize}
\item Start from index notation chain rule (\emph{always correct!}):
\begin{align*}
\pderiv[f_{pq}]{X_{mn}} = \sum_{rs} \pderiv[f_{pq}]{A_{rs}}\pderiv[A_{rs}]{X_{mn}}
\end{align*}\pause
\item Like matrix multiplication, but with \emph{vectorised} (vectors stacked column-by-column) matrices! 
\begin{align*}
\deriv[\mathrm{vec} (\vec f)]{\mathrm{vec} (\mat X)} = \deriv[\mathrm{vec} (\vec f)]{\mathrm{vec} (\mat A)}\deriv[\mathrm{vec} (\mat A)]{\mathrm{vec} (\mat X)}
\end{align*}
\pause
\item Keep track of grouping, sum over grouped indices:
\begin{align*}
\underbrace{\deriv[\vec f]{\mathbf X}}_{\text{\colchar{$(P\times Q)$}{orange} $\times$ \colchar{$(M\times N)$}{blue}}} = \underbrace{\deriv[\vec f]{\mathbf A}}_{\text{\colchar{$(P\times Q)$}{orange} $\times$ \colchar{$(R\times S)$}{green}}} \cdot \underbrace{\deriv[\mathbf A]{\mathbf X}}_{\text{\colchar{$(R\times S)$}{green} $\times$ \colchar{$(M\times N)$}{blue}}}
\end{align*}
\end{itemize}
\end{frame}


\begin{frame}{Summary: Matrix differentiation}
We saw:
\begin{itemize}
\item Principle is the same for matrix and vector differentiation \pause
\item Difference: Management of the numbers. It's about \emph{convention} \pause
\item Mathematical principle is index notation, convention is defined \pause
\end{itemize}

\vspace{0.4cm}

You should be able to:
\begin{itemize}
\item Do the bookkeeping of matrix derivative shapes
\item Compute derivatives of matrices
\item Abstract complex derivatives into the well-defined chain rule.
\item Describe the detailed index-wise summation for the chain rule.
\end{itemize}
\end{frame}



\end{document}