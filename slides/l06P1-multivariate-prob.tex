%% Time-stamp: <2018-10-18 20:24:12 (marc)>
\documentclass[xcolor=x11names,compress,mathserif]{beamer}

\newcommand{\hackspace}{\hspace{4.2mm}}
\newcommand{\showstudent}[1]{}
\newcommand\hmmax{0}
\newcommand\bmmax{0}



% talk/author information
\newcommand{\authorname}{Yingzhen Li}
\newcommand{\authoremail}{yingzhen.li@imperial.ac.uk}
\newcommand{\authoraffiliation}{
  Department of Computing\\Imperial
  College London}
\newcommand{\authortwitter}{liyzhen2}
\newcommand{\slidesettitle}{\imperialBlue{Multivariate Probability}}
\newcommand{\footertitle}{Multivariate Probability}
\newcommand{\location}{Imperial College London}
\newcommand{\talkDate}{October 28, 2022}



\date{\imperialGray{\talkDate}}



% load defaults
%\usepackage{../MarkMathCmds}
\selectcolormodel{rgb}
\usepackage{ifxetex,ifluatex}
\newif\ifxetexorluatex
\ifxetex
  \xetexorluatextrue
\else
  \ifluatex
    \xetexorluatextrue
  \else
    \xetexorluatexfalse
  \fi
\fi

\usepackage{textpos}
%\usepackage{arabtex}
\usepackage{tikz}
\usetikzlibrary{decorations.markings}
\usetikzlibrary{arrows}
\usetikzlibrary{shapes}
\usetikzlibrary{plotmarks}
\usetikzlibrary{mindmap,trees,backgrounds}

\tikzstyle{every picture}+=[remember picture]

%\usepackage{movie15}
% \usepackage{pdfpages}
%\usepackage{xmpmulti}

\usepackage{anyfontsize}
\usepackage{wrapfig}
\usepackage{animate}
\usepackage{multirow}
\usepackage{multimedia}
\usepackage{xmpmulti}
%\usepackage[latin9]{inputenc}
\usepackage[english]{babel}
\usepackage{scalefnt}
\usepackage{verbatim}
\usepackage{url}
% \usepackage{pgf,pgfarrows,pgfnodes}
\usepackage{textpos}
\usepackage[tight,ugly]{units}
\usepackage{url}
\usepackage{bbm}
\usepackage[english]{babel}
\usepackage{fancyhdr}
\usepackage{bm} % correct bold symbols, like \bm
\usepackage{amsmath}
\usepackage{amsfonts}
\usepackage{amssymb}
\usepackage{mathrsfs}
\usepackage{mathtools}
\usepackage{color}
\usepackage{cancel}
\usepackage{algorithm}
\usepackage{algpseudocode}
\usepackage{mathrsfs}
\usepackage{listings}
\usepackage{graphicx} % for pdf, bitmapped graphics files
\usepackage{mathtools}
\usepackage{units}
\usepackage{subfig}
\usepackage{enumerate}
\usepackage{natbib}
\usepackage{dsfont}


\ifxetexorluatex
\usepackage{fontspec}
\setmainfont[Scale=0.8]{OpenDyslexic-Regular}
\else
\usefonttheme{professionalfonts}
\fi

\renewcommand{\vec}[1]{{\boldsymbol{{#1}}}} % vector
\newcommand{\mat}[1]{{\boldsymbol{{#1}}}} % matrix
% \newcommand{\KL}[2]{\mathrm{KL}(#1\|#2)} % KL divergence
\newcommand{\R}[0]{\mathds{R}} % real numbers
\newcommand{\Z}[0]{\mathds{Z}} % integers
\newcommand{\tr}[0]{\text{tr}} % trace
% \newcommand{\inv}{^{-1}}
% \DeclareMathOperator*{\diag}{diag}
\newcommand{\E}{\mathds{E}} % expectation
\newcommand{\var}{\mathds{V}}
\newcommand{\gauss}[2]{\mathcal{N}\big(#1,\,#2\big)}
\newcommand{\gaussx}[3]{\mathcal{N}\big(#1\,|\,#2,\,#3\big)}
\newcommand{\gaussBig}[2]{\mathcal{N}\left(#1,\,#2\right)}
\newcommand{\gaussxBig}[3]{\mathcal{N}\left(#1\,\left|\,#2,\,#3\right.\right)}
\newcommand{\Ber}[0]{\mathrm{Ber}} % Bernoulli distribution
\DeclareMathOperator{\cov}{Cov}
\ifxetexorluatex
\renewcommand{\T}[0]{^\top}
\renewcommand{\d}[0]{\text{d}} % derivative
\else
\newcommand{\T}[0]{^\top}
\renewcommand{\d}[0]{\text{d}} % derivative
\fi
% calculus
\newcommand{\pdiff}[1]{\frac{\partial}{\partial #1}}
\newcommand{\pdiffF}[2]{\frac{\partial #1}{\partial #2}}
\newcommand{\diffF}[2]{\frac{{\d}#1}{{\d}#2}}
\newcommand{\diffFII}[2]{\frac{{\d}^2 #1}{{\d}#2^2}}
\newcommand{\diff}[1]{\frac{{\d}}{{\d}#1}}
\newcommand{\diffII}[1]{\frac{{\d}^2}{{\d}#1^2}}
\newcommand{\class}[0]{\mathcal{C}}

\newcommand{\idx}[1]{{(#1)}}
% \newcommand{\norm}[1]{\left\|#1\right\|}
\newcommand{\proj}[1]{\tilde{#1}}
\newcommand{\pcacoord}{z}
\newcommand{\pcacoordnew}{\zeta}
\newcommand{\latent}{z}
% \newcommand{\given}{\,|\,}
\newcommand{\genset}[1]{\mathrm{span}[#1]} % generating set
\newcommand{\set}[1]{\mathcal{#1}} % set
\newcommand{\fixgmfont}[1]{\scalebox{0.8}{#1}}



\usepackage{pifont}% http://ctan.org/pkg/pifont
\newcommand{\cmark}{{\color{green!40!black}\ding{51}}}%
\newcommand{\xmark}{{\color{red}\ding{55}}}%
\newcommand{\green}[1]{{\bf{\textcolor{green}{#1}}}}
\newcommand{\red}[1]{{\bf{\textcolor{red}{#1}}}}

\newcommand<>\red[1]{{\color#2[rgb]{1,0,0}#1}}
\newcommand<>\blue[1]{{\color#2[rgb]{0,0,1}#1}}
\newcommand<>\yellow[1]{{\color#2{camyellow}#1}}
\newcommand<>\green[1]{{\color#2[rgb]{0,0.6,0.0}#1}}
\newcommand<>\violet[1]{{\color#2[rgb]{0.6,0,0.6}#1}}
\newcommand<>\orange[1]{{\color#2[rgb]{1,0.5,0}#1}}
\newcommand<>\black[1]{{\color#2[rgb]{0,0,0}#1}}
\newcommand<>\steel[1]{{\color#2[rgb]{0,0,0.8}#1}}
\newcommand<>\darkblue[1]{{\color#2[rgb]{0,0,0.6}#1}}
\newcommand<>\lightblue[1]{{\color#2[rgb]{0.4,0.4,0.7}#1}}
\newcommand<>\gray[1]{{\color#2[rgb]{0.4,0.4,0.4}#1}}
\newcommand<>\greenish[1]{{\color#2[rgb]{0.45, 0.66, 0.45}#1}}
\newcommand<>\redish[1]{{\color#2[rgb]{0.7843    0.3706    0.3706}#1}}
\definecolor{redishTIKZ}{rgb}{0.7843, 0.3706, 0.3706}
\definecolor{imperialBlue}{rgb}{0.058, 0.219, 0.418}
\definecolor{aimsbrown}{rgb}{0.539, 0.117, 0.015}
% \definecolor{imperialGray}{rgb}{0.414, 0.488, 0.671 }
\definecolor{imperialGray}{RGB}{109,153, 204}
\definecolor{aimslightbrown}{RGB}{138,88,84}
\newcommand<>\imperialBlue[1]{{\color#2[rgb]{0.058, 0.219, 0.418}#1}}
\newcommand<>\aimsbrown[1]{{\color#2[rgb]{0.539, 0.117, 0.015}#1}}
%\newcommand<>\imperialGray[1]{{\color#2[rgb]{0.414, 0.488, 0.671}#1}}
\newcommand<>\imperialGray[1]{{\color#2[RGB]{109,153, 204}#1}}
\newcommand<>\aimslightbrown[1]{{\color#2[RGB]{138,88,84}#1}}
\newcommand<>\lightgray[1]{{\color#2[rgb]{0.8,0.8,0.8}#1}}
%\newcommand<>\highlightcolor[1]{{\color#2[rgb]{0,0,1}#1}}
\newcommand{\highlight}[1]{{\bf\steel{#1}}}
%\newcommand{\newblock}[0]{}

%\newcommand{\arrow}[0]{\includegraphics[height=5pt]{./figures/arrow}\hspace{3pt}}

\renewcommand{\emph}[1]{\textbf{\steel{{#1}}}}

\renewcommand{\alert}[1]{{\bf\red{{#1}}}}

\newcommand{\arrow}{
\begin{tikzpicture}
\draw [black!40!green, fill=black!40!green] (0,-0.12) -- (0,0.12) --
(0.15,0);
\draw [black!40!green, fill=black!40!green] (0.15,-0.12) -- (0.15,0.12) --
(0.3,0); 
\end{tikzpicture}
}

\geometry{left=0.45cm,top=0cm,right=0.45cm}


\newcommand{\logoimagepath}{./figures/imperial}
\newcommand{\highlightcolor}{blue!80!black}
%\newcommand{\headbarcolor}{imperialBlue}
\newcommand{\headbarcolor}{imperialBlue}
\institute{}

\newcommand{\coursetitle}{}

\newcommand{\slidesetsubtitle}{}
\newcommand{\slidesetnumber}{01}
\usefonttheme{professionalfonts}


\usetikzlibrary{decorations.fractals}
\input{../includes/tikzlibrarybayesnet.code.tex}
\input{../includes/tikzlibraryipe.code.tex}
\usetikzlibrary{matrix,positioning,decorations.pathreplacing}
\usetikzlibrary{calc,quotes,angles}
\usetikzlibrary{arrows, arrows.meta, patterns}

\usetikzlibrary{decorations.pathreplacing}
\tikzset{
    position label/.style={
       above = 3pt,
       text height = 2ex,
       text depth = 1ex
    }
}

% \usetikzlibrary{decorations.markings}
\tikzset{
  font={\fontsize{14pt}{12}\selectfont}
}



\useoutertheme[subsection=false,shadow]{miniframes}
\useinnertheme{default}
\usefonttheme{serif}
%\usepackage{palatino}
\usepackage{mathpazo}
%\usepackage{utopia}
\usepackage{stmaryrd} % for varodot, bigodot 
\usepackage{mathabx} % for \coAsterisk
%\usepackage{mnsymbol}
%\setbeamertemplate{itemize item}{\scriptsize\raise1.7pt\hbox{\donotcoloroutermaths$\Asterisk$}}
%\setbeamertemplate{itemize item}{\scriptsize\raise1.7pt\hbox{\donotcoloroutermaths$\varodot$}}
%\setbeamertemplate{itemize subitem}{\scriptsize\raise1.25pt\hbox{\donotcoloroutermaths$\rhd$}}

\usepackage{xifthen}% provides \isempty tesst

\setbeamerfont{title like}{shape=\scshape}
\setbeamerfont{frametitle}{}



\setbeamercolor*{lower separation line head}{bg=blue} 
\setbeamercolor*{normal text}{fg=black,bg=white} 
\setbeamercolor*{alerted text}{fg=red} 
\setbeamercolor*{example text}{fg=black} 
%\setbeamercolor*{frametitle}{fg=aimsbrown} 
\setbeamercolor*{frametitle}{fg=imperialBlue} 
\setbeamercolor*{structure}{fg=black} 
 
\setbeamercolor*{palette tertiary}{fg=black,bg=black!10} 
\setbeamercolor*{palette quaternary}{fg=black,bg=black!10} 

%\renewcommand{\(}{\begin{columns}}
%\renewcommand{\)}{\end{columns}}
%\newcommand{\<}[1]{\begin{column}{#1}}
%\renewcommand{\>}{\end{column}}

% ======================================
% custom commands 
\newcommand{\cemph}[1]{\textcolor{\highlightcolor}{#1}}
\newcommand{\calert}[1]{\textcolor{red}{#1}}

\setbeamertemplate{navigation symbols}{}
%\renewcommand\frametitle[1]{{\textsc{\Large \textcolor{\highlightcolor}{#1}}}\vspace{0.6cm}\par}

\setbeamertemplate{frametitle}
{
{\textsc\bf \insertframetitle}\vspace{0.2cm}\par
}


%%%%%%%%%%%%%%%%%%%%%%%%%%%%%%%%%%%%%%%%%%%%%%%%%%
\setbeamertemplate{headline}{% 
	\setbeamercolor{head1}{bg=\headbarcolor}
	 \hbox{%
  \begin{beamercolorbox}[wd=.01\paperwidth,ht=2.25ex,dp=50ex,center]{head1}%
  \fontsize{5}{5}\selectfont  
  \end{beamercolorbox}%
  }
  \vspace{-50ex}
}
\setbeamertemplate{footline}{
\begin{tiny}
\setbeamercolor{foot1}{fg=black,bg=gray!10}
\setbeamercolor{foot2}{fg=gray,bg=gray!15}
\setbeamercolor{foot3}{fg=gray,bg=gray!10}
\setbeamercolor{foot4}{fg=black,bg=gray!20}
\setbeamercolor{foot5}{fg=gray,bg=gray!15}
\setbeamercolor{foot6}{fg=black,bg=gray!20}

% taken from theme infolines and adapted
  \leavevmode%
  \hbox{%
  \begin{beamercolorbox}[wd=.45\paperwidth,ht=2.25ex,dp=1ex,center]{foot1}%
  \fontsize{5}{5}\selectfont
  \flushleft \hspace*{2ex}{\footertitle}
  \end{beamercolorbox}%
  % \begin{beamercolorbox}[wd=.08\paperwidth,ht=2.25ex,dp=1ex,center]{foot2}
  % \end{beamercolorbox}%
  %   \begin{beamercolorbox}[wd=.05\paperwidth,ht=2.25ex,dp=1ex,center]{foot3}
  % \end{beamercolorbox}%
    \begin{beamercolorbox}[wd=.45\paperwidth,ht=2.25ex,dp=1ex,center]{foot4}%
  \fontsize{5}{5}\selectfont
  \authorname\hspace{5mm}@\location, \talkDate%\ (\authorweb) 
  \end{beamercolorbox}%
  % \begin{beamercolorbox}[wd=.05\paperwidth,ht=2.25ex,dp=1ex,center]{foot5}
  % \end{beamercolorbox}%
  \begin{beamercolorbox}[wd=.1\paperwidth,ht=2.25ex,dp=1ex,right]{foot6}%
	\insertframenumber{}  \hspace*{2ex} 
  \end{beamercolorbox}}%
  \vskip0pt%
\end{tiny}
\vskip0pt
}


\setbeamertemplate{blocks}[rounded][shadow=false]


\newenvironment<>{myblock}[1]{%
  \begin{actionenv}#2%
      \def\insertblocktitle{#1}%
      \par%
      \mode<presentation>{%
%       \setbeamercolor{block title}{fg=black,bg=aimslightbrown!50!white}
      \setbeamercolor{block title}{fg=black,bg=imperialBlue!45!white}
       \setbeamercolor{block body}{fg=black,bg=gray!20}
       \setbeamercolor{itemize item}{fg=blue!40!white}
       \setbeamertemplate{itemize item}[triangle]
     }%
      \usebeamertemplate{block begin}}
    {\par\usebeamertemplate{block end}\end{actionenv}}

\newenvironment<>{myblock2}[1]{%
  \begin{actionenv}#2%
      \def\insertblocktitle{#1}%
      \par%
      \mode<presentation>{%
       \setbeamercolor{block title}{fg=white,bg=blue!80!black}
       \setbeamercolor{block body}{fg=black,bg=gray!20}
       \setbeamercolor{itemize item}{fg=green!60!black}
       \setbeamertemplate{itemize item}[triangle]
     }%
      \usebeamertemplate{block begin}}
    {\par\usebeamertemplate{block end}\end{actionenv}}

\gdef\colchar#1#2{%
  \tikz[baseline]{%
%  \node[anchor=base,inner sep=2pt,outer sep=0pt,fill = #2!20]
%  {\large{#1}};
  \node[anchor=base,inner sep=1pt,outer sep=0pt,fill = #2!20]
  {{\fontsize{11}{13}\selectfont #1}};
    }%
}%
\gdef\drawfontframe#1#2{%
  \tikz[baseline]{%
  \node[anchor=base,inner sep=2pt,outer sep=0pt,fill = #2!20] {#1};
    }%
  }%


\makeatletter
\let\@@magyar@captionfix\relax
\makeatother

%%% Local Variables:
%%% mode: latex
%%% TeX-master: "2018-09-arusha-linear-regression"
%%% End:

\usepackage{amssymb, amsmath, amsthm}
\usepackage{bm}
\DeclareMathOperator*{\argmax}{arg\,max}
\DeclareMathOperator*{\argmin}{arg\,min}

\newcommand{\bo}{\omega}
\newcommand{\KL}{\text{KL}}
\newcommand{\train}{\text{train}}
\newcommand{\D}{\mathcal{D}}
\newcommand{\softmax}{\text{Softmax}}

\newcommand{\logsumexp}{\text{log-sum-exp}}

%\newcommand{\R}{\mathbb{R}}
\newcommand{\N}{\mathcal{N}}
\newcommand{\cL}{\mathcal{L}}
\newcommand{\cO}{\mathcal{O}}
\newcommand{\svert}{~|~}
\newcommand{\td}{\text{d}}
\newcommand{\f}{\mathbf{f}}
\newcommand{\x}{\bm{x}}
\newcommand{\Bb}{\mathbf{b}}
\newcommand{\BB}{\mathbf{B}}
\newcommand{\BS}{\mathbf{S}}
\newcommand{\BA}{\mathbf{A}}
\newcommand{\BQ}{\mathbf{Q}}
\newcommand{\BP}{\mathbf{P}}
\newcommand{\BU}{\mathbf{U}}
\newcommand{\BV}{\mathbf{V}}
\newcommand{\Bg}{\mathbf{g}}
%\newcommand{\sBb}{\mathtt{b}}
\newcommand{\sBb}{\mathtt{z}}
\newcommand{\bx}{\overline{\x}}
\newcommand{\bb}{\overline{b}}
\newcommand{\y}{\mathbf{y}}
\newcommand{\z}{\bm{z}}
\newcommand{\bv}{\bm{v}}
\newcommand{\bV}{\mathbf{V}}
\newcommand{\bk}{\mathbf{k}}
\newcommand{\w}{\mathbf{w}}
\newcommand{\W}{\mathbf{W}}
\newcommand{\ba}{\mathbf{a}}
\newcommand{\m}{\mathbf{m}}
\newcommand{\ls}{\mathbf{l}}
\newcommand{\bL}{\mathbf{L}}
\newcommand{\A}{\mathbf{A}}
\newcommand{\X}{\mathbf{X}}
\newcommand{\Y}{\mathbf{Y}}
\newcommand{\F}{\mathbf{F}}
%\newcommand{\I}{\mathbf{I}}
\newcommand{\M}{\mathbf{M}}
\newcommand{\p}{\mathbf{p}}
\newcommand{\bp}{\overline{\p}}
\newcommand{\bz}{\mathbf{0}}
\newcommand{\bepsilon}{\text{\boldmath$\epsilon$}}
\newcommand{\bgamma}{\text{\boldmath$\gamma$}}
\newcommand{\s}{\mathbf{s}}
\newcommand{\Unif}{\text{Unif}}
\newcommand{\boh}{\widehat{\text{\boldmath$\omega$}}}
\newcommand{\bsigma}{\text{\boldmath$\sigma$}}
\newcommand{\bSigma}{\text{\boldmath$\Sigma$}}
\newcommand{\bmu}{\text{\boldmath$\mu$}}
\newcommand{\bphi}{\text{\boldmath$\phi$}}
\newcommand{\K}{\mathbf{K}}
\newcommand{\Kh}{\widehat{\mathbf{K}}}
\newcommand{\Cov}{\text{Cov}}
\newcommand{\Var}{\text{Var}}
%\newcommand{\tr}{\text{tr}}
\newcommand{\tdet}{\text{det}}
\newcommand{\diag}{\text{diag}}
% \newcommand{\KL}{\text{KL}}
\newcommand{\ind}{\mathds{1}}
\newcommand{\bc}{\mathbf{c}}
\newcommand{\reg}{\eta}
\newcommand{\weightdecay}{\lambda}
\newcommand{\h}{\mathbf{h}}

\newcommand{\ci}[0]{\perp\!\!\!\perp} % conditional independence

% variables
\newcommand{\mparam}{\bm{\theta}}	% model param
\newcommand{\vparam}{\bm{\phi}}	% variational param

% gradient approximation part
\newcommand{\hparam}{\bm{\varphi}}
\newcommand{\Xb}{\mathbb{X}}
\newcommand{\hgrad}{\overline{\nabla_{\x} \h}}
\newcommand{\Hmatrix}{\mathbf{H}}
\newcommand{\Grad}{\mathbf{G}}
\newcommand{\g}{\bm{g}}
\newcommand{\noise}{\bm{\epsilon}}
\newcommand{\data}{\mathcal{D}}




\newif\iflattersubsect

\AtBeginSection[] {
    \begin{frame}<beamer>
    \frametitle{Overview} %
    \tableofcontents[currentsection]  
    \end{frame}
    \lattersubsectfalse
}

\AtBeginSubsection[] {
    \iflattersubsect
    \begin{frame}<Coming Next>
    \frametitle{Overview} %
    \tableofcontents[currentsubsection]  
    \end{frame}
    \fi
    \lattersubsecttrue
}

\begin{document}


%%%%%%%%%%%%%%%%%%%%%%%%%%%%%%%%%%%%%%%%%%%%%%%%%%%%%%

{\setbeamertemplate{footline}{}
\begin{frame}
\title{\slidesettitle}
%\subtitle{SUBTITLE}
\author{\footnotesize
  \textbf{\authorname}
 }

 %%% LOGO

% \begin{flushright}
%   % \begin{columns}
%   %   \column{0.5\hsize}
%   %   \column{0.45\hsize}
%\includegraphics[height = 8mm]{./figures/qla}\hspace{2mm}
%     \includegraphics[height = 8mm]{./figures/aims-rwanda}\\[2mm]
%\includegraphics[height = 8mm]{./figures/imperial}
%%\end{columns}
%\end{flushright}

\vspace{-0cm}
%\begin{flushleft}
%\vspace{-1.5cm}{\small \textcolor{blue}{\coursetitle}}\\\vspace{2cm}
{\huge \slidesettitle \ifthenelse{\equal{\slidesetsubtitle}{}}%
    {}% if #1 is empty
    {: \\ {\large \slidesetsubtitle}}% if #1 is not empty
    } \\    
    %\vspace{20pt}
%\end{flushleft}
  
 
% this is all stuff below the talk title. make two columns, just in
% case you want to have a picture or a second affiliation here 
\begin{columns}[t]
\column{0.8\hsize}
%\begin{flushleft}
\begin{columns}[t]
\column{0.6\hsize}
\insertauthor \\[2mm]
\authoraffiliation\\[2mm]
\column{0.25\hsize}
\\[2mm]
\includegraphics[height = 0.3cm]{./figures-general/twitter}{\small @\authortwitter}\\[-1mm]
\mbox{\small \url{\authoremail}}
\end{columns}
\column{0.14\hsize}
\end{columns}
% \authorweb\\
\vspace{7mm}
% \aimslightbrown{The Nelson Mandela African Institute of Science and
%   Technology\\Arusha, Tanzania}\\[2mm]
\insertdate
%\end{flushleft}
\end{frame}
}

%%% Local Variables:
%%% mode: latex
%%% TeX-master: t
%%% End:

\linespread{1.2} 

%%%%% recap %%%%%%
\begin{frame}{Recap: univariate probability}
Univariate probability examples:

Bernoulli:
\begin{itemize}
	\item $X$ takes binary values $\{ 0, 1 \}$
	\item PMF: $p(X = 1) = \rho, \quad \rho \in [0, 1]$
\end{itemize}

Categorical:
\begin{itemize}
	\item $X$ takes values in $\{1, ..., C \}$
	\item PMF satisfies $\sum_{c=1}^C p(X = c) = 1, \quad p(X = c) \geq 0$
\end{itemize}

\end{frame}

\begin{frame}{Recap: univariate probability}
Univariate probability examples:

Gaussian:
\begin{itemize}
	\item $X$ takes continuous real number values in $\mathbb{R}$
	\item PDF: $p(X = x) = \frac{1}{\sqrt{2 \pi \sigma^2}} \exp[- \frac{1}{2 \sigma^2} (x - \mu)^2]$
	\item CDF: $F(x) = P(X \leq x) = \int_{-\infty}^{x} p(X = \alpha) d\alpha$
	\item Notice that $p(x) = \frac{dF(x)}{dx}$
\end{itemize}

\end{frame}

\begin{frame}{Lectures on multivariate probability}

Topic of today and next Monday: multivariate probability

\begin{itemize}
	\item Definitions and some examples
	\item Joint, marginal, and conditional distributions
	\item Sum rule and product rule
\end{itemize}

Lots of techniques to learn and master!
\begin{itemize}
	\item Change-of-variables rule
	\item Computing mean/variance/expectations
\end{itemize}

\end{frame}

%%%%%%%%% definition %%%%%%%

\begin{frame}{Multivariate probability}

We want to work with multiple random variables $X_1, ..., X_k$
\begin{itemize}
	\item Reuse the concepts introduced in univariate probability:
	\begin{itemize}
		\item Sample space $\Omega$, Event space $\mathcal{E}=2^{\Omega}$, Probability: $\mathbb{P}: \mathcal{E} \rightarrow [0, 1]$
	\end{itemize}

	\item $X_n: \Omega \rightarrow V_{X_n}$ maps $\omega \in \Omega$ to some integer/real value \pause

	\item We can define the support $\mathcal{A}$ in the value space of $X_1, ..., X_n$: \\
	$\mathcal{A} = \{ (x_1, x_2, ..., x_N) : X_n(\omega) = x_n, \omega \in \Omega \}$
	\begin{itemize}
		\item This means event $E \subset \Omega$ can be mapped to a measureable set $A \subset \mathcal{A}$, 
		so $P(A) := \mathbb{P}(E) \in [0, 1]$
	\end{itemize} \pause
	
	\item Multivariate PMF/PDF satisfies $p(x_1, x_2, ..., x_N) \geq 0$ and:
	$$\text{PMF: } \quad \sum_{(x_1, x_2, ..., x_N) \in A} p(x_1, x_2, ..., x_N) = P(A), \quad \forall A \subset \mathcal{A}.$$
	$$\text{PDF: } \quad \int_{A} p(x_1, x_2, ..., x_N) d x_1 d x_2 ... d x_N = P(A), \quad \forall A \subset \mathcal{A}.$$

\end{itemize}

\end{frame}

\begin{frame}{Multivariate probability}
Let's say you are in a zoo that has infinite number of animals:
\begin{figure}
\vspace{-0.7em}
\centering
\includegraphics[width=0.75\linewidth]{figures-multivariate-prob/sample_space_multivariate_vis.pdf}
\vspace{-1.3em}
\end{figure}
%
\only<1>{
\begin{itemize}
	\item Support: $\mathcal{A} \subset \mathbb{R} \times \mathbb{R} \times \mathbb{N}^{+}$
	\item A measurable subset in $\mathcal{A}$ can be $A = \{ 10.0 \leq x_1 \leq 50.0, 1 \leq x_2 \leq 10.0, x_3 \in \{2, 3, 4 \} \}$ \\ (``The animal's height, weight and fur colour are within some values/regimes'')
\end{itemize}
}
\only<2>{
\begin{itemize}
	\item Let's assume the event space $\mathcal{E} = 2^{\Omega}$
	\item Figuring out $P(A)$: find the biggest set $E \subset \Omega$ such that \\
	$(X_1, ..., X_N)(E) := \{ (X_1(\omega), ..., X_N(\omega)): \omega \in E \} \subset A$, \\
	then set $P(A) := \mathbb{P}(E)$
\end{itemize}
}
\only<3>{
\begin{itemize}
	\item $p(x_1, x_2, x_3)$ satisfies:
	$$\int \sum_{(x_1, x_2, x_3) \in A} p(x_1, x_2, x_3) d x_1 d x_2 = P(A), \quad \forall A \subset \mathcal{A}.$$
\end{itemize}
}

\end{frame}

\begin{frame}{Example: multinomial distribution}
Rolling a $k$-sided dice independently for $n$ times, define $X_i = \#$ side $i$
\begin{figure}
\vspace{-1em}
\centering
\includegraphics[width=0.6\linewidth]{figures-multivariate-prob/multinomial_dice_vis.pdf}
\vspace{-1em}
\end{figure}

\begin{itemize}
	\item Support: $\mathcal{A} = \{ x_1, ..., x_n \in \mathbb{N}: \sum_{i=1}^k x_i = n \} \subset \mathbb{N}^k$
	\item PMF: note that permuting elements in $\omega$ does not change $X_i$
	$$p(X_1 = x_1, ..., X_k = x_k) = \frac{n!}{x_1! \cdots x_k!} p_1^{x_1} \cdots p_k^{x_k}$$
\end{itemize}

\end{frame}

\begin{frame}{Example: multivariate Gaussian distribution}

Univariate Gaussian: $p(x) = \frac{1}{\sqrt{2 \pi \sigma^2}} \exp[- \frac{1}{2 \sigma^2} (x - \mu)^2]$

Multivariate Gaussian: $\x = (x_1, ..., x_d)^\top$
$$p(\x) =  \frac{1}{\sqrt{(2 \pi)^d |\Sigma|}} \exp[- \frac{1}{2} (\x - \bm{\mu})^{\top} \Sigma^{-1} (\x - \bm{\mu})]$$

\begin{itemize}
	\item Independent Gaussians: $\Sigma = diag(\sigma_1^2, ..., \sigma_d^2)$
\end{itemize}
\begin{figure}
\vspace{-1em}
\centering
\includegraphics[width=0.9\linewidth]{figures-multivariate-prob/gaussian_contours.pdf}
\vspace{-1em}
\end{figure}
\end{frame}

\begin{frame}{Example: multivariate Gaussian distribution}

Multivariate Gaussian: $\x = (x_1, ..., x_d)^\top$
$$p(\x) =  \frac{1}{\sqrt{(2 \pi)^d |\Sigma|}} \exp[- \frac{1}{2} \underbrace{(\x - \bm{\mu})^{\top} \Sigma^{-1} (\x - \bm{\mu})}_{:= \Delta^2} ]$$

\begin{itemize}
	\item Eigen decomposition of $\Sigma$:
	$$\Sigma = U \Lambda U^{\top} \quad \Rightarrow \quad \Sigma^{-1} = U \Lambda^{-1} U^{\top}, \quad \Lambda = diag(\lambda_1, ..., \lambda_d)$$
\end{itemize}

\begin{minipage}{0.58\linewidth}
\begin{itemize}
	\item Define $\y = U^\top (\x - \bm{\mu})$
	$$\Delta^2 = \y^\top \Lambda^{-1} \y = \sum_{i=1}^d \frac{y_i^2}{\lambda_i}$$
	\item Contour $\Delta^2 = C$ has an ``ellipse'' shape
\end{itemize}
\end{minipage}
\hfill
\begin{minipage}{0.4\linewidth}
\begin{figure}
\centering
\includegraphics[width=0.95\linewidth]{figures-multivariate-prob/gaussian_contour_bishop_book.png}
\vspace{-1em}
\end{figure}
\tiny{\hfill Fig from Bishop's PRML book}
\end{minipage}

\end{frame}



%%%%%%%% change of variable rule %%%%%%%%

\begin{frame}{Going beyond Gaussian: change-of-variables rule}
A common way to construct multivariate distribution beyond e.g., multinomial and Gaussian:
\begin{itemize}
	\item start from random variable $Z = (Z_1, ..., Z_K)$ with distribution $p_Z$
	\item use a transformation to get $X = T(Z)$
	\item this induces a distribution $p_X$ depending on $T$ and $p_Z$
\end{itemize}
\alert{Q:} What is the PMF/PDF of $X$ given $T$ and PMF/PDF of $Z$?
\end{frame}



\begin{frame}{Going beyond Gaussian: change-of-variables rule}

\emph{Key idea:} $p_X$ preserves the \emph{event probability} given by $\mathbb{P}$

\begin{figure}
\vspace{-0.5em}
\centering
\includegraphics[width=0.8\linewidth]{figures-multivariate-prob/change_of_variable_sample_space_vis.pdf}
\vspace{-1em}
\end{figure}\pause

Discrete case: if $T$ is invertible, then the PMF is
$$p_X(X = x) = p_Z(Z = T^{-1}(x)).$$

\end{frame}

\begin{frame}{Going beyond Gaussian: change-of-variables rule}
\emph{Key idea:} $p_X$ preserves the \emph{event probability} given by $\mathbb{P}$

Continuous case:
\begin{itemize}
	\item for any $x \in V_X = \mathbb{R}^{dim(X)}$, can work out $E_x = \{ \omega \in \Omega: T(Z(\omega)) = x \}$
	\item this means for any $S \subset V_X$, can work out $E_S = \cup_{x \in S} E_x$
	\item note that $\mathbb{P}$ does not change!
\end{itemize}
Assume $Z: \Omega \rightarrow V_Z = \mathbb{R}^{dim(Z)}$ maps $E_S$ to $U \subset V_Z$, then:
$$ U = \{z \in V_z: T(z) \in S \} := T^{-1}(S)$$
$$\Rightarrow \quad P_X(X \in S) = P_Z(Z \in T^{-1}(S)) = \mathbb{P}(E_S)$$
$$\Rightarrow \quad \int_{\alpha \in S} p_X(X = \alpha) d\alpha = \int_{\beta \in T^{-1}(S)} p_Z(Z = \beta) d\beta$$
\end{frame}

\begin{frame}{Going beyond Gaussian: change-of-variables rule}
\emph{Key idea:} $p_X$ preserves the \emph{event probability} given by $\mathbb{P}$

Continuous case: PDFs satisfy
$$\int_{\alpha \in S} p_X(X = \alpha) d\alpha = \int_{\beta \in T^{-1}(S)} p_Z(Z = \beta) d\beta, \ T^{-1}(S) = \{z \in V_z: T(z) \in S \}$$

For invertible and continuous $T$, to compute PDF $p_X$:
\only<1>{
\begin{itemize}
	\item let $dz$ be a very small neighbourhood around $z$, such that $p_Z(Z = z') \approx p_Z(Z = z), \forall z' \in dz$
	$$\Rightarrow \quad \int_{\beta \in dz} p_Z(Z = \beta) d\beta \approx p_Z(Z = z) dz$$
	\item $T(z) = x$, $\Rightarrow dx = T(dz)$ is also a very small neighbourhood around $x$, such that $p_X(X = x') \approx p_X(X = x), \forall x' \in dx$ 
	$$\Rightarrow \quad \int_{\alpha \in dx} p_X(X = \alpha) d\alpha \approx p_X(X = x) dx$$
\end{itemize}
}
%
\only<2>{
\begin{minipage}{0.5\linewidth}
\begin{itemize}
	\item Matching probability mass for the same event: \\
	$ p_X(X = x) dx = p_Z(Z = z) dz $
\end{itemize}
\end{minipage}
\hfill
\begin{minipage}{0.45\linewidth}
\begin{figure}
\centering
\includegraphics[width=0.9\linewidth]{figures-multivariate-prob/change_of_variable_volume_vis.pdf}
\end{figure}
\end{minipage}
\vspace{0.5em}

$$\Rightarrow \quad p_X(X = x) = p_Z(Z = z) |\frac{dz}{dx}| = p_{Z}(Z = T^{-1}(x)) |\frac{dT^{-1}(x)}{dx}|$$
}
\end{frame}

\begin{frame}{Going beyond Gaussian: change-of-variables rule}

\emph{Summary} of computing PMF/PDF of $X$ for invertible $T$:
\begin{itemize}
	\item Discrete case: $p_X(X = x) = p_Z(Z = T^{-1}(x)) $
	\item Continuous case: $p_X(X = x) = p_{Z}(Z = T^{-1}(x)) |\frac{dT^{-1}(x)}{dx}|$
\end{itemize}

\emph{Key idea:} $p_X$ preserves the \emph{event probability} given by $\mathbb{P}$
\begin{itemize}
	\item Probability $\mathbb{P}$ is defined on subsets of $\Omega$ 
	\item For $E \subset \Omega$, $U = Z(E)$ $S = T(U) = T \circ Z(E)$ are two different sets of ``quantitative descriptions'' of the elements in $E$
	\item So the underlying probability shouldn't change, i.e.,
	$$P_X(X \in S) = P_Z(Z \in U) = \mathbb{P}(E) $$
	\item PMF/PDF can be work out by ensuring this match for any $E \subset \Omega$
\end{itemize}

\end{frame}

%%%%%%%%% LOTUS %%%%%%%%

\begin{frame}{Law of the unconscious statistician (LOTUS)}

Computing expectation of $X$ given that $X = T(Z)$:

LOTUS rule:
$$\mathbb{E}_{X}[f(X)] = \mathbb{E}_{Z}[f(T(Z))]$$

\emph{Proof} for discrete case:

$$\mathbb{E}_{X}[f(X)] = \sum_{x} p_X(X = x) f(x)$$

\only<1>{
Recall from change-of-variables rule for discrete distribution:
\begin{equation*}
\begin{aligned}
	p_X(X = x) &= P_X(X \in \{x \}) = P_Z(Z \in T^{-1}(x)) \\
	&= \sum_{z \in T^{-1}(x)} P_Z(Z \in {z}) = \sum_{z: T(z) = x} p_Z(Z = z)
\end{aligned}
\end{equation*}
}
%
\only<2>{
\begin{equation*}
\begin{aligned}
	\Rightarrow \quad \mathbb{E}_{X}[f(X)] &= \sum_{x} \left( \sum_{z: T(z) = x} p_Z(Z = z) \right) f(x) \\
	&= \sum_{z} p_Z(Z = z) f(T(z)) = \mathbb{E}_{Z}[f(T(Z))]
\end{aligned}
\end{equation*}
}

\end{frame}

\begin{frame}{Law of the unconscious statistician (LOTUS)}

Computing expectation of $X$ given that $X = T(Z)$:

LOTUS rule:
$$\mathbb{E}_{X}[f(X)] = \mathbb{E}_{Z}[f(T(Z))]$$

\emph{Proof} for continuous case, assuming $T$ is invertible and continuous:

$$\mathbb{E}_{X}[f(X)] = \int p_X(x) f(x) dx$$

\only<1>{
Recall from change-of-variables rule for continuous distribution:
$$p_X(X = x) = p_{Z}(Z = T^{-1}(x)) |\frac{dT^{-1}(x)}{dx}|$$
Also note that $|\frac{dT^{-1}(x)}{dx}| = |\frac{dz}{dx}|$ for $z = T^{-1}(x)$
}
%
\only<2>{
\begin{equation*}
\begin{aligned}
	\Rightarrow \quad \mathbb{E}_{X}[f(X)] &= \int \left( p_{Z}(Z = z) |\frac{dz}{dx}| f(T(z)) \right)_{z = T^{-1}(x)} dx \\
	&= \int p_{Z}(Z = z) f(T(z)) dz= \mathbb{E}_{Z}[f(T(Z))]
\end{aligned}
\end{equation*}
}

\end{frame}

\begin{frame}{Law of the unconscious statistician (LOTUS)}

Computing expectation of $X$ given that $X = T(Z)$:

LOTUS rule:
$$\mathbb{E}_{X}[f(X)] = \mathbb{E}_{Z}[f(T(Z))]$$

LOTUS is true even when $T$ is not an invertible mapping

(The proof uses measure theory, not discussed in this course)

\end{frame}

%%%
\begin{frame}{Summary}

Today we covered:
\begin{itemize}
	\item Multivariate probability: definition and common examples
	\item Change-of-variables rule
	\item LOTUS
\end{itemize}

Next lecture: more on multivariate probability
\begin{itemize}
	\item vector mean and variance
	\item conditional distribution
	\item sum rule and product rule
\end{itemize}

\end{frame}


\end{document}
%%% Local Variables: 
%%% mode: latex
%%% TeX-master: t
%%% End: 
