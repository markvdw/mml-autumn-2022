%% Time-stamp: <2018-10-18 20:24:12 (marc)>
\documentclass[xcolor=x11names,compress,mathserif]{beamer}

\newcommand{\hackspace}{\hspace{4.2mm}}
\newcommand{\showstudent}[1]{}
\newcommand\hmmax{0}
\newcommand\bmmax{0}



% talk/author information
\newcommand{\authorname}{Yingzhen Li}
\newcommand{\authoremail}{yingzhen.li@imperial.ac.uk}
\newcommand{\authoraffiliation}{
  Department of Computing\\Imperial
  College London}
\newcommand{\authortwitter}{liyzhen2}
\newcommand{\slidesettitle}{\imperialBlue{More on Multivariate Probability}}
\newcommand{\footertitle}{More on Multivariate Probability}
\newcommand{\location}{Imperial College London}
\newcommand{\talkDate}{October 31, 2022}



\date{\imperialGray{\talkDate}}



% load defaults
%\usepackage{../MarkMathCmds}
\selectcolormodel{rgb}
\usepackage{ifxetex,ifluatex}
\newif\ifxetexorluatex
\ifxetex
  \xetexorluatextrue
\else
  \ifluatex
    \xetexorluatextrue
  \else
    \xetexorluatexfalse
  \fi
\fi

\usepackage{textpos}
%\usepackage{arabtex}
\usepackage{tikz}
\usetikzlibrary{decorations.markings}
\usetikzlibrary{arrows}
\usetikzlibrary{shapes}
\usetikzlibrary{plotmarks}
\usetikzlibrary{mindmap,trees,backgrounds}

\tikzstyle{every picture}+=[remember picture]

%\usepackage{movie15}
% \usepackage{pdfpages}
%\usepackage{xmpmulti}

\usepackage{anyfontsize}
\usepackage{wrapfig}
\usepackage{animate}
\usepackage{multirow}
\usepackage{multimedia}
\usepackage{xmpmulti}
%\usepackage[latin9]{inputenc}
\usepackage[english]{babel}
\usepackage{scalefnt}
\usepackage{verbatim}
\usepackage{url}
% \usepackage{pgf,pgfarrows,pgfnodes}
\usepackage{textpos}
\usepackage[tight,ugly]{units}
\usepackage{url}
\usepackage{bbm}
\usepackage[english]{babel}
\usepackage{fancyhdr}
\usepackage{bm} % correct bold symbols, like \bm
\usepackage{amsmath}
\usepackage{amsfonts}
\usepackage{amssymb}
\usepackage{mathrsfs}
\usepackage{mathtools}
\usepackage{color}
\usepackage{cancel}
\usepackage{algorithm}
\usepackage{algpseudocode}
\usepackage{mathrsfs}
\usepackage{listings}
\usepackage{graphicx} % for pdf, bitmapped graphics files
\usepackage{mathtools}
\usepackage{units}
\usepackage{subfig}
\usepackage{enumerate}
\usepackage{natbib}
\usepackage{dsfont}


\ifxetexorluatex
\usepackage{fontspec}
\setmainfont[Scale=0.8]{OpenDyslexic-Regular}
\else
\usefonttheme{professionalfonts}
\fi

\renewcommand{\vec}[1]{{\boldsymbol{{#1}}}} % vector
\newcommand{\mat}[1]{{\boldsymbol{{#1}}}} % matrix
% \newcommand{\KL}[2]{\mathrm{KL}(#1\|#2)} % KL divergence
\newcommand{\R}[0]{\mathds{R}} % real numbers
\newcommand{\Z}[0]{\mathds{Z}} % integers
\newcommand{\tr}[0]{\text{tr}} % trace
% \newcommand{\inv}{^{-1}}
% \DeclareMathOperator*{\diag}{diag}
\newcommand{\E}{\mathds{E}} % expectation
\newcommand{\var}{\mathds{V}}
\newcommand{\gauss}[2]{\mathcal{N}\big(#1,\,#2\big)}
\newcommand{\gaussx}[3]{\mathcal{N}\big(#1\,|\,#2,\,#3\big)}
\newcommand{\gaussBig}[2]{\mathcal{N}\left(#1,\,#2\right)}
\newcommand{\gaussxBig}[3]{\mathcal{N}\left(#1\,\left|\,#2,\,#3\right.\right)}
\newcommand{\Ber}[0]{\mathrm{Ber}} % Bernoulli distribution
\DeclareMathOperator{\cov}{Cov}
\ifxetexorluatex
\renewcommand{\T}[0]{^\top}
\renewcommand{\d}[0]{\text{d}} % derivative
\else
\newcommand{\T}[0]{^\top}
\renewcommand{\d}[0]{\text{d}} % derivative
\fi
% calculus
\newcommand{\pdiff}[1]{\frac{\partial}{\partial #1}}
\newcommand{\pdiffF}[2]{\frac{\partial #1}{\partial #2}}
\newcommand{\diffF}[2]{\frac{{\d}#1}{{\d}#2}}
\newcommand{\diffFII}[2]{\frac{{\d}^2 #1}{{\d}#2^2}}
\newcommand{\diff}[1]{\frac{{\d}}{{\d}#1}}
\newcommand{\diffII}[1]{\frac{{\d}^2}{{\d}#1^2}}
\newcommand{\class}[0]{\mathcal{C}}

\newcommand{\idx}[1]{{(#1)}}
% \newcommand{\norm}[1]{\left\|#1\right\|}
\newcommand{\proj}[1]{\tilde{#1}}
\newcommand{\pcacoord}{z}
\newcommand{\pcacoordnew}{\zeta}
\newcommand{\latent}{z}
% \newcommand{\given}{\,|\,}
\newcommand{\genset}[1]{\mathrm{span}[#1]} % generating set
\newcommand{\set}[1]{\mathcal{#1}} % set
\newcommand{\fixgmfont}[1]{\scalebox{0.8}{#1}}



\usepackage{pifont}% http://ctan.org/pkg/pifont
\newcommand{\cmark}{{\color{green!40!black}\ding{51}}}%
\newcommand{\xmark}{{\color{red}\ding{55}}}%
\newcommand{\green}[1]{{\bf{\textcolor{green}{#1}}}}
\newcommand{\red}[1]{{\bf{\textcolor{red}{#1}}}}

\newcommand<>\red[1]{{\color#2[rgb]{1,0,0}#1}}
\newcommand<>\blue[1]{{\color#2[rgb]{0,0,1}#1}}
\newcommand<>\yellow[1]{{\color#2{camyellow}#1}}
\newcommand<>\green[1]{{\color#2[rgb]{0,0.6,0.0}#1}}
\newcommand<>\violet[1]{{\color#2[rgb]{0.6,0,0.6}#1}}
\newcommand<>\orange[1]{{\color#2[rgb]{1,0.5,0}#1}}
\newcommand<>\black[1]{{\color#2[rgb]{0,0,0}#1}}
\newcommand<>\steel[1]{{\color#2[rgb]{0,0,0.8}#1}}
\newcommand<>\darkblue[1]{{\color#2[rgb]{0,0,0.6}#1}}
\newcommand<>\lightblue[1]{{\color#2[rgb]{0.4,0.4,0.7}#1}}
\newcommand<>\gray[1]{{\color#2[rgb]{0.4,0.4,0.4}#1}}
\newcommand<>\greenish[1]{{\color#2[rgb]{0.45, 0.66, 0.45}#1}}
\newcommand<>\redish[1]{{\color#2[rgb]{0.7843    0.3706    0.3706}#1}}
\definecolor{redishTIKZ}{rgb}{0.7843, 0.3706, 0.3706}
\definecolor{imperialBlue}{rgb}{0.058, 0.219, 0.418}
\definecolor{aimsbrown}{rgb}{0.539, 0.117, 0.015}
% \definecolor{imperialGray}{rgb}{0.414, 0.488, 0.671 }
\definecolor{imperialGray}{RGB}{109,153, 204}
\definecolor{aimslightbrown}{RGB}{138,88,84}
\newcommand<>\imperialBlue[1]{{\color#2[rgb]{0.058, 0.219, 0.418}#1}}
\newcommand<>\aimsbrown[1]{{\color#2[rgb]{0.539, 0.117, 0.015}#1}}
%\newcommand<>\imperialGray[1]{{\color#2[rgb]{0.414, 0.488, 0.671}#1}}
\newcommand<>\imperialGray[1]{{\color#2[RGB]{109,153, 204}#1}}
\newcommand<>\aimslightbrown[1]{{\color#2[RGB]{138,88,84}#1}}
\newcommand<>\lightgray[1]{{\color#2[rgb]{0.8,0.8,0.8}#1}}
%\newcommand<>\highlightcolor[1]{{\color#2[rgb]{0,0,1}#1}}
\newcommand{\highlight}[1]{{\bf\steel{#1}}}
%\newcommand{\newblock}[0]{}

%\newcommand{\arrow}[0]{\includegraphics[height=5pt]{./figures/arrow}\hspace{3pt}}

\renewcommand{\emph}[1]{\textbf{\steel{{#1}}}}

\renewcommand{\alert}[1]{{\bf\red{{#1}}}}

\newcommand{\arrow}{
\begin{tikzpicture}
\draw [black!40!green, fill=black!40!green] (0,-0.12) -- (0,0.12) --
(0.15,0);
\draw [black!40!green, fill=black!40!green] (0.15,-0.12) -- (0.15,0.12) --
(0.3,0); 
\end{tikzpicture}
}

\geometry{left=0.45cm,top=0cm,right=0.45cm}


\newcommand{\logoimagepath}{./figures/imperial}
\newcommand{\highlightcolor}{blue!80!black}
%\newcommand{\headbarcolor}{imperialBlue}
\newcommand{\headbarcolor}{imperialBlue}
\institute{}

\newcommand{\coursetitle}{}

\newcommand{\slidesetsubtitle}{}
\newcommand{\slidesetnumber}{01}
\usefonttheme{professionalfonts}


\usetikzlibrary{decorations.fractals}
\input{../includes/tikzlibrarybayesnet.code.tex}
\input{../includes/tikzlibraryipe.code.tex}
\usetikzlibrary{matrix,positioning,decorations.pathreplacing}
\usetikzlibrary{calc,quotes,angles}
\usetikzlibrary{arrows, arrows.meta, patterns}

\usetikzlibrary{decorations.pathreplacing}
\tikzset{
    position label/.style={
       above = 3pt,
       text height = 2ex,
       text depth = 1ex
    }
}

% \usetikzlibrary{decorations.markings}
\tikzset{
  font={\fontsize{14pt}{12}\selectfont}
}



\useoutertheme[subsection=false,shadow]{miniframes}
\useinnertheme{default}
\usefonttheme{serif}
%\usepackage{palatino}
\usepackage{mathpazo}
%\usepackage{utopia}
\usepackage{stmaryrd} % for varodot, bigodot 
\usepackage{mathabx} % for \coAsterisk
%\usepackage{mnsymbol}
%\setbeamertemplate{itemize item}{\scriptsize\raise1.7pt\hbox{\donotcoloroutermaths$\Asterisk$}}
%\setbeamertemplate{itemize item}{\scriptsize\raise1.7pt\hbox{\donotcoloroutermaths$\varodot$}}
%\setbeamertemplate{itemize subitem}{\scriptsize\raise1.25pt\hbox{\donotcoloroutermaths$\rhd$}}

\usepackage{xifthen}% provides \isempty tesst

\setbeamerfont{title like}{shape=\scshape}
\setbeamerfont{frametitle}{}



\setbeamercolor*{lower separation line head}{bg=blue} 
\setbeamercolor*{normal text}{fg=black,bg=white} 
\setbeamercolor*{alerted text}{fg=red} 
\setbeamercolor*{example text}{fg=black} 
%\setbeamercolor*{frametitle}{fg=aimsbrown} 
\setbeamercolor*{frametitle}{fg=imperialBlue} 
\setbeamercolor*{structure}{fg=black} 
 
\setbeamercolor*{palette tertiary}{fg=black,bg=black!10} 
\setbeamercolor*{palette quaternary}{fg=black,bg=black!10} 

%\renewcommand{\(}{\begin{columns}}
%\renewcommand{\)}{\end{columns}}
%\newcommand{\<}[1]{\begin{column}{#1}}
%\renewcommand{\>}{\end{column}}

% ======================================
% custom commands 
\newcommand{\cemph}[1]{\textcolor{\highlightcolor}{#1}}
\newcommand{\calert}[1]{\textcolor{red}{#1}}

\setbeamertemplate{navigation symbols}{}
%\renewcommand\frametitle[1]{{\textsc{\Large \textcolor{\highlightcolor}{#1}}}\vspace{0.6cm}\par}

\setbeamertemplate{frametitle}
{
{\textsc\bf \insertframetitle}\vspace{0.2cm}\par
}


%%%%%%%%%%%%%%%%%%%%%%%%%%%%%%%%%%%%%%%%%%%%%%%%%%
\setbeamertemplate{headline}{% 
	\setbeamercolor{head1}{bg=\headbarcolor}
	 \hbox{%
  \begin{beamercolorbox}[wd=.01\paperwidth,ht=2.25ex,dp=50ex,center]{head1}%
  \fontsize{5}{5}\selectfont  
  \end{beamercolorbox}%
  }
  \vspace{-50ex}
}
\setbeamertemplate{footline}{
\begin{tiny}
\setbeamercolor{foot1}{fg=black,bg=gray!10}
\setbeamercolor{foot2}{fg=gray,bg=gray!15}
\setbeamercolor{foot3}{fg=gray,bg=gray!10}
\setbeamercolor{foot4}{fg=black,bg=gray!20}
\setbeamercolor{foot5}{fg=gray,bg=gray!15}
\setbeamercolor{foot6}{fg=black,bg=gray!20}

% taken from theme infolines and adapted
  \leavevmode%
  \hbox{%
  \begin{beamercolorbox}[wd=.45\paperwidth,ht=2.25ex,dp=1ex,center]{foot1}%
  \fontsize{5}{5}\selectfont
  \flushleft \hspace*{2ex}{\footertitle}
  \end{beamercolorbox}%
  % \begin{beamercolorbox}[wd=.08\paperwidth,ht=2.25ex,dp=1ex,center]{foot2}
  % \end{beamercolorbox}%
  %   \begin{beamercolorbox}[wd=.05\paperwidth,ht=2.25ex,dp=1ex,center]{foot3}
  % \end{beamercolorbox}%
    \begin{beamercolorbox}[wd=.45\paperwidth,ht=2.25ex,dp=1ex,center]{foot4}%
  \fontsize{5}{5}\selectfont
  \authorname\hspace{5mm}@\location, \talkDate%\ (\authorweb) 
  \end{beamercolorbox}%
  % \begin{beamercolorbox}[wd=.05\paperwidth,ht=2.25ex,dp=1ex,center]{foot5}
  % \end{beamercolorbox}%
  \begin{beamercolorbox}[wd=.1\paperwidth,ht=2.25ex,dp=1ex,right]{foot6}%
	\insertframenumber{}  \hspace*{2ex} 
  \end{beamercolorbox}}%
  \vskip0pt%
\end{tiny}
\vskip0pt
}


\setbeamertemplate{blocks}[rounded][shadow=false]


\newenvironment<>{myblock}[1]{%
  \begin{actionenv}#2%
      \def\insertblocktitle{#1}%
      \par%
      \mode<presentation>{%
%       \setbeamercolor{block title}{fg=black,bg=aimslightbrown!50!white}
      \setbeamercolor{block title}{fg=black,bg=imperialBlue!45!white}
       \setbeamercolor{block body}{fg=black,bg=gray!20}
       \setbeamercolor{itemize item}{fg=blue!40!white}
       \setbeamertemplate{itemize item}[triangle]
     }%
      \usebeamertemplate{block begin}}
    {\par\usebeamertemplate{block end}\end{actionenv}}

\newenvironment<>{myblock2}[1]{%
  \begin{actionenv}#2%
      \def\insertblocktitle{#1}%
      \par%
      \mode<presentation>{%
       \setbeamercolor{block title}{fg=white,bg=blue!80!black}
       \setbeamercolor{block body}{fg=black,bg=gray!20}
       \setbeamercolor{itemize item}{fg=green!60!black}
       \setbeamertemplate{itemize item}[triangle]
     }%
      \usebeamertemplate{block begin}}
    {\par\usebeamertemplate{block end}\end{actionenv}}

\gdef\colchar#1#2{%
  \tikz[baseline]{%
%  \node[anchor=base,inner sep=2pt,outer sep=0pt,fill = #2!20]
%  {\large{#1}};
  \node[anchor=base,inner sep=1pt,outer sep=0pt,fill = #2!20]
  {{\fontsize{11}{13}\selectfont #1}};
    }%
}%
\gdef\drawfontframe#1#2{%
  \tikz[baseline]{%
  \node[anchor=base,inner sep=2pt,outer sep=0pt,fill = #2!20] {#1};
    }%
  }%


\makeatletter
\let\@@magyar@captionfix\relax
\makeatother

%%% Local Variables:
%%% mode: latex
%%% TeX-master: "2018-09-arusha-linear-regression"
%%% End:

\usepackage{amssymb, amsmath, amsthm}
\usepackage{bm}
\DeclareMathOperator*{\argmax}{arg\,max}
\DeclareMathOperator*{\argmin}{arg\,min}

\newcommand{\bo}{\omega}
\newcommand{\KL}{\text{KL}}
\newcommand{\train}{\text{train}}
\newcommand{\D}{\mathcal{D}}
\newcommand{\softmax}{\text{Softmax}}

\newcommand{\logsumexp}{\text{log-sum-exp}}

%\newcommand{\R}{\mathbb{R}}
\newcommand{\N}{\mathcal{N}}
\newcommand{\cL}{\mathcal{L}}
\newcommand{\cO}{\mathcal{O}}
\newcommand{\svert}{~|~}
\newcommand{\td}{\text{d}}
\newcommand{\f}{\mathbf{f}}
\newcommand{\x}{\bm{x}}
\newcommand{\Bb}{\mathbf{b}}
\newcommand{\BB}{\mathbf{B}}
\newcommand{\BS}{\mathbf{S}}
\newcommand{\BA}{\mathbf{A}}
\newcommand{\BQ}{\mathbf{Q}}
\newcommand{\BP}{\mathbf{P}}
\newcommand{\BU}{\mathbf{U}}
\newcommand{\BV}{\mathbf{V}}
\newcommand{\Bg}{\mathbf{g}}
%\newcommand{\sBb}{\mathtt{b}}
\newcommand{\sBb}{\mathtt{z}}
\newcommand{\bx}{\overline{\x}}
\newcommand{\bb}{\overline{b}}
\newcommand{\y}{\mathbf{y}}
\newcommand{\z}{\bm{z}}
\newcommand{\bv}{\bm{v}}
\newcommand{\bV}{\mathbf{V}}
\newcommand{\bk}{\mathbf{k}}
\newcommand{\w}{\mathbf{w}}
\newcommand{\W}{\mathbf{W}}
\newcommand{\ba}{\mathbf{a}}
\newcommand{\m}{\mathbf{m}}
\newcommand{\ls}{\mathbf{l}}
\newcommand{\bL}{\mathbf{L}}
\newcommand{\A}{\mathbf{A}}
\newcommand{\X}{\mathbf{X}}
\newcommand{\Y}{\mathbf{Y}}
\newcommand{\F}{\mathbf{F}}
%\newcommand{\I}{\mathbf{I}}
\newcommand{\M}{\mathbf{M}}
\newcommand{\p}{\mathbf{p}}
\newcommand{\bp}{\overline{\p}}
\newcommand{\bz}{\mathbf{0}}
\newcommand{\bepsilon}{\text{\boldmath$\epsilon$}}
\newcommand{\bgamma}{\text{\boldmath$\gamma$}}
\newcommand{\s}{\mathbf{s}}
\newcommand{\Unif}{\text{Unif}}
\newcommand{\boh}{\widehat{\text{\boldmath$\omega$}}}
\newcommand{\bsigma}{\text{\boldmath$\sigma$}}
\newcommand{\bSigma}{\text{\boldmath$\Sigma$}}
\newcommand{\bmu}{\text{\boldmath$\mu$}}
\newcommand{\bphi}{\text{\boldmath$\phi$}}
\newcommand{\K}{\mathbf{K}}
\newcommand{\Kh}{\widehat{\mathbf{K}}}
\newcommand{\Cov}{\text{Cov}}
\newcommand{\Var}{\text{Var}}
%\newcommand{\tr}{\text{tr}}
\newcommand{\tdet}{\text{det}}
\newcommand{\diag}{\text{diag}}
% \newcommand{\KL}{\text{KL}}
\newcommand{\ind}{\mathds{1}}
\newcommand{\bc}{\mathbf{c}}
\newcommand{\reg}{\eta}
\newcommand{\weightdecay}{\lambda}
\newcommand{\h}{\mathbf{h}}

\newcommand{\ci}[0]{\perp\!\!\!\perp} % conditional independence

% variables
\newcommand{\mparam}{\bm{\theta}}	% model param
\newcommand{\vparam}{\bm{\phi}}	% variational param

% gradient approximation part
\newcommand{\hparam}{\bm{\varphi}}
\newcommand{\Xb}{\mathbb{X}}
\newcommand{\hgrad}{\overline{\nabla_{\x} \h}}
\newcommand{\Hmatrix}{\mathbf{H}}
\newcommand{\Grad}{\mathbf{G}}
\newcommand{\g}{\bm{g}}
\newcommand{\noise}{\bm{\epsilon}}
\newcommand{\data}{\mathcal{D}}




\newif\iflattersubsect

\AtBeginSection[] {
    \begin{frame}<beamer>
    \frametitle{Overview} %
    \tableofcontents[currentsection]  
    \end{frame}
    \lattersubsectfalse
}

\AtBeginSubsection[] {
    \iflattersubsect
    \begin{frame}<Coming Next>
    \frametitle{Overview} %
    \tableofcontents[currentsubsection]  
    \end{frame}
    \fi
    \lattersubsecttrue
}

\begin{document}


%%%%%%%%%%%%%%%%%%%%%%%%%%%%%%%%%%%%%%%%%%%%%%%%%%%%%%

{\setbeamertemplate{footline}{}
\begin{frame}
\title{\slidesettitle}
%\subtitle{SUBTITLE}
\author{\footnotesize
  \textbf{\authorname}
 }

 %%% LOGO

% \begin{flushright}
%   % \begin{columns}
%   %   \column{0.5\hsize}
%   %   \column{0.45\hsize}
%\includegraphics[height = 8mm]{./figures/qla}\hspace{2mm}
%     \includegraphics[height = 8mm]{./figures/aims-rwanda}\\[2mm]
%\includegraphics[height = 8mm]{./figures/imperial}
%%\end{columns}
%\end{flushright}

\vspace{-0cm}
%\begin{flushleft}
%\vspace{-1.5cm}{\small \textcolor{blue}{\coursetitle}}\\\vspace{2cm}
{\huge \slidesettitle \ifthenelse{\equal{\slidesetsubtitle}{}}%
    {}% if #1 is empty
    {: \\ {\large \slidesetsubtitle}}% if #1 is not empty
    } \\    
    %\vspace{20pt}
%\end{flushleft}
  
 
% this is all stuff below the talk title. make two columns, just in
% case you want to have a picture or a second affiliation here 
\begin{columns}[t]
\column{0.8\hsize}
%\begin{flushleft}
\begin{columns}[t]
\column{0.6\hsize}
\insertauthor \\[2mm]
\authoraffiliation\\[2mm]
\column{0.25\hsize}
\\[2mm]
\includegraphics[height = 0.3cm]{./figures-general/twitter}{\small @\authortwitter}\\[-1mm]
\mbox{\small \url{\authoremail}}
\end{columns}
\column{0.14\hsize}
\end{columns}
% \authorweb\\
\vspace{7mm}
% \aimslightbrown{The Nelson Mandela African Institute of Science and
%   Technology\\Arusha, Tanzania}\\[2mm]
\insertdate
%\end{flushleft}
\end{frame}
}

%%% Local Variables:
%%% mode: latex
%%% TeX-master: t
%%% End:

\linespread{1.2} 

\begin{frame}{Recap: Multivariate probability}
Let's say you are in a zoo that has infinite number of animals:
\begin{figure}
\vspace{-0.7em}
\centering
\includegraphics[width=0.65\linewidth]{figures-multivariate-prob/sample_space_multivariate_vis.pdf}
\vspace{-1.5em}
\end{figure}
%
\begin{itemize}
	\item Support: $\mathcal{A} = \{(x_1, x_2, x_3): X_n(\omega) = x_n, \omega \in \Omega \}$
	\item Set $P(A) := \mathbb{P}(E)$ for the biggest event set $E \subset \Omega$ such that \\
	$(X_1, ..., X_N)(E) := \{ (X_1(\omega), ..., X_N(\omega)): \omega \in E \} \subset A$, \\
	\item PMF/PDF $p(x_1, x_2, x_3)$ satisfies:
	$$\int_{(x_1, x_2, x_3) \in A} p(x_1, x_2, x_3) d x_1 d x_2 d x_3 = P(A), \quad \forall A \subset \mathcal{A}.$$
\end{itemize}

\end{frame}


%%%%%% conditional probability %%%%%%%%%

\begin{frame}{Conditional probability}

\emph{Joint probability} of events $E_A, E_B \subset \Omega$, $\mathcal{E} = 2^{\Omega}$:
$$ \mathbb{P}(E_A \cap E_B)$$

\emph{Conditional probability}: given that $E_A$ occurs, what is the probability that $E_B$ also occurs?
$$ \mathbb{P}(E_B | E_A) := \frac{\mathbb{P}(E_A \cap E_B)}{ \mathbb{P}(E_A)}$$

By symmetricity:
$$ \mathbb{P}(E_A | E_B) := \frac{\mathbb{P}(E_A \cap E_B)}{ \mathbb{P}(E_B)}$$

We can write these expressions in random variable support space: \\ for $A, B \subset \mathcal{A}$,
$$ P(B | A) := \frac{P(A \cap B)}{ P(A)}$$

\end{frame}

\begin{frame}{Conditional probability}

\emph{Conditional probability} for $A, B \subset \mathcal{A}$,  
$\mathcal{A} = \{ (x_1, x_2, ..., x_N) : X_n(\omega) = x_n, \omega \in \Omega \}$,
$$ P(B | A) := \frac{P(A \cap B)}{ P(A)}$$

Specifically, we can define $P(X_2 \in A_2 | x_1 \in A_1)$ if we define:
$$A = \{ (x_1, x_2, ..., x_N) : X_n(\omega) = x_n, \omega \in \Omega, \textcolor{red}{x_1 \in A_1} \}$$
$$B = \{ (x_1, x_2, ..., x_N) : X_n(\omega) = x_n, \omega \in \Omega, \textcolor{red}{x_2 \in A_2} \}$$

\emph{Conditional} PMF/PDF $p(X_2 = x_2 | x_1 \in A_1)$ can be defined similar to the joint PMF/PDF case: just need to ensure $\forall A_1 \subset V_{X_1}, A_2 \subset V_{X_2}$
$$\int_{A_2} p(X_2 = x_2 | X_1 \in A_1) d x_2 = P(X_2 \in A_2 | X_1 \in A_1).$$

\end{frame}

\begin{frame}{Conditional probability}
Let's say you are in a zoo that has infinite number of animals:
\begin{figure}
\vspace{-0.7em}
\centering
\includegraphics[width=0.75\linewidth]{figures-multivariate-prob/sample_space_multivariate_vis.pdf}
\vspace{-1.3em}
\end{figure}
%
Joint probability $P( 1.0 \leq X_2 \leq 10.0, 10.0 \leq X_1 \leq 50.0)$:
\begin{itemize}
	\only<1>{
	\item Figure out the event sets 
	$$\hspace{-2em} E_1 = \{ \omega \in \Omega | 10.0 \leq X_1(\omega) \leq 50.0 \}, \quad E_2 = \{ \omega \in \Omega | 1.0 \leq X_2(\omega) \leq 10.0 \}$$
	$$\Rightarrow \quad E_1 \cap E_2 = \{ \omega \in \Omega | 10.0 \leq X_1(\omega) \leq 50.0, 1.0 \leq X_2(\omega) \leq 10.0 \}$$
	}
	\only<2>{
	\item Compute $P( 1.0 \leq X_2 \leq 10.0, 10.0 \leq X_1 \leq 50.0)$ as
	$$P( 1.0 \leq X_2 \leq 10.0, 10.0 \leq X_1 \leq 50.0) = \mathbb{P}(E_1 \cap E_2)$$
	}
\end{itemize}

\end{frame}

\begin{frame}{Conditional probability}
Let's say you are in a zoo that has infinite number of animals:
\begin{figure}
\vspace{-0.7em}
\centering
\includegraphics[width=0.75\linewidth]{figures-multivariate-prob/sample_space_multivariate_vis.pdf}
\vspace{-1.3em}
\end{figure}
%
Marginal probability $P( 10.0 \leq X_1 \leq 50.0)$:
\begin{itemize}
	\item Figure out the event $E_1 = \{ \omega \in \Omega | 10.0 \leq X_1(\omega) \leq 50.0 \}$, then
	$$P( 10.0 \leq X_1 \leq 50.0 ) = \mathbb{P}(E_1)$$
\end{itemize}

\end{frame}

\begin{frame}{Conditional probability}
Let's say you are in a zoo that has infinite number of animals:
\begin{figure}
\vspace{-0.7em}
\centering
\includegraphics[width=0.75\linewidth]{figures-multivariate-prob/sample_space_multivariate_vis.pdf}
\vspace{-1.3em}
\end{figure}
%
Conditional probability $P( 1.0 \leq X_2 \leq 10.0 | 10.0 \leq X_1 \leq 50.0)$:
\begin{itemize}
	\item Compute $P( 1.0 \leq X_2 \leq 10.0 | 10.0 \leq X_1 \leq 50.0)$ as
	$$P( 1.0 \leq X_2 \leq 10.0 | 10.0 \leq X_1 \leq 50.0) = \frac{\mathbb{P}(E_1 \cap E_2)}{\mathbb{P}(E_1)}$$
\end{itemize}

\end{frame}


%%%%%% sum rule and product rule %%%%%%

\begin{frame}{Sum rule and product rule}

By definition of the \emph{conditional probability}:
$$P(X_2 \in A_2 | X_1 \in A_1) := \frac{P(X_2 \in A_2, X_1 \in A_1)}{P(X_1 \in A_1)}$$

\emph{Product rule}:
$$P(X_2 \in A_2, X_1 \in A_1) = P(X_2 \in A_2 | X_1 \in A_1) \times P(X_1 \in A_1)$$
\begin{center}
``Joint dist. = conditional dist. $\times$ marginal dist.''
\end{center}

\visible<2->{
\emph{Sum rule}:
$$P(X_1 \in A_1) = \int_{V_{X_2}} p(X_1 \in A_1, X_2 = x_2) d x_2$$
\begin{center}
``Marginal dist. = sum/integral of joint dist.''
\end{center}
}

\visible<3>{
\emph{Combining both}:
$$P(X_1 \in A_1) = \int_{V_{X_2}} P(X_2 = x_2 | X_1 \in A_1) \times P(X_1 \in A_1) d x_2$$
}

\end{frame}


%%%%%%% conditional independence %%%%%%%%

\begin{frame}{Conditional independence}
\emph{Independence} of two random variables $X_1, X_2$:
$$X_1 \ci X_2 \quad \Leftrightarrow \quad p(X_1, X_2) = p(X_1) p(X_2)$$

Equivalently, using product rule:
$$X_1 \ci X_2 \quad \Leftrightarrow \quad p(X_1 | X_2) = p(X_1), p(X_2 | X_1) = p(X_2)$$

\visible<2>{
\emph{Conditional Independence} of two random variables $X_1, X_2$ given $X_3$:
$$X_1 \ci X_2 | X_3 \quad \Leftrightarrow \quad p(X_1, X_2 | X_3) = p(X_1 | X_3) p(X_2 | X_3)$$

Equivalently, using product rule:
$$X_1 \ci X_2 | X_3 \quad \Leftrightarrow \quad p(X_1 | X_2, X_3) = p(X_1 | X_3), p(X_2 | X_1, X_3) = p(X_2 | X_3)$$
}

\end{frame}


\begin{frame}{Conditional independence}
Example: drawing 5 cards from a standard 52-card poker deck

Define the following random variables:
\begin{itemize}
	\item $X_1$: number of hearts $\heartsuit$
	\item $X_2$: number of diamonds $\diamondsuit$
	\item $X_3$: number of clubs $\clubsuit$
	\item $X_4$: number of spades $\spadesuit$
\end{itemize}

What is the joint distribution $p(X_1, X_2, X_3, X_4)$ for the card draws --
\begin{itemize}
	\item with replacement?
	\item without replacement?
\end{itemize}

\end{frame}

%%%%%% vector mean & variance %%%%%%%%

\begin{frame}{Vector mean \& covariance}

Univariate case:
$$\text{mean:} \quad \mathbb{E}[X] = \int x p_X(x) dx $$
$$\text{variance:} \mathbb{V}[X] = \mathbb{E}[(X - \mathbb{E}[X])^2] = \int (x - \mathbb{E}[X])^2 p_X(x) dx $$

Multivariate case: write $X = (X_1, ..., X_N)^\top$
$$\text{mean:} \quad \mathbb{E}[X] = (\mathbb{E}[X_1], ..., \mathbb{E}[X_N])^\top $$
$$\text{covariance:} \quad \mathbb{V}[X] = \Sigma, \quad \Sigma_{ij} = \mathbb{E}[(X_i - \mathbb{E}[X_i]) (X_j - \mathbb{E}[X_j])] $$

Using \emph{sum rule}: only need marginals
$$\mathbb{E}[X_i] = \int x_i p_X(\x) d \x = \int x_i \textcolor{red}{p_X(X_i = x_i) d x_i} $$
\vspace{-1em}
\begin{align*}
\Sigma_{ij} &= \int (x_i - \mathbb{E}[X_i]) (x_j - \mathbb{E}[X_j]) p_X(\x) d \x \\
&= \int (x_i - \mathbb{E}[X_i]) (x_j - \mathbb{E}[X_j]) \textcolor{red}{p_X(X_i = x_i, X_j = x_j) d x_i d x_j}
\end{align*}
\end{frame}


\begin{frame}{Vector mean \& covariance}
Connecting univariate \& multivariate cases:

\begin{align*}
\mathbb{E}[\ba^\top X] &= \int \ba^\top \x p_X(\x) d \x
= \int \sum_i a_i x_i p(\x) d \x = \sum_i a_i \int x_i p(\x) d \x \\
&= \sum_i a_i \int x_i p(x_i) p(\{x_j\}_{j\neq i}|x_i) d \x  \\
&= \sum_i a_i \int x_i p(x_i) dx_i \int p(\{x_j\}_{j\neq i} | x_i) d \x_{-i} \\
&= \ba^\top (\mathbb{E}[X_1], ..., \mathbb{E}[X_N])^\top = \ba^\top \mathbb{E}[X]
\end{align*}
$\implies$ mean vector is the mean of each marginal!
\end{frame}


\begin{frame}{Vector mean \& coariance}
Connecting univariate \& multivariate cases: 

Write $\bar \x := \mathbb{E}[X]$ and use sum rule
\begin{align*}
\mathbb{V}[\ba^\top X] &= \int (\ba^\top \x - \ba^\top \bar \x)^2 p_X(\x) d \x \\
&= \int \big(\sum_i a_i x_i - a_i\bar x_i\big)\big(\sum_j a_j x_j - a_j\bar x_j\big) p_X(\x) d \x \\
&= \sum_i\sum_j a_ia_j \int (x_i-\bar x_i)(x_j-\bar x_j) \textcolor{red}{p_X(X_i = x_i, X_j = x_j) d x_i d x_j} \\
&= \sum_i\sum_j a_ia_j \Sigma_{ij} 
= \ba^\top \Sigma \ba
\end{align*}

The covariance $\Sigma$ allows us to find the scalar variance in any direction. 
\end{frame}


%%%%%%% conditional moments %%%%%%%%

\begin{frame}{Conditional expectations}

\emph{Expectation} of a function $f(X)$ under distribution $p(X)$:
$$\mathbb{E}_{p(X)}[f(X)] = \int f(x) p(X = x) dx$$


\emph{Conditional expectation} of a function $g(Y)$ given $X = x$ under conditional distribution $p(Y | X = x)$:

$$\mathbb{E}_{p(Y | X = x)}[g(Y)] = \int g(y) p(Y = y | X = x) dy$$

An equivalent notation: $\mathbb{E}[g(Y) | X = x] = \mathbb{E}_{p(Y | X = x)}[g(Y)]$

\end{frame}


\begin{frame}{Conditional expectations}
\emph{Conditional expectation} of a function $g(Y)$ given $X = x$:
$$\mathbb{E}[g(Y) | X = x] = \mathbb{E}_{p(Y | X = x)}[g(Y)] =  \int g(y) p(Y = y | X = x) dy$$

\begin{itemize}
	\item $\mathbb{E}[g(Y) | X = x]$ is a function of $x$
	\item As $X$ is a random variable, $\mathbb{E}[g(Y) | X]$ is also a random variable
\end{itemize}

\emph{Law of Total Expectation}:
\only<1>{
$$\mathbb{E}[Y] = \mathbb{E}[\mathbb{E}[Y | X]]$$
}
\only<2>{
$$\mathbb{E}_{p(Y)}[Y] = \mathbb{E}_{p(X)}[\mathbb{E}_{p(Y | X)}[Y]]$$
}
\only<3>{
\begin{equation*}
\begin{aligned}
	\mathbb{E}_{p(Y)}[Y] &= \mathbb{E}_{p(X)}[\mathbb{E}_{p(Y | X)}[Y]] \\
	&= \int p(X = x) \mathbb{E}_{p(Y | X = x)}[Y] dx
\end{aligned}
\end{equation*}
}
\only<4>{
\begin{equation*}
\begin{aligned}
	\mathbb{E}_{p(Y)}[Y] &= \mathbb{E}_{p(X)}[\mathbb{E}_{p(Y | X)}[Y]] \\
	&= \int p(X = x) p(Y = y | X = x) y dy dx
\end{aligned}
\end{equation*}
}
\only<5>{
\begin{equation*}
\begin{aligned}
	\mathbb{E}_{p(Y)}[Y] &= \mathbb{E}_{p(X)}[\mathbb{E}_{p(Y | X)}[Y]] \\
	&= \int p(Y = y, X = x) y dy dx	\quad \textcolor{red}{\text{(product rule)}}
\end{aligned}
\end{equation*}
}
\only<6>{
\begin{equation*}
\begin{aligned}
	\mathbb{E}_{p(Y)}[Y] &= \mathbb{E}_{p(X)}[\mathbb{E}_{p(Y | X)}[Y]] \\
	&= \int p(Y = y) y dy	\quad \textcolor{red}{\text{(sum rule)}}
\end{aligned}
\end{equation*}
}
\only<7>{

Extention to functions of $Y$:
$$\mathbb{E}[g(Y)] = \mathbb{E}[\mathbb{E}[g(Y) | X]]$$

}

\end{frame}


\begin{frame}{Conditional expectations}

\emph{Conditional variance} of $Y$ given $X = x$:
\begin{equation*}
\begin{aligned}
	\mathbb{V}[Y | X = x] &:= \mathbb{V}_{p(Y | X = x)}[Y] \\
	&= \mathbb{E}_{p(Y | X = x)}[ ( Y - \mathbb{E}_{p(Y | X = x)}[Y] )^2 ] \\
	&= \mathbb{E}[(Y - \mathbb{E}[Y | X = x])^2 | X = x]
\end{aligned}
\end{equation*}
\begin{itemize}
	\item $\mathbb{V}[Y | X = x]$ is a function of $x$
	\item As $X$ is a random variable, $\mathbb{V}[Y | X]$ is also a random variable
\end{itemize}

\emph{Law of Total Variance}:
$$\mathbb{V}[Y] = \mathbb{E}[\mathbb{V}[Y | X]] + \mathbb{V}[\mathbb{E}[Y | X]]$$

\end{frame}

%%%%%% summary %%%%%%%

\begin{frame}{Summary}

Topics we've covered about multivariate probability:

\begin{itemize}
	\item Definitions and some examples
	\item Joint, marginal, and conditional distributions
	\item Sum rule and product rule
	\item Change-of-variables rule
	\item Computing mean/variance/expectations
\end{itemize}

Next lecture: Model selection via cross-validation

\end{frame}

%%%%%%% appendix %%%%%%%
\begin{frame}{Appendix: math formula for deriving sum rule}

Deriving \emph{sum rule} using $\mathbb{P}$ defined on sets: 

Define for any $A_1 \subset V_{X_1}$, $A_2 \subset V_{X_2}$
$$\mathcal{A}(A_1) = \{(X_1(\omega), X_2(\omega), ..., X_N(\omega)): \textcolor{red}{X_1(\omega) \in A_1} \}$$
$$\mathcal{A}(A_2) = \{(X_1(\omega), X_2(\omega), ..., X_N(\omega)): \textcolor{blue}{X_2(\omega) \in A_2} \}$$
%
Now we can define a ``split'' of $X_2$ value space $V_{X_2}$:
$$ V_{X_2} = \cup_{k=1}^K A_2^k, \quad A_2^k \neq \emptyset, \quad A_2^i \cap A_2^j = \emptyset, \forall i \neq j$$
%
\only<1>{
\begin{equation*}
\begin{aligned}
\Rightarrow \quad \cup_{k=1}^K (\mathcal{A}(A_1) \cap \mathcal{A}(A_2^k)) &= \mathcal{A}(A_1) \cap (\cup_{k=1}^K \mathcal{A}(A_2^k)) \\
&= \mathcal{A}(A_1) \cap \mathcal{A}(V_{X_2}) \\
&= \mathcal{A}(A_1) \cap \mathcal{A} = \mathcal{A}(A_1)
\end{aligned}
\end{equation*}
}
%
\only<2>{
\begin{equation*}
\begin{aligned}
\Rightarrow \quad P(\cup_{k=1}^K (\mathcal{A}(A_1) \cap \mathcal{A}(A_2^k))) &= P(\mathcal{A}(A_1) \cap (\cup_{k=1}^K \mathcal{A}(A_2^k))) \\
&= P(\mathcal{A}(A_1) \cap \mathcal{A}(V_{X_2})) \\
&= P(\mathcal{A}(A_1) \cap \mathcal{A}) = P(\mathcal{A}(A_1))
\end{aligned}
\end{equation*}
}
%
\only<3>{

As $A_2^i \cap A_2^j = \emptyset \quad \Rightarrow \quad (\mathcal{A}(A_1) \cap \mathcal{A}(A_2^k)) \cap (\mathcal{A}(A_1) \cap \mathcal{A}(A_2^k)) = \emptyset$
%
$$\Rightarrow \quad P(\mathcal{A}(A_1)) = P(\cup_{k=1}^K (\mathcal{A}(A_1) \cap \mathcal{A}(A_2^k))) = \sum_{k=1}^K P( \mathcal{A}(A_1) \cap \mathcal{A}(A_2^k) )$$

}
%
\only<4>{

As $A_2^i \cap A_2^j = \emptyset \quad \Rightarrow \quad (\mathcal{A}(A_1) \cap \mathcal{A}(A_2^k)) \cap (\mathcal{A}(A_1) \cap \mathcal{A}(A_2^k)) = \emptyset$
%
$$\Rightarrow \quad P(X_1 \in A_1) = \sum_{k=1}^K P( X_1 \in A_1, X_2 \in A_2^k )$$

}
\only<5>{

As $A_2^i \cap A_2^j = \emptyset \quad \Rightarrow \quad (\mathcal{A}(A_1) \cap \mathcal{A}(A_2^k)) \cap (\mathcal{A}(A_1) \cap \mathcal{A}(A_2^k)) = \emptyset$
%
\begin{equation*}
\begin{aligned}
\Rightarrow \quad P(X_1 \in A_1)  &= \sum_{k=1}^K P( X_1 \in A_1, X_2 \in A_2^k ) \\ 
&= \sum_{k=1}^K \int_{A_2^k} p( X_1 \in A_1, X_2 = x_2 ) d x_2
\end{aligned}
\end{equation*}
}
\only<6>{

As $A_2^i \cap A_2^j = \emptyset \quad \Rightarrow \quad (\mathcal{A}(A_1) \cap \mathcal{A}(A_2^k)) \cap (\mathcal{A}(A_1) \cap \mathcal{A}(A_2^k)) = \emptyset$
%
\begin{equation*}
\begin{aligned}
\Rightarrow \quad P(X_1 \in A_1)  &= \sum_{k=1}^K P( X_1 \in A_1, X_2 \in A_2^k ) \\ 
&= \int_{V_{X_2}} p( X_1 \in A_1, X_2 = x_2 ) d x_2 \quad \text{\textcolor{red}{(sum rule)}}
\end{aligned}
\end{equation*}
}
\end{frame}

\end{document}
%%% Local Variables: 
%%% mode: latex
%%% TeX-master: t
%%% End: 
