%% Time-stamp: <2018-10-18 20:24:12 (marc)>
\documentclass[xcolor=x11names,compress,mathserif]{beamer}

\newcommand{\hackspace}{\hspace{4.2mm}}
\newcommand{\showstudent}[1]{}
\newcommand\hmmax{0}
\newcommand\bmmax{0}



% talk/author information
\newcommand{\authorname}{Yingzhen Li}
\newcommand{\authoremail}{yingzhen.li@imperial.ac.uk}
\newcommand{\authoraffiliation}{
  Department of Computing\\Imperial
  College London}
\newcommand{\authortwitter}{liyzhen2}
\newcommand{\slidesettitle}{\imperialBlue{Probabilistic Modelling Principles}}
\newcommand{\footertitle}{Probabilistic Modelling Principles}
\newcommand{\location}{Imperial College London}
\newcommand{\talkDate}{October 21, 2022}



\date{\imperialGray{\talkDate}}



% load defaults
%\usepackage{../MarkMathCmds}
\selectcolormodel{rgb}
\usepackage{ifxetex,ifluatex}
\newif\ifxetexorluatex
\ifxetex
  \xetexorluatextrue
\else
  \ifluatex
    \xetexorluatextrue
  \else
    \xetexorluatexfalse
  \fi
\fi

\usepackage{textpos}
%\usepackage{arabtex}
\usepackage{tikz}
\usetikzlibrary{decorations.markings}
\usetikzlibrary{arrows}
\usetikzlibrary{shapes}
\usetikzlibrary{plotmarks}
\usetikzlibrary{mindmap,trees,backgrounds}

\tikzstyle{every picture}+=[remember picture]

%\usepackage{movie15}
% \usepackage{pdfpages}
%\usepackage{xmpmulti}

\usepackage{anyfontsize}
\usepackage{wrapfig}
\usepackage{animate}
\usepackage{multirow}
\usepackage{multimedia}
\usepackage{xmpmulti}
%\usepackage[latin9]{inputenc}
\usepackage[english]{babel}
\usepackage{scalefnt}
\usepackage{verbatim}
\usepackage{url}
% \usepackage{pgf,pgfarrows,pgfnodes}
\usepackage{textpos}
\usepackage[tight,ugly]{units}
\usepackage{url}
\usepackage{bbm}
\usepackage[english]{babel}
\usepackage{fancyhdr}
\usepackage{bm} % correct bold symbols, like \bm
\usepackage{amsmath}
\usepackage{amsfonts}
\usepackage{amssymb}
\usepackage{mathrsfs}
\usepackage{mathtools}
\usepackage{color}
\usepackage{cancel}
\usepackage{algorithm}
\usepackage{algpseudocode}
\usepackage{mathrsfs}
\usepackage{listings}
\usepackage{graphicx} % for pdf, bitmapped graphics files
\usepackage{mathtools}
\usepackage{units}
\usepackage{subfig}
\usepackage{enumerate}
\usepackage{natbib}
\usepackage{dsfont}


\ifxetexorluatex
\usepackage{fontspec}
\setmainfont[Scale=0.8]{OpenDyslexic-Regular}
\else
\usefonttheme{professionalfonts}
\fi

\renewcommand{\vec}[1]{{\boldsymbol{{#1}}}} % vector
\newcommand{\mat}[1]{{\boldsymbol{{#1}}}} % matrix
% \newcommand{\KL}[2]{\mathrm{KL}(#1\|#2)} % KL divergence
\newcommand{\R}[0]{\mathds{R}} % real numbers
\newcommand{\Z}[0]{\mathds{Z}} % integers
\newcommand{\tr}[0]{\text{tr}} % trace
% \newcommand{\inv}{^{-1}}
% \DeclareMathOperator*{\diag}{diag}
\newcommand{\E}{\mathds{E}} % expectation
\newcommand{\var}{\mathds{V}}
\newcommand{\gauss}[2]{\mathcal{N}\big(#1,\,#2\big)}
\newcommand{\gaussx}[3]{\mathcal{N}\big(#1\,|\,#2,\,#3\big)}
\newcommand{\gaussBig}[2]{\mathcal{N}\left(#1,\,#2\right)}
\newcommand{\gaussxBig}[3]{\mathcal{N}\left(#1\,\left|\,#2,\,#3\right.\right)}
\newcommand{\Ber}[0]{\mathrm{Ber}} % Bernoulli distribution
\DeclareMathOperator{\cov}{Cov}
\ifxetexorluatex
\renewcommand{\T}[0]{^\top}
\renewcommand{\d}[0]{\text{d}} % derivative
\else
\newcommand{\T}[0]{^\top}
\renewcommand{\d}[0]{\text{d}} % derivative
\fi
% calculus
\newcommand{\pdiff}[1]{\frac{\partial}{\partial #1}}
\newcommand{\pdiffF}[2]{\frac{\partial #1}{\partial #2}}
\newcommand{\diffF}[2]{\frac{{\d}#1}{{\d}#2}}
\newcommand{\diffFII}[2]{\frac{{\d}^2 #1}{{\d}#2^2}}
\newcommand{\diff}[1]{\frac{{\d}}{{\d}#1}}
\newcommand{\diffII}[1]{\frac{{\d}^2}{{\d}#1^2}}
\newcommand{\class}[0]{\mathcal{C}}

\newcommand{\idx}[1]{{(#1)}}
% \newcommand{\norm}[1]{\left\|#1\right\|}
\newcommand{\proj}[1]{\tilde{#1}}
\newcommand{\pcacoord}{z}
\newcommand{\pcacoordnew}{\zeta}
\newcommand{\latent}{z}
% \newcommand{\given}{\,|\,}
\newcommand{\genset}[1]{\mathrm{span}[#1]} % generating set
\newcommand{\set}[1]{\mathcal{#1}} % set
\newcommand{\fixgmfont}[1]{\scalebox{0.8}{#1}}



\usepackage{pifont}% http://ctan.org/pkg/pifont
\newcommand{\cmark}{{\color{green!40!black}\ding{51}}}%
\newcommand{\xmark}{{\color{red}\ding{55}}}%
\newcommand{\green}[1]{{\bf{\textcolor{green}{#1}}}}
\newcommand{\red}[1]{{\bf{\textcolor{red}{#1}}}}

\newcommand<>\red[1]{{\color#2[rgb]{1,0,0}#1}}
\newcommand<>\blue[1]{{\color#2[rgb]{0,0,1}#1}}
\newcommand<>\yellow[1]{{\color#2{camyellow}#1}}
\newcommand<>\green[1]{{\color#2[rgb]{0,0.6,0.0}#1}}
\newcommand<>\violet[1]{{\color#2[rgb]{0.6,0,0.6}#1}}
\newcommand<>\orange[1]{{\color#2[rgb]{1,0.5,0}#1}}
\newcommand<>\black[1]{{\color#2[rgb]{0,0,0}#1}}
\newcommand<>\steel[1]{{\color#2[rgb]{0,0,0.8}#1}}
\newcommand<>\darkblue[1]{{\color#2[rgb]{0,0,0.6}#1}}
\newcommand<>\lightblue[1]{{\color#2[rgb]{0.4,0.4,0.7}#1}}
\newcommand<>\gray[1]{{\color#2[rgb]{0.4,0.4,0.4}#1}}
\newcommand<>\greenish[1]{{\color#2[rgb]{0.45, 0.66, 0.45}#1}}
\newcommand<>\redish[1]{{\color#2[rgb]{0.7843    0.3706    0.3706}#1}}
\definecolor{redishTIKZ}{rgb}{0.7843, 0.3706, 0.3706}
\definecolor{imperialBlue}{rgb}{0.058, 0.219, 0.418}
\definecolor{aimsbrown}{rgb}{0.539, 0.117, 0.015}
% \definecolor{imperialGray}{rgb}{0.414, 0.488, 0.671 }
\definecolor{imperialGray}{RGB}{109,153, 204}
\definecolor{aimslightbrown}{RGB}{138,88,84}
\newcommand<>\imperialBlue[1]{{\color#2[rgb]{0.058, 0.219, 0.418}#1}}
\newcommand<>\aimsbrown[1]{{\color#2[rgb]{0.539, 0.117, 0.015}#1}}
%\newcommand<>\imperialGray[1]{{\color#2[rgb]{0.414, 0.488, 0.671}#1}}
\newcommand<>\imperialGray[1]{{\color#2[RGB]{109,153, 204}#1}}
\newcommand<>\aimslightbrown[1]{{\color#2[RGB]{138,88,84}#1}}
\newcommand<>\lightgray[1]{{\color#2[rgb]{0.8,0.8,0.8}#1}}
%\newcommand<>\highlightcolor[1]{{\color#2[rgb]{0,0,1}#1}}
\newcommand{\highlight}[1]{{\bf\steel{#1}}}
%\newcommand{\newblock}[0]{}

%\newcommand{\arrow}[0]{\includegraphics[height=5pt]{./figures/arrow}\hspace{3pt}}

\renewcommand{\emph}[1]{\textbf{\steel{{#1}}}}

\renewcommand{\alert}[1]{{\bf\red{{#1}}}}

\newcommand{\arrow}{
\begin{tikzpicture}
\draw [black!40!green, fill=black!40!green] (0,-0.12) -- (0,0.12) --
(0.15,0);
\draw [black!40!green, fill=black!40!green] (0.15,-0.12) -- (0.15,0.12) --
(0.3,0); 
\end{tikzpicture}
}

\geometry{left=0.45cm,top=0cm,right=0.45cm}


\newcommand{\logoimagepath}{./figures/imperial}
\newcommand{\highlightcolor}{blue!80!black}
%\newcommand{\headbarcolor}{imperialBlue}
\newcommand{\headbarcolor}{imperialBlue}
\institute{}

\newcommand{\coursetitle}{}

\newcommand{\slidesetsubtitle}{}
\newcommand{\slidesetnumber}{01}
\usefonttheme{professionalfonts}


\usetikzlibrary{decorations.fractals}
\input{../includes/tikzlibrarybayesnet.code.tex}
\input{../includes/tikzlibraryipe.code.tex}
\usetikzlibrary{matrix,positioning,decorations.pathreplacing}
\usetikzlibrary{calc,quotes,angles}
\usetikzlibrary{arrows, arrows.meta, patterns}

\usetikzlibrary{decorations.pathreplacing}
\tikzset{
    position label/.style={
       above = 3pt,
       text height = 2ex,
       text depth = 1ex
    }
}

% \usetikzlibrary{decorations.markings}
\tikzset{
  font={\fontsize{14pt}{12}\selectfont}
}



\useoutertheme[subsection=false,shadow]{miniframes}
\useinnertheme{default}
\usefonttheme{serif}
%\usepackage{palatino}
\usepackage{mathpazo}
%\usepackage{utopia}
\usepackage{stmaryrd} % for varodot, bigodot 
\usepackage{mathabx} % for \coAsterisk
%\usepackage{mnsymbol}
%\setbeamertemplate{itemize item}{\scriptsize\raise1.7pt\hbox{\donotcoloroutermaths$\Asterisk$}}
%\setbeamertemplate{itemize item}{\scriptsize\raise1.7pt\hbox{\donotcoloroutermaths$\varodot$}}
%\setbeamertemplate{itemize subitem}{\scriptsize\raise1.25pt\hbox{\donotcoloroutermaths$\rhd$}}

\usepackage{xifthen}% provides \isempty tesst

\setbeamerfont{title like}{shape=\scshape}
\setbeamerfont{frametitle}{}



\setbeamercolor*{lower separation line head}{bg=blue} 
\setbeamercolor*{normal text}{fg=black,bg=white} 
\setbeamercolor*{alerted text}{fg=red} 
\setbeamercolor*{example text}{fg=black} 
%\setbeamercolor*{frametitle}{fg=aimsbrown} 
\setbeamercolor*{frametitle}{fg=imperialBlue} 
\setbeamercolor*{structure}{fg=black} 
 
\setbeamercolor*{palette tertiary}{fg=black,bg=black!10} 
\setbeamercolor*{palette quaternary}{fg=black,bg=black!10} 

%\renewcommand{\(}{\begin{columns}}
%\renewcommand{\)}{\end{columns}}
%\newcommand{\<}[1]{\begin{column}{#1}}
%\renewcommand{\>}{\end{column}}

% ======================================
% custom commands 
\newcommand{\cemph}[1]{\textcolor{\highlightcolor}{#1}}
\newcommand{\calert}[1]{\textcolor{red}{#1}}

\setbeamertemplate{navigation symbols}{}
%\renewcommand\frametitle[1]{{\textsc{\Large \textcolor{\highlightcolor}{#1}}}\vspace{0.6cm}\par}

\setbeamertemplate{frametitle}
{
{\textsc\bf \insertframetitle}\vspace{0.2cm}\par
}


%%%%%%%%%%%%%%%%%%%%%%%%%%%%%%%%%%%%%%%%%%%%%%%%%%
\setbeamertemplate{headline}{% 
	\setbeamercolor{head1}{bg=\headbarcolor}
	 \hbox{%
  \begin{beamercolorbox}[wd=.01\paperwidth,ht=2.25ex,dp=50ex,center]{head1}%
  \fontsize{5}{5}\selectfont  
  \end{beamercolorbox}%
  }
  \vspace{-50ex}
}
\setbeamertemplate{footline}{
\begin{tiny}
\setbeamercolor{foot1}{fg=black,bg=gray!10}
\setbeamercolor{foot2}{fg=gray,bg=gray!15}
\setbeamercolor{foot3}{fg=gray,bg=gray!10}
\setbeamercolor{foot4}{fg=black,bg=gray!20}
\setbeamercolor{foot5}{fg=gray,bg=gray!15}
\setbeamercolor{foot6}{fg=black,bg=gray!20}

% taken from theme infolines and adapted
  \leavevmode%
  \hbox{%
  \begin{beamercolorbox}[wd=.45\paperwidth,ht=2.25ex,dp=1ex,center]{foot1}%
  \fontsize{5}{5}\selectfont
  \flushleft \hspace*{2ex}{\footertitle}
  \end{beamercolorbox}%
  % \begin{beamercolorbox}[wd=.08\paperwidth,ht=2.25ex,dp=1ex,center]{foot2}
  % \end{beamercolorbox}%
  %   \begin{beamercolorbox}[wd=.05\paperwidth,ht=2.25ex,dp=1ex,center]{foot3}
  % \end{beamercolorbox}%
    \begin{beamercolorbox}[wd=.45\paperwidth,ht=2.25ex,dp=1ex,center]{foot4}%
  \fontsize{5}{5}\selectfont
  \authorname\hspace{5mm}@\location, \talkDate%\ (\authorweb) 
  \end{beamercolorbox}%
  % \begin{beamercolorbox}[wd=.05\paperwidth,ht=2.25ex,dp=1ex,center]{foot5}
  % \end{beamercolorbox}%
  \begin{beamercolorbox}[wd=.1\paperwidth,ht=2.25ex,dp=1ex,right]{foot6}%
	\insertframenumber{}  \hspace*{2ex} 
  \end{beamercolorbox}}%
  \vskip0pt%
\end{tiny}
\vskip0pt
}


\setbeamertemplate{blocks}[rounded][shadow=false]


\newenvironment<>{myblock}[1]{%
  \begin{actionenv}#2%
      \def\insertblocktitle{#1}%
      \par%
      \mode<presentation>{%
%       \setbeamercolor{block title}{fg=black,bg=aimslightbrown!50!white}
      \setbeamercolor{block title}{fg=black,bg=imperialBlue!45!white}
       \setbeamercolor{block body}{fg=black,bg=gray!20}
       \setbeamercolor{itemize item}{fg=blue!40!white}
       \setbeamertemplate{itemize item}[triangle]
     }%
      \usebeamertemplate{block begin}}
    {\par\usebeamertemplate{block end}\end{actionenv}}

\newenvironment<>{myblock2}[1]{%
  \begin{actionenv}#2%
      \def\insertblocktitle{#1}%
      \par%
      \mode<presentation>{%
       \setbeamercolor{block title}{fg=white,bg=blue!80!black}
       \setbeamercolor{block body}{fg=black,bg=gray!20}
       \setbeamercolor{itemize item}{fg=green!60!black}
       \setbeamertemplate{itemize item}[triangle]
     }%
      \usebeamertemplate{block begin}}
    {\par\usebeamertemplate{block end}\end{actionenv}}

\gdef\colchar#1#2{%
  \tikz[baseline]{%
%  \node[anchor=base,inner sep=2pt,outer sep=0pt,fill = #2!20]
%  {\large{#1}};
  \node[anchor=base,inner sep=1pt,outer sep=0pt,fill = #2!20]
  {{\fontsize{11}{13}\selectfont #1}};
    }%
}%
\gdef\drawfontframe#1#2{%
  \tikz[baseline]{%
  \node[anchor=base,inner sep=2pt,outer sep=0pt,fill = #2!20] {#1};
    }%
  }%


\makeatletter
\let\@@magyar@captionfix\relax
\makeatother

%%% Local Variables:
%%% mode: latex
%%% TeX-master: "2018-09-arusha-linear-regression"
%%% End:

\usepackage{amssymb, amsmath, amsthm}
\usepackage{bm}
\DeclareMathOperator*{\argmax}{arg\,max}
\DeclareMathOperator*{\argmin}{arg\,min}

\newcommand{\bo}{\omega}
\newcommand{\KL}{\text{KL}}
\newcommand{\train}{\text{train}}
\newcommand{\D}{\mathcal{D}}
\newcommand{\softmax}{\text{Softmax}}

\newcommand{\logsumexp}{\text{log-sum-exp}}

%\newcommand{\R}{\mathbb{R}}
\newcommand{\N}{\mathcal{N}}
\newcommand{\cL}{\mathcal{L}}
\newcommand{\cO}{\mathcal{O}}
\newcommand{\svert}{~|~}
\newcommand{\td}{\text{d}}
\newcommand{\f}{\mathbf{f}}
\newcommand{\x}{\bm{x}}
\newcommand{\Bb}{\mathbf{b}}
\newcommand{\BB}{\mathbf{B}}
\newcommand{\BS}{\mathbf{S}}
\newcommand{\BA}{\mathbf{A}}
\newcommand{\BQ}{\mathbf{Q}}
\newcommand{\BP}{\mathbf{P}}
\newcommand{\BU}{\mathbf{U}}
\newcommand{\BV}{\mathbf{V}}
\newcommand{\Bg}{\mathbf{g}}
%\newcommand{\sBb}{\mathtt{b}}
\newcommand{\sBb}{\mathtt{z}}
\newcommand{\bx}{\overline{\x}}
\newcommand{\bb}{\overline{b}}
\newcommand{\y}{\mathbf{y}}
\newcommand{\z}{\bm{z}}
\newcommand{\bv}{\bm{v}}
\newcommand{\bV}{\mathbf{V}}
\newcommand{\bk}{\mathbf{k}}
\newcommand{\w}{\mathbf{w}}
\newcommand{\W}{\mathbf{W}}
\newcommand{\ba}{\mathbf{a}}
\newcommand{\m}{\mathbf{m}}
\newcommand{\ls}{\mathbf{l}}
\newcommand{\bL}{\mathbf{L}}
\newcommand{\A}{\mathbf{A}}
\newcommand{\X}{\mathbf{X}}
\newcommand{\Y}{\mathbf{Y}}
\newcommand{\F}{\mathbf{F}}
%\newcommand{\I}{\mathbf{I}}
\newcommand{\M}{\mathbf{M}}
\newcommand{\p}{\mathbf{p}}
\newcommand{\bp}{\overline{\p}}
\newcommand{\bz}{\mathbf{0}}
\newcommand{\bepsilon}{\text{\boldmath$\epsilon$}}
\newcommand{\bgamma}{\text{\boldmath$\gamma$}}
\newcommand{\s}{\mathbf{s}}
\newcommand{\Unif}{\text{Unif}}
\newcommand{\boh}{\widehat{\text{\boldmath$\omega$}}}
\newcommand{\bsigma}{\text{\boldmath$\sigma$}}
\newcommand{\bSigma}{\text{\boldmath$\Sigma$}}
\newcommand{\bmu}{\text{\boldmath$\mu$}}
\newcommand{\bphi}{\text{\boldmath$\phi$}}
\newcommand{\K}{\mathbf{K}}
\newcommand{\Kh}{\widehat{\mathbf{K}}}
\newcommand{\Cov}{\text{Cov}}
\newcommand{\Var}{\text{Var}}
%\newcommand{\tr}{\text{tr}}
\newcommand{\tdet}{\text{det}}
\newcommand{\diag}{\text{diag}}
% \newcommand{\KL}{\text{KL}}
\newcommand{\ind}{\mathds{1}}
\newcommand{\bc}{\mathbf{c}}
\newcommand{\reg}{\eta}
\newcommand{\weightdecay}{\lambda}
\newcommand{\h}{\mathbf{h}}

\newcommand{\ci}[0]{\perp\!\!\!\perp} % conditional independence

% variables
\newcommand{\mparam}{\bm{\theta}}	% model param
\newcommand{\vparam}{\bm{\phi}}	% variational param

% gradient approximation part
\newcommand{\hparam}{\bm{\varphi}}
\newcommand{\Xb}{\mathbb{X}}
\newcommand{\hgrad}{\overline{\nabla_{\x} \h}}
\newcommand{\Hmatrix}{\mathbf{H}}
\newcommand{\Grad}{\mathbf{G}}
\newcommand{\g}{\bm{g}}
\newcommand{\noise}{\bm{\epsilon}}
\newcommand{\data}{\mathcal{D}}




\newif\iflattersubsect

\AtBeginSection[] {
    \begin{frame}<beamer>
    \frametitle{Overview} %
    \tableofcontents[currentsection]  
    \end{frame}
    \lattersubsectfalse
}

\AtBeginSubsection[] {
    \iflattersubsect
    \begin{frame}<Coming Next>
    \frametitle{Overview} %
    \tableofcontents[currentsubsection]  
    \end{frame}
    \fi
    \lattersubsecttrue
}

\begin{document}


%%%%%%%%%%%%%%%%%%%%%%%%%%%%%%%%%%%%%%%%%%%%%%%%%%%%%%

{\setbeamertemplate{footline}{}
\begin{frame}
\title{\slidesettitle}
%\subtitle{SUBTITLE}
\author{\footnotesize
  \textbf{\authorname}
 }

 %%% LOGO

% \begin{flushright}
%   % \begin{columns}
%   %   \column{0.5\hsize}
%   %   \column{0.45\hsize}
%\includegraphics[height = 8mm]{./figures/qla}\hspace{2mm}
%     \includegraphics[height = 8mm]{./figures/aims-rwanda}\\[2mm]
%\includegraphics[height = 8mm]{./figures/imperial}
%%\end{columns}
%\end{flushright}

\vspace{-0cm}
%\begin{flushleft}
%\vspace{-1.5cm}{\small \textcolor{blue}{\coursetitle}}\\\vspace{2cm}
{\huge \slidesettitle \ifthenelse{\equal{\slidesetsubtitle}{}}%
    {}% if #1 is empty
    {: \\ {\large \slidesetsubtitle}}% if #1 is not empty
    } \\    
    %\vspace{20pt}
%\end{flushleft}
  
 
% this is all stuff below the talk title. make two columns, just in
% case you want to have a picture or a second affiliation here 
\begin{columns}[t]
\column{0.8\hsize}
%\begin{flushleft}
\begin{columns}[t]
\column{0.6\hsize}
\insertauthor \\[2mm]
\authoraffiliation\\[2mm]
\column{0.25\hsize}
\\[2mm]
\includegraphics[height = 0.3cm]{./figures-general/twitter}{\small @\authortwitter}\\[-1mm]
\mbox{\small \url{\authoremail}}
\end{columns}
\column{0.14\hsize}
\end{columns}
% \authorweb\\
\vspace{7mm}
% \aimslightbrown{The Nelson Mandela African Institute of Science and
%   Technology\\Arusha, Tanzania}\\[2mm]
\insertdate
%\end{flushleft}
\end{frame}
}

%%% Local Variables:
%%% mode: latex
%%% TeX-master: t
%%% End:

\linespread{1.2} 

%\begin{frame}{Reading for this week}
%
%\begin{center}
%Read MML book: Sections 4.1 - 4.4, 7.1, Chapter 9 up to 9.2.3 \\
%Do MML book exercises: Exercises 4.1 - 4.7 \\
%An extra exercise will be uploaded to course materials.
%\end{center}
%
%\end{frame}


%%%% linear regression example %%%%%%%


%%%%%%%%%%%%%%%% probabilistic models %%%%%%%%%%%%%%%

\begin{frame}{Principles of probabilistic modelling}

Have you ever wondered about the following questions:
\begin{itemize}
	\item Why using $\ell_2$ loss in many regression problems?
	\item Where does the cross-entropy loss come from?
	\item What is a good principle for choosing a good loss function?
\end{itemize} \pause

\begin{center}
\emph{Probabilistic modelling} gives you good answers for all of them!
\end{center}

\end{frame}

\begin{frame}{Principles of probabilistic modelling}

Probabilistic modelling is about:
\begin{itemize}
	\item[1.] making model assumptions on \alert{how the data is generated}
	\item[2.] estimating model parameters under probabilistic principles
	\item[3.] model checking using data, and repeat 1 - 3
	\item[4.] using the fitted model for downstream tasks
\end{itemize}

\end{frame}

%%% adding coin flip example %%%%%%
\begin{frame}{Example: coin flipping}

Imagine you'd like to predict the next coin flip result:
\begin{figure}
\centering
\includegraphics[width=0.6\linewidth]{figures-probmodel/coin_flips.png}
\end{figure}

\begin{itemize}
	\item Assume $x_1, x_2, ..., x_N$ are observed \emph{independent} coin flip results using the \emph{same} coin,
	\item I.e., $x_1, ..., x_N$ are sampled i.i.d. from the same \emph{data distribution} $\pi(x)$
	\item However, we don't know $\pi(x)$
\end{itemize}
\end{frame}

\begin{frame}{Example: coin flipping}

Imagine you'd like to predict the next coin flip result:
\begin{figure}
\centering
\includegraphics[width=0.6\linewidth]{figures-probmodel/coin_flips.png}
\end{figure}

Probabilistic modelling is about:
\begin{itemize}
	\item[1.] Assume $x$ is sampled from $p(x | \mparam)$ \alert{$\Leftarrow$ our probabilistic model}
	\item[2.] estimating $\bm{\theta}$ under probabilistic principles \\ such as MLE, MAP, posterior inference \alert{$\Leftarrow$ learning the model}
	\item[3.] check if $p(x | \bm{\theta}^*)$ fits $\pi(x)$ well, and repeat 1 - 3 \alert{$\Leftarrow$ model checking}
	\item[4.] making prediction for next coin flip result using $p(x | \mparam^*)$
\end{itemize}
\end{frame}

\begin{frame}{Example: coin flipping}

Imagine you'd like to predict the next coin flip result:
\begin{figure}
\centering
\includegraphics[width=0.6\linewidth]{figures-probmodel/coin_flips.png}
\end{figure}

Step 1: Assume $x$ is sampled from $p(x | \mparam)$
\begin{equation*}
x = \begin{cases}
1, \quad \text{with probability } \mparam \\
0, \quad \text{with probability } 1 - \mparam
\end{cases}, \quad \mparam \in [0, 1].
\end{equation*}
$$\Leftrightarrow \quad p(x | \mparam) = \text{Bern}(\mparam).$$

\begin{itemize}
	\item Likelihood of $\mparam$ given observed $x$: $\ell(\mparam) = p(x | \mparam)$
\end{itemize}

\end{frame}

\begin{frame}{Example: coin flipping}

Imagine you'd like to predict the next coin flip result:
\begin{figure}
\centering
\includegraphics[width=0.6\linewidth]{figures-probmodel/coin_flips.png}
\end{figure}

Step 2: estimating $\mparam$ using probabilistic principles

Here we consider \emph{maximum likelihood estimation (MLE)}

Idea of MLE: for datapoints $x$ sampled from $\pi(x)$
\begin{itemize}
	\item We want to find $\bm{\theta}^*$ such that $p(x | \bm{\theta}^*) \approx \pi(x)$ \pause
	\only<2>{
	\item We need to measure the ``closeness'' of the two distributions $\Rightarrow$ use the KL divergence
	$$\mathrm{KL}[\pi(x) || p(x | \bm{\theta})] = \mathbb{E}_{\pi(x)} \left[ \log \frac{\pi(x)}{p(x | \bm{\theta})} \right]$$
	}
	\only<3>{
	\item We want this KL to be small:
	$$ \bm{\theta}^* = \arg\min_{\bm{\theta}} \mathrm{KL}[\pi(x) || p(x | \bm{\theta})] $$
	}
	\only<4>{
	$$ \Leftrightarrow \quad \bm{\theta}^* = \arg\max_{\bm{\theta}} \mathbb{E}_{\pi(x)}[\log p(x | \bm{\theta})] $$
	}
	\only<5>{
	\item Estimate using dataset $\mathcal{D} = \{x_1, ..., x_N \}$ sampled from $\pi(x)$: 
	$$ \bm{\theta}^* = \arg\max_{\bm{\theta}} \frac{1}{N} \sum_{n=1}^N \log p(x_n | \bm{\theta}) $$
	}
\end{itemize}

\end{frame}

\begin{frame}{Example: coin flipping}

Imagine you'd like to predict the next coin flip result:
\begin{figure}
\centering
\includegraphics[width=0.6\linewidth]{figures-probmodel/coin_flips.png}
\end{figure}

Step 2: estimating $\mparam$ using probabilistic principles

Here we consider \emph{maximum likelihood estimation (MLE)}

\begin{itemize}
	\item Estimate using dataset $\mathcal{D} = \{x_1, ..., x_N \}$ sampled from $\pi(x)$: 
	$$ \bm{\theta}^* = \arg\max_{\bm{\theta}} \frac{1}{N} \sum_{n=1}^N \log p(x_n | \bm{\theta}) $$
	\only<1>{
	\item model assumption: $p(x | \mparam) = \text{Bern}(\mparam)$
	$$ \Rightarrow \quad \bm{\theta}^* = \arg\max_{\bm{\theta}} \frac{1}{N} \sum_{n=1}^N x_n \log \mparam + (1 - x_n) \log (1 - \mparam)$$
	}
	\only<2>{
	\item solution by zeroing the gradient:
	$$ \frac{1}{N} \sum_{n=1}^N x_n \mparam^{-1} - (1 - x_n) (1 - \mparam)^{-1} = 0 \quad \Rightarrow \quad \mparam^* =  \frac{1}{N} \sum_{n=1}^N x_n$$
	}
\end{itemize}

\end{frame}

\begin{frame}{Example: coin flipping}

Imagine you'd like to predict the next coin flip result:
\begin{figure}
\centering
\includegraphics[width=0.6\linewidth]{figures-probmodel/coin_flips.png}
\end{figure}

Step 3: check if $p(x | \mparam^*)$ fits $\pi(x)$ well

(We assume the model has passed here)
\end{frame}

\begin{frame}{Example: coin flipping}

Imagine you'd like to predict the next coin flip result:
\begin{figure}
\centering
\includegraphics[width=0.6\linewidth]{figures-probmodel/coin_flips.png}
\end{figure}

Step 4: making prediction for next coin flip result using $p(x | \mparam^*)$

$$\mparam^* = \frac{1}{N} \sum_{x \in \mathcal{D}} x$$

\begin{equation*}
\Rightarrow \quad x_{N+1} = \begin{cases}
1, \quad \text{with probability } \frac{1}{N} \sum_{n=1}^N x_n \\
0, \quad \text{with probability } 1 - \frac{1}{N} \sum_{n=1}^N x_n
\end{cases}.
\end{equation*}

\end{frame}


%%%%%%%%%%

\begin{frame}{Principles of probabilistic modelling}

Datapoints $(\x, y)$ are sampled from an \emph{unknown ground truth distribution} $\pi(\x, y)$

Probabilistic modelling is about (in supervised learning case):
\begin{itemize}
	\item[1.] Assuming the output $y$ given $\x$ is sampled from 
	$$p(y | \x, \bm{\theta})$$
	\item[2.] estimating $\bm{\theta}$ under probabilistic principles \\ such as MLE, MAP, posterior inference
	\item[3.] check if $p(y | \x, \bm{\theta}^*)$ fits $\pi(y | \x)$ well, and repeat 1 - 3
	\item[4.] using $p(y | \x, \bm{\theta}^*)$ for predictions
\end{itemize}

\end{frame}

%%%% linear regression

\begin{frame}
\frametitle{Probabilistic modelling: linear regression}
\begin{minipage}{0.45\linewidth}
\begin{figure}
\centering
\includegraphics[width=0.7\linewidth]{figures-probmodel/linear_regression.png}
\end{figure}
$$\text{Linear regression}$$
$$f(\x, \mparam) = \mparam^\top \x ,$$
$$y = f(\x, \mparam) + \epsilon, \epsilon \sim \mathcal{N}(0, \sigma^2)$$
\end{minipage}
\only<2->{
\hfill $\Rightarrow$ \hfill
\begin{minipage}{0.45\linewidth}
\begin{figure}
\centering
\includegraphics[width=0.7\linewidth]{figures-probmodel/non_linear_regression.png}
\end{figure}
$$\text{Non-linear regression}$$
$$f(\x, \mparam) =  \mparam^\top \textcolor{red}{\phi(\x)}$$
$$y = f(\x, \mparam) + \epsilon, \epsilon \sim \mathcal{N}(0, \sigma^2)$$
\end{minipage}
}
\end{frame}

\begin{frame}{Probabilistic modelling: linear regression}

Step 1: making assumptions about the output generation process
$$y = \bm{\theta}^\top \phi(\x) + \epsilon, \quad \epsilon \sim \mathcal{N}(0, \sigma^2)$$
\vspace{-2em}
\begin{itemize}
	\item $\bm{\theta}$ is the model parameter
	\item $\phi(\x)$ is a pre-defined feature mapping (e.g., polynomial features)
\end{itemize} \pause
\vspace{1em}

Probabilistic formulation:
\begin{itemize}
	\item The distribution of $y$ given $\x$ under model assumption:
	$$p(y | \x, \bm{\theta}) = \mathcal{N}(\bm{\theta}^\top \phi(\x), \sigma^2)$$

	\item Likelihood of $\bm{\theta}$ given observed data $(\x, y)$:
	$$\ell(\bm{\theta}) = p(y | \x, \bm{\theta})$$
\end{itemize}

\end{frame}

\begin{frame}{Probabilistic modelling: linear regression}

Step 2: estimating $\bm{\theta}$ using \emph{maximum likelihood estimation (MLE)}

Idea of MLE: for datapoints $(\x, y)$ sampled from $\pi(\x, y)$
\begin{itemize}
	\item We want to find $\bm{\theta}^*$ such that $p(y | \x, \bm{\theta}^*) \approx \pi(y | \x)$ \pause
	\item We need to measure the ``closeness'' of the two distributions $\Rightarrow$ use the KL divergence
	$$\mathrm{KL}[\pi(y | \x) || p(y | \x, \bm{\theta})] = \mathbb{E}_{\pi(y | \x)} \left[ \log \frac{\pi(y | \x)}{p(y | \x, \bm{\theta})} \right]$$ \pause
	\item We want this KL to be small for all $\x$ sampled from $\pi(\x)$
	\only<3>{
	$$ \bm{\theta}^* = \arg\min_{\bm{\theta}} \mathbb{E}_{\pi(\x)}[\mathrm{KL}[\pi(y | \x) || p(y | \x, \bm{\theta})]] $$
	}
	\only<4>{
	$$ \Leftrightarrow \quad \bm{\theta}^* = \arg\max_{\bm{\theta}} \mathbb{E}_{\pi(\x, \y)}[\log p(y | \x, \bm{\theta})] $$
	}
	\only<5>{
	Estimate using dataset $\mathcal{D} = \{ (\x_n, y_n) \}_{n=1}^N$ from $\pi(\x, y)$: 
	$$ \quad \bm{\theta}^* = \arg\max_{\bm{\theta}} \frac{1}{N} \sum_{ (\x_n, y_n) \in \mathcal{D}} \log p(y_n | \x_n, \bm{\theta}) $$
	}
\end{itemize}
\end{frame}

\begin{frame}{Probabilistic modelling: linear regression}

Step 2: estimating $\bm{\theta}$ using \emph{maximum likelihood estimation (MLE)}

MLE: find $\bm{\theta}^*$ by 
$$ \bm{\theta}^* = \arg\max_{\bm{\theta}} \frac{1}{N} \sum_{ (\x_n, y_n) \in \mathcal{D}} \log p(y_n | \x_n, \bm{\theta}) $$

\begin{itemize}
	\item We assumed the probabilistic model to be
	$$p(y | \x, \bm{\theta}) = \mathcal{N}(\bm{\theta}^\top \phi(\x), \sigma^2)$$ \pause
\end{itemize}
\vspace{-2.5em}

\begin{equation*}
\begin{aligned}
\Rightarrow \bm{\theta}^* &= \arg\max_{\bm{\theta}} \frac{1}{N} \sum_{ (\x_n, y_n) \in \mathcal{D}} \log \mathcal{N}(\bm{\theta}^\top \phi(\x), \sigma^2) \pause \\ 
&= \arg\max_{\bm{\theta}} \frac{1}{N} \sum_{ (\x_n, y_n) \in \mathcal{D}} - \frac{1}{2\sigma^2} || y_n - \bm{\theta}^\top \phi(\x_n) ||_2^2 + \text{const} \pause \\
&= \arg\min_{\bm{\theta}} \frac{1}{N} \sum_{ (\x_n, y_n) \in \mathcal{D}} \frac{1}{2\sigma^2} || y_n - \bm{\theta}^\top \phi(\x_n) ||_2^2
\end{aligned}
\end{equation*}
\end{frame}

\begin{frame}{Probabilistic modelling: linear regression}
Step 2: estimating $\bm{\theta}$ using \emph{maximum likelihood estimation (MLE)}

$$\arg\min_{\bm{\theta}} \frac{1}{N} \sum_{ (\x_n, y_n) \in \mathcal{D}} \frac{1}{2\sigma^2} || y_n - \bm{\theta}^\top \phi(\x_n) ||_2^2
$$

Writing the objective in matrix form: $\bm{\Phi} = (\phi(x_1), ..., \phi(x_N))^\top, \y = (y_1, ..., y_N)^\top$
$$\mparam^* = \argmin_{\mparam} L(\mparam), \quad L(\mparam) = \frac{1}{2 \sigma^2} || \y - \bm{\Phi} \mparam ||_2^2 $$

\begin{itemize}
	\item Gradient of the loss $\nabla_{\mparam} L(\mparam)$: \pause
	$$\nabla_{\mparam} L(\mparam) = \frac{1}{\sigma^2}\bm{\Phi}^\top(\bm{\Phi} \mparam - \y)$$
	\item Setting $\nabla_{\mparam} L(\mparam) = 0$: \pause
	$$\Rightarrow \frac{1}{\sigma^2} \bm{\Phi}^\top \bm{\Phi} \mparam^* = \frac{1}{\sigma^2} \bm{\Phi}^\top \y \quad \textcolor{red}{\Rightarrow \mparam^* = (\bm{\Phi}^\top \bm{\Phi})^{-1} \bm{\Phi}^\top \y}$$
\end{itemize}

\end{frame}

\begin{frame}{Probabilistic modelling: linear regression}
Step 3: check if $p(y | \x, \bm{\theta}^*)$ fits $\pi(y | \x)$ well

Typical approaches:
\begin{itemize}
	\item Cross validation
	\item Model selection with marginal likelihood
\end{itemize} \pause

If model fit is bad:
\begin{itemize}
	\item Try another set of features $\phi'(\x) \neq \phi(\x)$
	\item Use other classes of models other than linear regression
\end{itemize}
\end{frame}

\begin{frame}{Probabilistic modelling: linear regression}
Step 4: using $p(y | \x, \mparam^*)$ to make predictions

Assume new test input $\x_{test}$:

$$\mparam^* = (\bm{\Phi}^\top \bm{\Phi})^{-1} \bm{\Phi}^\top \y$$

$$\Rightarrow \quad p(y_{test} | \x_{test}, \mparam^*) = \mathcal{N}( \y^{\top} \bm{\Phi} (\bm{\Phi}^\top \bm{\Phi})^{-1} \phi(\x_{test}), \sigma^2)$$

\end{frame}

%%%%%% logistic regression

\begin{frame}{Probabilistic modelling: logistic regression}

Step 1: making assumptions about the output generation process
\begin{equation*}
y = \begin{cases}
1, \quad \text{with probability } \rho \\
0, \quad \text{with probability } 1 - \rho
\end{cases}, 
\quad \rho = sigmoid(\bm{\theta}^\top \phi(\x))
\end{equation*}

Probabilistic formulation:
\begin{itemize}
	\item The distribution of $y$ given $\x$ under model assumption:
	$$p(y | \x, \bm{\theta}) = \text{Bern}(sigmoid(\bm{\theta}^\top \phi(\x)))$$
\end{itemize}
\end{frame}

\begin{frame}{Probabilistic modelling: linear regression}

Step 2: estimating $\bm{\theta}$ using \emph{maximum likelihood estimation (MLE)}

MLE: find $\bm{\theta}^*$ by 
$$ \bm{\theta}^* = \arg\max_{\bm{\theta}} \frac{1}{| \mathcal{D} |} \sum_{ (\x, \y) \in \mathcal{D}} \log p(y | \x, \bm{\theta}) $$

\begin{itemize}
	\item We assumed the probabilistic model to be
	$$p(y | \x, \bm{\theta}) = \text{Bern}(sigmoid(\bm{\theta}^\top \phi(\x)))$$ \pause
\end{itemize}
\vspace{-2.5em}

\begin{equation*}
\begin{aligned}
\Rightarrow \bm{\theta}^* = \arg\max_{\bm{\theta}} \frac{1}{| \mathcal{D} |} \sum_{ (\x, \y) \in \mathcal{D}} &\log \text{Bern}(sigmoid(\bm{\theta}^\top \phi(\x))) \pause \\ 
= \arg\max_{\bm{\theta}} \frac{1}{| \mathcal{D} |} \sum_{ (\x, \y) \in \mathcal{D}} & y \log \hat{y}(\x; \bm{\theta}) + (1 - y) \log (1 - \hat{y}(\x; \bm{\theta})), \\
& \hat{y}(\x; \bm{\theta}) = sigmoid(\bm{\theta}^\top \phi(\x))
\end{aligned}
\end{equation*}
\end{frame}


\begin{frame}{Probabilistic modelling: linear regression}
Step 2: estimating $\bm{\theta}$ using \emph{maximum likelihood estimation (MLE)}
$$\arg\max_{\bm{\theta}} L(\mparam), \quad L(\mparam) = \frac{1}{| \mathcal{D} |} \sum_{ (\x, \y) \in \mathcal{D}} y \log \hat{y}(\x; \bm{\theta}) + (1 - y) \log (1 - \hat{y}(\x; \bm{\theta})), $$
$$\hat{y}(\x; \bm{\theta}) = sigmoid(\bm{\theta}^\top \phi(\x))$$

\begin{itemize}
	\item Gradient of the loss $\nabla_{\mparam} L(\mparam)$:
	$$ \nabla_{\mparam} L(\mparam) = \frac{1}{| \mathcal{D} |} \sum_{ (\x, \y) \in \mathcal{D}} [ y - \hat{y}(\x; \bm{\theta})] \phi(\x) $$
\end{itemize}
No analytic solutions!
\end{frame}

\begin{frame}{Gradient descent based optimisation}
Algorithm: Gradient Descent (gradient \emph{ascent} in MLE case) \\
Define \emph{starting point} $\mparam_0$, sequence of \emph{step sizes} $\gamma_t$, set $t\gets 0$.
\begin{enumerate}
\item Set $\mparam_{t+1} = \mparam_t + \gamma_t \nabla_{\mparam} L(\mparam_t)$, $t \leftarrow t+1$
\item Repeat 1 until stopping criterion.
\end{enumerate}
  \begin{figure}
    \centering
    \includegraphics[width = 0.7\hsize]{./figures-probmodel/gd-contour.png}
  \end{figure}
\end{frame}

\begin{frame}{Probabilistic modelling: logistic regression}
Step 3: check if $p(y | \x, \bm{\theta}^*)$ fits $\pi(y | \x)$ well

Typical approaches:
\begin{itemize}
	\item Cross validation
	\item Model selection with marginal likelihood
\end{itemize} 

If model fit is bad:
\begin{itemize}
	\item Try another set of features $\phi'(\x) \neq \phi(\x)$
	\item Use other classes of models other than logistic regression
\end{itemize}
\end{frame}

\begin{frame}{Probabilistic modelling: logistic regression}
Step 4: using $p(y | \x, \mparam^*)$ to make predictions

Assume new test input $\x_{test}$:

$\mparam^*$ obtained by gradient descent

$$\Rightarrow \quad p(y_{test} | \x_{test}, \mparam^*) = \text{Bern}(Sigmoid((\mparam^*)^\top \phi(\x_{test})))$$

\end{frame}

\begin{frame}{Probabilistic modelling \& MLE: summary}
Have you ever wondered about the following questions:
\begin{itemize}
	\item Why using $\ell_2$ loss in many regression problems? \\
	\alert{A:} We assume the model to be $p(y | \x, \bm{\theta}) = \mathcal{N}(\bm{\theta}^\top \phi(\x), \sigma^2)$,  \\ and fit $\mparam$ using MLE \pause
	
	\item Where does the cross-entropy loss come from? \\
	\alert{A:} It comes from MLE, and in binary classification using $p(y | \x, \bm{\theta}) = \text{Bern}(sigmoid(\bm{\theta}^\top \phi(\x)))$ \pause
	
	\item What is a good principle for choosing a good loss function? \\
	\alert{A:} Build a probabilistic model for the data generation process, and fit the parameters using MLE (or MAP, posterior inference)
\end{itemize}

\end{frame}

\begin{frame}{Exercises}

Finish relevant exercises in the exercise sheet
\begin{itemize}
	\item You should be able to derive MLE objectives from probabilistic model assumptions, and vice versa
\end{itemize}

\vspace{1em}

Next lecture: convergence of gradient descent

Pre-requisite knowledge: Eigen-decomposition

(See e.g., \url{https://youtu.be/xgZ8oK9Wxzg} or search relevant videos from e.g., 3Blue1Brown)

\end{frame}



\end{document}
%%% Local Variables: 
%%% mode: latex
%%% TeX-master: t
%%% End: 
