%% Time-stamp: <2018-10-18 20:24:12 (marc)>
\documentclass[xcolor=x11names,compress,mathserif,handout]{beamer}

\newcommand{\hackspace}{\hspace{4.2mm}}
\newcommand{\showstudent}[1]{}
\newcommand\hmmax{0}
\newcommand\bmmax{0}





% talk/author information
\newcommand{\authorname}{Mark van der Wilk}
\newcommand{\authoremail}{m.vdwilk@imperial.ac.uk}
\newcommand{\authortwitter}{markvanderwilk}
\newcommand{\authoraffiliation}{
  Department of Computing\\Imperial
  College London}
\newcommand{\slidesettitle}{\imperialBlue{Theme: Curve Fitting}}
\newcommand{\footertitle}{Course Overview}
\newcommand{\location}{Imperial College London}
\newcommand{\talkDate}{October 3, 2022}



\date{\imperialGray{\talkDate}}




% load defaults
\usepackage{../includes/MarkMathCmds}
\selectcolormodel{rgb}
\usepackage{ifxetex,ifluatex}
\newif\ifxetexorluatex
\ifxetex
  \xetexorluatextrue
\else
  \ifluatex
    \xetexorluatextrue
  \else
    \xetexorluatexfalse
  \fi
\fi

\usepackage{textpos}
%\usepackage{arabtex}
\usepackage{tikz}
\usetikzlibrary{decorations.markings}
\usetikzlibrary{arrows}
\usetikzlibrary{shapes}
\usetikzlibrary{plotmarks}
\usetikzlibrary{mindmap,trees,backgrounds}

\tikzstyle{every picture}+=[remember picture]

%\usepackage{movie15}
% \usepackage{pdfpages}
%\usepackage{xmpmulti}

\usepackage{anyfontsize}
\usepackage{wrapfig}
\usepackage{animate}
\usepackage{multirow}
\usepackage{multimedia}
\usepackage{xmpmulti}
%\usepackage[latin9]{inputenc}
\usepackage[english]{babel}
\usepackage{scalefnt}
\usepackage{verbatim}
\usepackage{url}
% \usepackage{pgf,pgfarrows,pgfnodes}
\usepackage{textpos}
\usepackage[tight,ugly]{units}
\usepackage{url}
\usepackage{bbm}
\usepackage[english]{babel}
\usepackage{fancyhdr}
\usepackage{bm} % correct bold symbols, like \bm
\usepackage{amsmath}
\usepackage{amsfonts}
\usepackage{amssymb}
\usepackage{mathrsfs}
\usepackage{mathtools}
\usepackage{color}
\usepackage{cancel}
\usepackage{algorithm}
\usepackage{algpseudocode}
\usepackage{mathrsfs}
\usepackage{listings}
\usepackage{graphicx} % for pdf, bitmapped graphics files
\usepackage{mathtools}
\usepackage{units}
\usepackage{subfig}
\usepackage{enumerate}
\usepackage{natbib}
\usepackage{dsfont}


\ifxetexorluatex
\usepackage{fontspec}
\setmainfont[Scale=0.8]{OpenDyslexic-Regular}
\else
\usefonttheme{professionalfonts}
\fi

\renewcommand{\vec}[1]{{\boldsymbol{{#1}}}} % vector
\newcommand{\mat}[1]{{\boldsymbol{{#1}}}} % matrix
% \newcommand{\KL}[2]{\mathrm{KL}(#1\|#2)} % KL divergence
\newcommand{\R}[0]{\mathds{R}} % real numbers
\newcommand{\Z}[0]{\mathds{Z}} % integers
\newcommand{\tr}[0]{\text{tr}} % trace
% \newcommand{\inv}{^{-1}}
% \DeclareMathOperator*{\diag}{diag}
\newcommand{\E}{\mathds{E}} % expectation
\newcommand{\var}{\mathds{V}}
\newcommand{\gauss}[2]{\mathcal{N}\big(#1,\,#2\big)}
\newcommand{\gaussx}[3]{\mathcal{N}\big(#1\,|\,#2,\,#3\big)}
\newcommand{\gaussBig}[2]{\mathcal{N}\left(#1,\,#2\right)}
\newcommand{\gaussxBig}[3]{\mathcal{N}\left(#1\,\left|\,#2,\,#3\right.\right)}
\newcommand{\Ber}[0]{\mathrm{Ber}} % Bernoulli distribution
\DeclareMathOperator{\cov}{Cov}
\ifxetexorluatex
\renewcommand{\T}[0]{^\top}
\renewcommand{\d}[0]{\text{d}} % derivative
\else
\newcommand{\T}[0]{^\top}
\renewcommand{\d}[0]{\text{d}} % derivative
\fi
% calculus
\newcommand{\pdiff}[1]{\frac{\partial}{\partial #1}}
\newcommand{\pdiffF}[2]{\frac{\partial #1}{\partial #2}}
\newcommand{\diffF}[2]{\frac{{\d}#1}{{\d}#2}}
\newcommand{\diffFII}[2]{\frac{{\d}^2 #1}{{\d}#2^2}}
\newcommand{\diff}[1]{\frac{{\d}}{{\d}#1}}
\newcommand{\diffII}[1]{\frac{{\d}^2}{{\d}#1^2}}
\newcommand{\class}[0]{\mathcal{C}}

\newcommand{\idx}[1]{{(#1)}}
% \newcommand{\norm}[1]{\left\|#1\right\|}
\newcommand{\proj}[1]{\tilde{#1}}
\newcommand{\pcacoord}{z}
\newcommand{\pcacoordnew}{\zeta}
\newcommand{\latent}{z}
% \newcommand{\given}{\,|\,}
\newcommand{\genset}[1]{\mathrm{span}[#1]} % generating set
\newcommand{\set}[1]{\mathcal{#1}} % set
\newcommand{\fixgmfont}[1]{\scalebox{0.8}{#1}}



\usepackage{pifont}% http://ctan.org/pkg/pifont
\newcommand{\cmark}{{\color{green!40!black}\ding{51}}}%
\newcommand{\xmark}{{\color{red}\ding{55}}}%
\newcommand{\green}[1]{{\bf{\textcolor{green}{#1}}}}
\newcommand{\red}[1]{{\bf{\textcolor{red}{#1}}}}

\newcommand<>\red[1]{{\color#2[rgb]{1,0,0}#1}}
\newcommand<>\blue[1]{{\color#2[rgb]{0,0,1}#1}}
\newcommand<>\yellow[1]{{\color#2{camyellow}#1}}
\newcommand<>\green[1]{{\color#2[rgb]{0,0.6,0.0}#1}}
\newcommand<>\violet[1]{{\color#2[rgb]{0.6,0,0.6}#1}}
\newcommand<>\orange[1]{{\color#2[rgb]{1,0.5,0}#1}}
\newcommand<>\black[1]{{\color#2[rgb]{0,0,0}#1}}
\newcommand<>\steel[1]{{\color#2[rgb]{0,0,0.8}#1}}
\newcommand<>\darkblue[1]{{\color#2[rgb]{0,0,0.6}#1}}
\newcommand<>\lightblue[1]{{\color#2[rgb]{0.4,0.4,0.7}#1}}
\newcommand<>\gray[1]{{\color#2[rgb]{0.4,0.4,0.4}#1}}
\newcommand<>\greenish[1]{{\color#2[rgb]{0.45, 0.66, 0.45}#1}}
\newcommand<>\redish[1]{{\color#2[rgb]{0.7843    0.3706    0.3706}#1}}
\definecolor{redishTIKZ}{rgb}{0.7843, 0.3706, 0.3706}
\definecolor{imperialBlue}{rgb}{0.058, 0.219, 0.418}
\definecolor{aimsbrown}{rgb}{0.539, 0.117, 0.015}
% \definecolor{imperialGray}{rgb}{0.414, 0.488, 0.671 }
\definecolor{imperialGray}{RGB}{109,153, 204}
\definecolor{aimslightbrown}{RGB}{138,88,84}
\newcommand<>\imperialBlue[1]{{\color#2[rgb]{0.058, 0.219, 0.418}#1}}
\newcommand<>\aimsbrown[1]{{\color#2[rgb]{0.539, 0.117, 0.015}#1}}
%\newcommand<>\imperialGray[1]{{\color#2[rgb]{0.414, 0.488, 0.671}#1}}
\newcommand<>\imperialGray[1]{{\color#2[RGB]{109,153, 204}#1}}
\newcommand<>\aimslightbrown[1]{{\color#2[RGB]{138,88,84}#1}}
\newcommand<>\lightgray[1]{{\color#2[rgb]{0.8,0.8,0.8}#1}}
%\newcommand<>\highlightcolor[1]{{\color#2[rgb]{0,0,1}#1}}
\newcommand{\highlight}[1]{{\bf\steel{#1}}}
%\newcommand{\newblock}[0]{}

%\newcommand{\arrow}[0]{\includegraphics[height=5pt]{./figures/arrow}\hspace{3pt}}

\renewcommand{\emph}[1]{\textbf{\steel{{#1}}}}

\renewcommand{\alert}[1]{{\bf\red{{#1}}}}

\newcommand{\arrow}{
\begin{tikzpicture}
\draw [black!40!green, fill=black!40!green] (0,-0.12) -- (0,0.12) --
(0.15,0);
\draw [black!40!green, fill=black!40!green] (0.15,-0.12) -- (0.15,0.12) --
(0.3,0); 
\end{tikzpicture}
}

\geometry{left=0.45cm,top=0cm,right=0.45cm}


\newcommand{\logoimagepath}{./figures/imperial}
\newcommand{\highlightcolor}{blue!80!black}
%\newcommand{\headbarcolor}{imperialBlue}
\newcommand{\headbarcolor}{imperialBlue}
\institute{}

\newcommand{\coursetitle}{}

\newcommand{\slidesetsubtitle}{}
\newcommand{\slidesetnumber}{01}
\usefonttheme{professionalfonts}


\usetikzlibrary{decorations.fractals}
\input{../includes/tikzlibrarybayesnet.code.tex}
\input{../includes/tikzlibraryipe.code.tex}
\usetikzlibrary{matrix,positioning,decorations.pathreplacing}
\usetikzlibrary{calc,quotes,angles}
\usetikzlibrary{arrows, arrows.meta, patterns}

\usetikzlibrary{decorations.pathreplacing}
\tikzset{
    position label/.style={
       above = 3pt,
       text height = 2ex,
       text depth = 1ex
    }
}

% \usetikzlibrary{decorations.markings}
\tikzset{
  font={\fontsize{14pt}{12}\selectfont}
}



\useoutertheme[subsection=false,shadow]{miniframes}
\useinnertheme{default}
\usefonttheme{serif}
%\usepackage{palatino}
\usepackage{mathpazo}
%\usepackage{utopia}
\usepackage{stmaryrd} % for varodot, bigodot 
\usepackage{mathabx} % for \coAsterisk
%\usepackage{mnsymbol}
%\setbeamertemplate{itemize item}{\scriptsize\raise1.7pt\hbox{\donotcoloroutermaths$\Asterisk$}}
%\setbeamertemplate{itemize item}{\scriptsize\raise1.7pt\hbox{\donotcoloroutermaths$\varodot$}}
%\setbeamertemplate{itemize subitem}{\scriptsize\raise1.25pt\hbox{\donotcoloroutermaths$\rhd$}}

\usepackage{xifthen}% provides \isempty tesst

\setbeamerfont{title like}{shape=\scshape}
\setbeamerfont{frametitle}{}



\setbeamercolor*{lower separation line head}{bg=blue} 
\setbeamercolor*{normal text}{fg=black,bg=white} 
\setbeamercolor*{alerted text}{fg=red} 
\setbeamercolor*{example text}{fg=black} 
%\setbeamercolor*{frametitle}{fg=aimsbrown} 
\setbeamercolor*{frametitle}{fg=imperialBlue} 
\setbeamercolor*{structure}{fg=black} 
 
\setbeamercolor*{palette tertiary}{fg=black,bg=black!10} 
\setbeamercolor*{palette quaternary}{fg=black,bg=black!10} 

%\renewcommand{\(}{\begin{columns}}
%\renewcommand{\)}{\end{columns}}
%\newcommand{\<}[1]{\begin{column}{#1}}
%\renewcommand{\>}{\end{column}}

% ======================================
% custom commands 
\newcommand{\cemph}[1]{\textcolor{\highlightcolor}{#1}}
\newcommand{\calert}[1]{\textcolor{red}{#1}}

\setbeamertemplate{navigation symbols}{}
%\renewcommand\frametitle[1]{{\textsc{\Large \textcolor{\highlightcolor}{#1}}}\vspace{0.6cm}\par}

\setbeamertemplate{frametitle}
{
{\textsc\bf \insertframetitle}\vspace{0.2cm}\par
}


%%%%%%%%%%%%%%%%%%%%%%%%%%%%%%%%%%%%%%%%%%%%%%%%%%
\setbeamertemplate{headline}{% 
	\setbeamercolor{head1}{bg=\headbarcolor}
	 \hbox{%
  \begin{beamercolorbox}[wd=.01\paperwidth,ht=2.25ex,dp=50ex,center]{head1}%
  \fontsize{5}{5}\selectfont  
  \end{beamercolorbox}%
  }
  \vspace{-50ex}
}
\setbeamertemplate{footline}{
\begin{tiny}
\setbeamercolor{foot1}{fg=black,bg=gray!10}
\setbeamercolor{foot2}{fg=gray,bg=gray!15}
\setbeamercolor{foot3}{fg=gray,bg=gray!10}
\setbeamercolor{foot4}{fg=black,bg=gray!20}
\setbeamercolor{foot5}{fg=gray,bg=gray!15}
\setbeamercolor{foot6}{fg=black,bg=gray!20}

% taken from theme infolines and adapted
  \leavevmode%
  \hbox{%
  \begin{beamercolorbox}[wd=.45\paperwidth,ht=2.25ex,dp=1ex,center]{foot1}%
  \fontsize{5}{5}\selectfont
  \flushleft \hspace*{2ex}{\footertitle}
  \end{beamercolorbox}%
  % \begin{beamercolorbox}[wd=.08\paperwidth,ht=2.25ex,dp=1ex,center]{foot2}
  % \end{beamercolorbox}%
  %   \begin{beamercolorbox}[wd=.05\paperwidth,ht=2.25ex,dp=1ex,center]{foot3}
  % \end{beamercolorbox}%
    \begin{beamercolorbox}[wd=.45\paperwidth,ht=2.25ex,dp=1ex,center]{foot4}%
  \fontsize{5}{5}\selectfont
  \authorname\hspace{5mm}@\location, \talkDate%\ (\authorweb) 
  \end{beamercolorbox}%
  % \begin{beamercolorbox}[wd=.05\paperwidth,ht=2.25ex,dp=1ex,center]{foot5}
  % \end{beamercolorbox}%
  \begin{beamercolorbox}[wd=.1\paperwidth,ht=2.25ex,dp=1ex,right]{foot6}%
	\insertframenumber{}  \hspace*{2ex} 
  \end{beamercolorbox}}%
  \vskip0pt%
\end{tiny}
\vskip0pt
}


\setbeamertemplate{blocks}[rounded][shadow=false]


\newenvironment<>{myblock}[1]{%
  \begin{actionenv}#2%
      \def\insertblocktitle{#1}%
      \par%
      \mode<presentation>{%
%       \setbeamercolor{block title}{fg=black,bg=aimslightbrown!50!white}
      \setbeamercolor{block title}{fg=black,bg=imperialBlue!45!white}
       \setbeamercolor{block body}{fg=black,bg=gray!20}
       \setbeamercolor{itemize item}{fg=blue!40!white}
       \setbeamertemplate{itemize item}[triangle]
     }%
      \usebeamertemplate{block begin}}
    {\par\usebeamertemplate{block end}\end{actionenv}}

\newenvironment<>{myblock2}[1]{%
  \begin{actionenv}#2%
      \def\insertblocktitle{#1}%
      \par%
      \mode<presentation>{%
       \setbeamercolor{block title}{fg=white,bg=blue!80!black}
       \setbeamercolor{block body}{fg=black,bg=gray!20}
       \setbeamercolor{itemize item}{fg=green!60!black}
       \setbeamertemplate{itemize item}[triangle]
     }%
      \usebeamertemplate{block begin}}
    {\par\usebeamertemplate{block end}\end{actionenv}}

\gdef\colchar#1#2{%
  \tikz[baseline]{%
%  \node[anchor=base,inner sep=2pt,outer sep=0pt,fill = #2!20]
%  {\large{#1}};
  \node[anchor=base,inner sep=1pt,outer sep=0pt,fill = #2!20]
  {{\fontsize{11}{13}\selectfont #1}};
    }%
}%
\gdef\drawfontframe#1#2{%
  \tikz[baseline]{%
  \node[anchor=base,inner sep=2pt,outer sep=0pt,fill = #2!20] {#1};
    }%
  }%


\makeatletter
\let\@@magyar@captionfix\relax
\makeatother

%%% Local Variables:
%%% mode: latex
%%% TeX-master: "2018-09-arusha-linear-regression"
%%% End:






\newif\iflattersubsect

\AtBeginSection[] {
    \begin{frame}<beamer>
    \frametitle{Overview} %
    \tableofcontents[currentsection]  
    \end{frame}
    \lattersubsectfalse
}

\AtBeginSubsection[] {
    \iflattersubsect
    \begin{frame}<Coming Next>
    \frametitle{Overview} %
    \tableofcontents[currentsubsection]  
    \end{frame}
    \fi
    \lattersubsecttrue
}

\begin{document}


%%%%%%%%%%%%%%%%%%%%%%%%%%%%%%%%%%%%%%%%%%%%%%%%%%%%%%

{\setbeamertemplate{footline}{}
\begin{frame}
\title{\slidesettitle}
%\subtitle{SUBTITLE}
\author{\footnotesize
  \textbf{\authorname}
 }

 %%% LOGO

% \begin{flushright}
%   % \begin{columns}
%   %   \column{0.5\hsize}
%   %   \column{0.45\hsize}
%\includegraphics[height = 8mm]{./figures/qla}\hspace{2mm}
%     \includegraphics[height = 8mm]{./figures/aims-rwanda}\\[2mm]
%\includegraphics[height = 8mm]{./figures/imperial}
%%\end{columns}
%\end{flushright}

\vspace{-0cm}
%\begin{flushleft}
%\vspace{-1.5cm}{\small \textcolor{blue}{\coursetitle}}\\\vspace{2cm}
{\huge \slidesettitle \ifthenelse{\equal{\slidesetsubtitle}{}}%
    {}% if #1 is empty
    {: \\ {\large \slidesetsubtitle}}% if #1 is not empty
    } \\    
    %\vspace{20pt}
%\end{flushleft}
  
 
% this is all stuff below the talk title. make two columns, just in
% case you want to have a picture or a second affiliation here 
\begin{columns}[t]
\column{0.8\hsize}
%\begin{flushleft}
\begin{columns}[t]
\column{0.6\hsize}
\insertauthor \\[2mm]
\authoraffiliation\\[2mm]
\column{0.25\hsize}
\\[2mm]
\includegraphics[height = 0.3cm]{./figures-general/twitter}{\small @\authortwitter}\\[-1mm]
\mbox{\small \url{\authoremail}}
\end{columns}
\column{0.14\hsize}
\end{columns}
% \authorweb\\
\vspace{7mm}
% \aimslightbrown{The Nelson Mandela African Institute of Science and
%   Technology\\Arusha, Tanzania}\\[2mm]
\insertdate
%\end{flushleft}
\end{frame}
}

%%% Local Variables:
%%% mode: latex
%%% TeX-master: t
%%% End:

\linespread{1.2}


\section{What is regression?}




\begin{frame}{Curve Fitting (Regression) Examples}
% We will consider curve fitting, supervised learning


We will be considering \textit{curve fitting} or \textit{supervised learning}.
\begin{itemize}
\item Given a dataset of $N$ examples of inputs and outputs...
\item predict what the output will be for a new input.
\end{itemize}

\pause


\vspace{0.7cm}

\emph{Image classification}. Inputs $\in \mathbb{R}^D$, outputs $\in \mathbb{N}$:
  \begin{figure}
    \centering
    \includegraphics[width = 1.0\hsize]{./figures-intro-curvefitting/invariance-examples}
  \end{figure}

\pause
\vspace{0.2cm}

\emph{Translation}. Inputs $\in \bigcup_{\ell=1}^\infty \mathbb{N}^\ell$, outputs $\in \bigcup_{k=1}^\infty \mathbb{N}^k$:
\vspace{0.3cm}

\texttt{Wiskunde is belangrijk.}  $\qquad\to\qquad$ \texttt{Mathematics is important.} \\
\hspace{1.7cm}Dutch\hspace{5.7cm}English
\end{frame}



\begin{frame}{Regression Example}
% Mathematically these examples are conceptually similar to curve fitting in 1D
Curve fitting in 1D. Inputs $\in \mathbb{R}$, outputs $\in \mathbb{R}$:
  \begin{figure}
    \centering
    \includegraphics[width = 0.5\hsize]{./figures-intro-curvefitting/polynomial5}
  \end{figure}  
\end{frame}





\begin{frame}{Curve fitting}
\begin{center}
{\Large \textit{``All the impressive achievements of deep learning \\ amount to just curve fitting.''}} \\
--- Judea Pearl
\end{center}

% Now, I'm not the only person who says that there is a relationship between the simple 1D curve fitting
% from the previous slide, and complicated applications like machine translation.
% Judea Pearl, a Turing award winner, even went so far to say that 

% "All the impressive achievements of deep learning amount to just curve fitting."

% What is interesting about this quote, is that 
% Professor Pearl made this remark in the context of describing things that are wrong about deep learning.
% One argument he makes, is that curve fitting, and therefore deep learning, does not give our Artifically Intelligent systems
%  an idea about _causality_, or _why_ the output is what it is, in response to the input.
% However, he also notes that the achievements are impressive. And I certainly think that the achievements
% of deep learning over the last few years have been impressive. And if that's all done using curve fitting,
% then we better understand curve fitting really well.

\end{frame}


% \begin{frame}
%   \frametitle{Curve Fitting: Mathematical Description}

%   \begin{columns}[t]

%     \column{0.65\hsize}
%     \begin{center}
%       \Large Big Question: How do we describe the problem of curve fitting mathematically?
%     \end{center}
%     \column{0.3\hsize}
  
%   \begin{figure}
%     \centering
%     \includegraphics[width = \hsize]{./figures-intro-curvefitting/polynomial5}
%   \end{figure}
  
  
% \end{columns}

% \pause

% % Let's start by examining our assumptions
% \begin{itemize}
% \item Each input is associated with a single output. \pause
% \item Equivalent mathematical notion: A \emph{function}. \pause
% \end{itemize}

% So...
% \begin{itemize}
% \item Given a dataset of $N$ input-output pairs $\{(\vec x_n, y_n)\}_{n=1}^N$... \\
% where $\vec x_n \in \mathcal{X}$ (usually $\mathbb R^D$), $y_n \in \mathcal{Y}$ (in our case, usually $\Reals$) \pause
% \item Find a function $f: \mathcal{X} \to \mathcal{Y}$ that predicts well.
% \end{itemize}
% \end{frame}


\section{Representing Functions}
\begin{frame}
  \frametitle{Curve Fitting: Representing functions}
  \vspace{-0.8cm}
  \begin{columns}[t]

    \column{0.65\hsize}
    \vspace{0.4cm}
    \begin{center}
      \Large Q: How do we represent functions?
    \end{center}
    \column{0.3\hsize}
  
  \begin{figure}
    \centering
    \includegraphics[width = \hsize]{./figures-intro-curvefitting/polynomial5}
  \end{figure}
  
  
\end{columns}

\pause

\begin{itemize}
\item We need a \textit{collection} of functions from which to pick a good one. \pause
\item \emph{Parameterise} a set of functions, i.e.~take some numbers $\vec \theta$ that map to a function. \pause
\end{itemize}

\vspace{0.3cm}

For example, linear or polynomial functions:
\begin{align}
f_{\vec \theta}(x) &= a\cdot x + b\,, && \vec\theta = \begin{bmatrix}a \\ b\end{bmatrix} \,, \\
f_{\vec \theta}(x) &= a\cdot x^3 + b\cdot x^2 + c\cdot x + d\,, && \vec\theta = \begin{bmatrix}a && b && c && d\end{bmatrix}^{\tiny\mathrm T} \,.
\end{align}
% Neural networks are also parameterised functions, but complicated ones.
\end{frame}


\begin{frame}{Linear-in-the-Parameters representation}
\begin{align}
f_{\vtheta}(x) = \vphi(x)\transpose\vtheta \\
\vphi(x) && \text{Feature vector} \\
\vtheta && \text{Parameters}
\end{align}

\begin{itemize}
\item Can use this to represent complicated functions.
\item We call this ``linear in the parameters'' because the relationship betwen the function values and the parameters is linear.
\item This enables regressio solutions to be found in closed form.
\end{itemize}
\end{frame}



\section{Regression as Minimising a Loss}



\begin{frame}[t]{Regression Example}
  \begin{figure}
    \centering
    \includegraphics[width=1.0\hsize]{./resonance-example/resonance-regression.png}
  \end{figure}
\vspace{-0.3cm}
For some observed $x$, the world is generating data from $\pi(y|x)$.

We can choose two possible goals for regression:
\begin{itemize}
\item Loss view: Find a function $f(x)$ that goes ``near'' outputs $y$.
\item Stats view: Match a statistical model $p(y|x,\vtheta)$ to $\pi(y|x)$.
\end{itemize}
\end{frame}




\begin{frame}{Loss view: Good and bad functions}
We now have many functions that we can choose from:
  \begin{figure}
    \centering
    \includegraphics[width = \hsize]{./figures-intro-curvefitting/linear-regression}
    {\tiny Left: example functions. Middle: Training set. Right: A good fit.$\qquad$ Source: Mathematics for Machine Learning book.}
  \end{figure} \pause

    \begin{center}
      \Large Q: Which function do we pick?
    \end{center}
\pause

\begin{itemize}
\item Need to define what good and bad functions are. Good functions have $f(\vec x_i,\vec\theta^*)\approx y_i$. \pause
\item Define a \cemph{loss function}, e.g., $L(\vec\theta) = \sum_{i=1}^N(y_i - f(\vec x_i,
   \vec\theta))^2$ \pause
\item Choose a good function, i.e.~${\vec \theta}^* = \argmin_{\vec\theta} L(\vec\theta)$
\end{itemize}

\end{frame}











\section{A Statistical View on Regression}

\begin{frame}{Maximum Likelihood Estimation}
Revision from \texttt{50008: Probability \& Statistics}
\begin{itemize}
\item Model is a probability distribution on data: $p(y|\vtheta)$
\item For an observed dataset (fixed), we can evaluate the probability assigned to it for different $\vtheta$
\item This defines the likelihood $\ell(\vtheta) = p(y|\vtheta)$
\end{itemize} \pause

\vspace{0.3cm}

Maximum likelihood does:
\begin{align}
\vtheta^* = \argmax_{\vtheta} \ell(\vtheta) = \argmax_{\vtheta} \log \ell(\vtheta)
\end{align}

\end{frame}


\begin{frame}{Likelihood for Linear Regression}
Assume:
\begin{itemize}
\item Gaussian deviations from the function:
\begin{align}
p(y_n|x_n, \vtheta) = \NormDist{y_n; f_{\vtheta}(x_n), \sigma^2}
\end{align}
\item Independent deviations between datapoints. So denoting $y\in\Reals^N$, $x\in\Reals^N$ for $N$ datapoints, we get the likelihood:
\begin{align}
p(y|x,\vtheta) = \prod_{n=1}^N\NormDist{y_n; f_{\vtheta}(x_n), \sigma^2}
\end{align}
\item You will show that this is equivalent to the loss view (exercises).
\end{itemize}
\end{frame}





\section{Conclusion}





%%%%%%%%%%%%%%%%%%%%%%%%%%%%%%%%%%%%%%%%%
\begin{frame}
  \frametitle{Curve Fitting Summary}

%   \begin{columns}[t]

%     \column{0.65\hsize}
%      \begin{itemize}
%      \item Training data, e.g., $N$ pairs $(\vec x_i, y_i)$ of inputs
%       $\vec x_i$ and observations $y_i$
%      \item \cemph{Parameterise} functions as $f(\vx_i, \vec\theta)$
%   \item \cemph{Training the model} means finding parameters $\vec\theta^*$, such
%     that
%     \begin{itemize}
%       \item $f(\vec x_i,\vec\theta^*)\approx y_i$ (loss is minimised)
%       \item $p(y_n|x_n,\vtheta) \approx \pi(y_n|x_n,\vtheta)$ (max likelihood)
%     \end{itemize} \pause
%     \item Not discussed: How to find $\vtheta^*$
%   \end{itemize}
%     \column{0.3\hsize}
  
%   \begin{figure}
%     \centering
%     \includegraphics[width = \hsize]{./figures-intro-curvefitting/polynomial5}
%   \end{figure}
% \end{columns}

     \begin{itemize}
     \item Training data, e.g., $N$ pairs $(\vec x_i, y_i)$ of inputs
      $\vec x_i$ and observations $y_i$
     \item \cemph{Parameterise} functions as $f(\vx_i, \vec\theta)$
  \item \cemph{Training the model} means finding parameters $\vec\theta^*$, such
    that
    \begin{itemize}
      \item $f(\vec x_i,\vec\theta^*)\approx y_i$ (loss is minimised)
      \item $p(y_n|x_n,\vtheta) \approx \pi(y_n|x_n,\vtheta)$ (max likelihood)
    \end{itemize} \pause
    \item Not discussed: How to find $\vtheta^*$
  \end{itemize}


\end{frame}





\end{document}
%%% Local Variables: 
%%% mode: latex
%%% TeX-master: t
%%% End: 
